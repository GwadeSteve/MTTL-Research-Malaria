\chapter*{Introduction}
\addcontentsline{toc}{chapter}{Introduction}
\label{chap:intro}

Malaria remains a persistent global public health challenge. Transmitted by infected female \textit{Anopheles mosquitoes}, this parasitic disease places a disproportionate burden on low and middle-income countries, particularly in sub-Saharan Africa. The 2024 \textit{World Malaria Report} estimates there were nearly \textbf{263 million} cases and over \textbf{597,000 deaths} in 2023, with children under five and pregnant women bearing the highest risk \parencite{WHO2024GlobalMalariaReport}. Beyond direct health consequences, malaria slows economic development and strains healthcare systems, perpetuating cycles of poverty \parencite{Sachs2002MacroeconomicsHealth}. Despite decades of eradication efforts including insecticide-treated nets and preventive therapies the disease continues to pose a major obstacle to global health.\\

\noindent
Effective control relies on early, reliable, and widely available diagnosis. Missed or delayed diagnoses lead to incorrect treatment, increased mortality, and continued transmission \parencite{WHO2024GlobalMalariaReport}. However, diagnostic consistency remains a significant hurdle. Microscopy is highly accurate but demands trained technicians, stable electricity, and equipped laboratories resources often scarce in rural areas \parencite{Wongsrichanalai2007}. Rapid diagnostic tests (RDTs) offer an accessible alternative, yet their performance is inconsistent. Field studies in Cameroon highlight this variability, some tests show high specificity but low sensitivity \parencite{Ngalame2025}, others demonstrate the reverse, leading to high false-positive rates \parencite{Missoup2025}. Furthermore, factors such as storage conditions and parasite genetic variations, including \textit{pfhrp2/3} deletions, jeopardize RDT reliability \parencite{Gatton2020,Prosser2021}.\\

\noindent
These limitations urges the need for diagnostic tools that balance accuracy, speed, and field suitability. Digital microscopy and AI-based image analysis show promise in reducing reliance on specialized infrastructure while making precise diagnosis accessible \parencite{Poostchi2018,Maturana2022,WHO2024GlobalMalariaReport}.

\section*{The Malaria Diagnostic Challenge}
\label{sec:malaria_challenge}
Traditional diagnosis relies on microscopy and \acp{RDT}, both of which face operational constraints. Microscopy (Figure~\ref{fig:intro_malaria_diag_methods}) is the gold standard \parencite{WHO2015MalariaMicroscopyQA}. When performed by experts, it identifies parasite species and infection severity. However, its efficacy declines in real-world settings due to technician shortages, variable slide preparation, and the time-intensive nature of manual examination.\\

\begin{figure}[htbp]
	\centering
	\begin{subfigure}[b]{0.48\textwidth}
		\centering
		\includegraphics[width=\linewidth, height=4cm, keepaspectratio]{Images/MalariaMicroscope.jpeg}
		\caption{Manual microscopy of a blood smear.}
		\label{fig:microscopy_malaria}
	\end{subfigure}
	\hfill 
	\begin{subfigure}[b]{0.48\textwidth}
		\centering
		\includegraphics[width=\linewidth, height=4cm, keepaspectratio]{Images/RDT.jpeg}
		\caption{Rapid Diagnostic Test (RDT).}
		\label{fig:rdt_malaria}
	\end{subfigure}
	\caption{Common methods for malaria diagnosis.}
	\label{fig:intro_malaria_diag_methods}
\end{figure}

\noindent
Conversely, \acp{RDT} are cost-effective and easy to deploy in underserved regions. Yet, they lack the ability to quantify parasitemia or identify specific species and may fail to detect low-density infections or specific strains \parencite{Eyong2022Plasmodium, Jacobs2014}. The resulting \acp{FN} and \acp{FP} contribute to untreated cases or drug resistance through misuse. Combined with supply chain interruptions and high seasonal caseloads, these limitations leave large populations vulnerable to inaccurate diagnosis.

\section*{The Potential of AI in Medical Imaging}
\label{sec:potential_AI}

Advances in \ac{AI}, particularly \ac{CV} and \ac{DL}, offer new pathways for medical imaging. Models based on \acp{CNN} and transformer architectures can analyze data with consistency and precision that rivals human experts. Systems based on these architectures have reached near expert performance in tasks such as detecting tumors in X-rays, grading retinal diseases in fundus photographs \parencite{Gulshan2016DevelopmentValidationDLAlgorithm}, and classifying skin lesions from dermoscopic images \parencite{Esteva2017DermatologistlevelClassificationSkin}. In the context of malaria, \ac{AI} offers distinct advantages over traditional methods.\\

\noindent
First, \ac{AI} models can automatically detect parasites in blood smear images, reducing dependency on human expertise and accelerating turnaround times. Second, unlike standard \acp{RDT}, \acp{AI} can classify species and estimate parasite density metrics essential for appropriate treatment. Third, these systems can be embedded into low-cost digital microscopes or mobile applications, extending reach to low-resource environments. While currently limited to proof-of-concept studies \parencite{Rajaraman2018NIHMalariaDataset}, \ac{AI} offers scalability and adaptability, models can be retrained to recognize new strains or regional variations. By combining automated analysis with telemedicine, \ac{AI} can serve as a powerful complementary tool to strengthen malaria control programs.

\section*{Research Problem and Motivation}
\label{sec:problem_motivation}
While \ac{DL} and \ac{TL} have shown promise in detecting \textit{parasitized cells}, current systems often fall short of clinical utility. Most approaches focus on binary classification (infected vs. uninfected) or parasitized cell detection, whereas a reliable diagnosis requires locating parasites, distinguishing cell types (e.g., \acp{RBC}, \acp{WBC}), estimating parasitemia and even distinguishing parasite specie. The current literature of \ac{STL} approaches do not deliver this comprehensive information from a single smear using a single model.\\

\noindent
\acl{MTL}, a paradigm designed to learn related tasks simultaneously, has been largely overlooked in this domain. We go even further and propose that combining \ac{MTL} with the prior knowledge of \ac{TL} creates a superior hybrid strategy. To our knowledge, this combined paradigm \textbf{\ac{MTTL}} has not been formally applied to malaria diagnosis. This thesis addresses two central problems:

\begin{enumerate}
	\item \textbf{Practical Problem:} Existing works in the literature address only isolated aspects of diagnosis (e.g., infected cell detection, binary classification), providing clinicians with partial outputs rather than a diagnostic system.
	\item \textbf{Theoretical Problem:} There is no clear formulation in current literature unifying \ac{TL}, \ac{MTL}, and modern \ac{PEFT} techniques for medical image analysis. Our work closes this gap by proposing a unified formulation and using malaria detection as a representative case.
\end{enumerate}

\section*{Objectives and Research Questions}
\label{sec:objectives_rq}
The aim of this research is to design and empirically validate a \ac{MTTL} framework for automated malaria diagnosis.

\begin{itemize}
	\item \textbf{Main Objective:} Design a \ac{MTTL} approach and validate its effectiveness in malaria diagnosis.  
	\item \textbf{Specific Objectives:}
	\begin{itemize}
		\item Formalize and propose a mathematical formulation of \ac{MTTL}.  
		\item Apply the proposed framework to malaria diagnosis tasks.  
		\item Compare the performance of \ac{MTTL} against \ac{STL} baselines.  
		\item Evaluate the contribution of auxiliary tasks to identify optimal configurations.  
	\end{itemize}
\end{itemize}
\noindent \textbf{Research Question:} \textit{Does an \ac{MTTL} approach provide superior efficacy and robustness compared to \ac{STL} for automated malaria diagnosis?} \\

\noindent This research question guided our work and led us to the conclusions drawn at the end of this work.

\section*{Hypotheses}
\label{sec:hypotheses}
\noindent\textbf{Main Hypothesis}:
\begin{quote}
	We hypothesize that initializing a shared backbone with relevant prior knowledge and jointly learning related tasks forces the model to learn a more powerful, generalizable shared representation, improving performance on individual tasks compared to independent training.
\end{quote}

\noindent\textbf{Supporting Hypotheses}
\begin{itemize}
	\item \textbf{H1 (\ac{MTTL} vs. \ac{STL})}: The \ac{MTTL} paradigm will outperform \ac{STL} baselines in diagnostic completeness and generalization, specifically in multi-class cell detection.
	\item \textbf{H2 (Auxiliary Task Synergy)}: Auxiliary tasks will act as effective regularizers.
	\item \textbf{H3 (Robustness)}: \ac{MTTL} models will demonstrate higher robustness on specialized subsets (Infected-Only and Healthy-Only) compared to \ac{STL} baselines.
\end{itemize}

\section*{Contributions and Thesis Outline}
\label{sec:contributions_outline}

This work contributes to medical imaging and \ac{ML} by providing:
\begin{itemize}
	\item \textbf{An \ac{MTTL} Framework:} A mathematical formulation combining \ac{TL}, \ac{MTL}, and \ac{PEFT} for medical image analysis.
	\item \textbf{A Diagnostic System:} The design and validation of a comprehensive, multi-task system for malaria diagnosis.
	\item \textbf{Empirical Evidence:} A rigorous study demonstrating the superiority of the \ac{MTTL} paradigm over \ac{STL} for complex diagnostics.
\end{itemize}

\noindent
The rest of this document is structured as follows. \textbf{Chapter \ref{chap:theory}} establishes the theoretical foundations of \ac{DL}, \acp{CNN}, \ac{TL}, \ac{MTL}, \ac{PEFT}, and the current state of literature for AI applied to medical imaging and malaria specifically. \textbf{Chapter \ref{chap:framework}} presents our core contribution, the mathematical framework for \ac{MTTL}. \textbf{Chapter \ref{chap:methodology}} details the methodological application to malaria diagnosis, the experimental implementation, including dataset and architecture. \textbf{Chapter \ref{chap:results}} reports the empirical findings and comparative analysis. Finally, \textbf{Chapter \ref{chap:discussion}} interprets the implications and limitations of the study, and \textbf{Chapter \ref{chap:conclusion}} summarizes findings and outlines future directions.