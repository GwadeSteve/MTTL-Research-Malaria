\begin{figure}[htbp]
	\centering
	\begin{tikzpicture}[
		scale=0.9, every node/.style={scale=0.9, align=center, font=\footnotesize},
		% Define styles for different layer types
		block_style/.style={
			rectangle, draw, rounded corners, minimum height=1cm, text width=2cm, thick, inner sep=4pt},
		arrow_style/.style={-Stealth, thick, shorten >=2pt, shorten <=2pt}
		]
		
		% Nodes arranged horizontally
		\node[block_style, fill=blue!10] (input) {Input Image};
		\node[block_style, fill=green!10, right=1cm of input] (layer1) {Simple Features};
		\node[block_style, fill=green!20, right=1cm of layer1] (layer2) {Combined Features};
		\node[block_style, fill=green!30, right=1cm of layer2] (layer3) {Abstract Features};
		\node[block_style, fill=red!10, right=1cm of layer3] (output) {Prediction};
		
		% Arrows connecting the layers
		\draw[arrow_style] (input.east) -- (layer1.west);
		\draw[arrow_style] (layer1.east) -- (layer2.west);
		\draw[arrow_style] (layer2.east) -- (layer3.west);
		\draw[arrow_style] (layer3.east) -- (output.west);
		
		% Label for direction of abstraction
		\node[above=0.6cm of layer2, font=\small\bfseries, text width=7cm] {Increasing Feature Abstraction $\longrightarrow$};
		
	\end{tikzpicture}
	\caption{Hierarchical feature learning in a deep network.}
	\label{fig:hierarchical_features_simple}
\end{figure}