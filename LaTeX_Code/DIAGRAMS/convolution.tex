\begin{figure}[htbp]
\centering
\begin{tikzpicture}[
    scale=0.8, every node/.style={scale=0.8, font=\footnotesize, align=center},
    cell/.style={rectangle, draw, minimum size=0.7cm, inner sep=2pt},
    filter_cell/.style={rectangle, draw=blue, fill=blue!10, minimum size=0.7cm, inner sep=2pt},
    output_cell/.style={rectangle, draw=red, fill=red!10, minimum size=0.7cm, inner sep=2pt},
    op_symbol/.style={circle, draw, fill=yellow!30, minimum size=0.5cm, inner sep=1pt, font=\Large},
    arrow_style/.style={-Stealth, thick, shorten >=1pt, shorten <=1pt}
  ]

    \matrix (input_patch) [matrix of nodes, nodes={cell}, column sep=-\pgflinewidth, row sep=-\pgflinewidth, anchor=center] at (0,0) {
        \pgfmathrandominteger{\val}{0}{9}\val & \pgfmathrandominteger{\val}{0}{9}\val & \pgfmathrandominteger{\val}{0}{9}\val \\
        \pgfmathrandominteger{\val}{0}{9}\val & \pgfmathrandominteger{\val}{0}{9}\val & \pgfmathrandominteger{\val}{0}{9}\val \\
        \pgfmathrandominteger{\val}{0}{9}\val & \pgfmathrandominteger{\val}{0}{9}\val & \pgfmathrandominteger{\val}{0}{9}\val \\
    };
    \node[above=0.2cm of input_patch] {Input Patch};

    \matrix (filter) [matrix of nodes, nodes={filter_cell}, column sep=-\pgflinewidth, row sep=-\pgflinewidth, anchor=center] at (3,1) { % Positioned slightly above center
        \pgfmathrandominteger{\w}{0}{1}\w & \pgfmathrandominteger{\w}{0}{1}\w \\
        \pgfmathrandominteger{\w}{0}{1}\w & \pgfmathrandominteger{\w}{0}{1}\w \\
    };
    \node[above=0.2cm of filter] {Filter (Kernel)};

    \begin{scope}[on background layer]
        \node [fit=(input_patch-1-1) (input_patch-2-2), draw=blue, thick, fill=blue!5, inner sep=-1pt, dashed] {};
    \end{scope}

    \node[op_symbol] (operation) at (2.5, -1.5) {$\otimes$}; % Using \otimes for convolution-like operation
    \node[below=0.1cm of operation, text width=2.5cm, font=\tiny] {Element-wise Product + Sum + Bias};

    \node[output_cell] (output_value) at (5, -1.5) {Result};
    \node[above=0.2cm of output_value] {Output Element};

    \draw[arrow_style] (input_patch-2-2.south) .. controls +(0,-0.5) and +(-0.5,0.5) .. (operation.west);
    \draw[arrow_style] (filter.south) .. controls +(0,-0.5) and +(0.5,0.5) .. (operation.east);
    \draw[arrow_style] (operation.east) -- (output_value.west);

    \draw[->, blue!70!black, dashed, shorten >=2pt, shorten <=2pt] (filter.east) ++(0.2,0) -- ++(0.7,0) node[right, font=\tiny, black] {Filter slides};

\end{tikzpicture}
\caption{Principle of the convolution operation.}
\label{fig:convolution_simplified}
\end{figure}