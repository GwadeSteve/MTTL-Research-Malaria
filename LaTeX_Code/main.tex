%\documentclass[english,12pt,a4paper,twoside,openright]{report}
\documentclass[english,12pt,a4paper]{report}

\usepackage[a4paper,tmargin=2.0cm,bmargin=2.0cm,rmargin=2cm,lmargin=2cm]{geometry}
\usepackage[utf8]{inputenc} 

\usepackage{float}
\usepackage{subcaption}

\usepackage{amssymb, amsfonts}
\usepackage{graphicx}
\usepackage{caption}
\usepackage{subcaption}
\usepackage{adjustbox}
\usepackage{algpseudocode}
\usepackage{csquotes}
\usepackage{xcolor}
\definecolor{thesisblue}{RGB}{20, 68, 106}
\definecolor{thesisgray}{RGB}{80, 80, 80}
\usepackage{lmodern}
\usepackage{array}
\usepackage{booktabs} 
\usepackage{tabularx}
\newcolumntype{Y}{>{\centering\arraybackslash}X}

\usepackage{enumitem} 
\usepackage{colortbl}
\usepackage{lscape}   

\usepackage{pgfplots}  
\usepgfplotslibrary{groupplots}
\pgfplotsset{compat=1.18} 

\definecolor{highlightcolor}{RGB}{218, 232, 252} 
\usepackage{collcell}
\newcommand{\maxval}[1]{\pgfmathparse{#1}\ifdim\pgfmathresult pt=\thisrowmax pt \bfseries\cellcolor{highlightcolor}{#1}\else{#1}\fi}
\newcolumntype{H}{>{\collectcell\maxval}c<{\endcollectcell}} % H for Highlight

\usepackage{float}

\usepackage{etoolbox}
\usepackage{tabularray}

\usepackage{tikz}
\usetikzlibrary{
	positioning,
	fit,
	arrows.meta,
	shapes.geometric,
	calc,
	matrix,
	patterns,
	shadows,
	backgrounds,
	decorations.pathreplacing,
	shapes.misc
}

\tikzset{
	% For 3D-like convolutional layers
	cuboid/.style={
		shape=trapezium,
		trapezium left angle=70,
		trapezium right angle=-70,
		shape border rotate=90,
		minimum width=2.2cm,
		minimum height=1cm,
		draw, thick,
		align=center,
		font=\small
	},
	% Default cuboid color (Frozen)
	cuboidfill/.style={
		fill=blue!20
	},
	% Style for trainable components (sets cuboid color to orange)
	trainable/.style={
		fill=orange!30
	},
	% Style for generic blocks
	block/.style={
		rectangle, draw, thick,
		minimum width=3cm, minimum height=1.5cm, align=center,
		rounded corners=3pt
	},
	% Standard block for operations
	op/.style={
		circle, draw, thick, fill=gray!10, minimum size=0.8cm
	},
	% Main arrows for data flow
	arrow/.style={
		-Stealth, thick, draw=black!80
	},
	% Orange LoRA arrows
	loraarrow/.style={
		-Stealth, thick, draw=orange!80!black
	},
	% Dashed highlight around LoRA path
	lorabox/.style={
		draw=orange!70!black, dashed, thick,
		inner sep=6pt, rounded corners
	},
	% Title for a diagram section
	blocktitle/.style={
		font=\sffamily\bfseries\small, text=black!90
	}, 
	outnode/.style={
		rectangle, draw, fill=blue!10, rounded corners, minimum height=1cm
	}
}

\renewcommand{\familydefault}{\rmdefault} 
\usepackage{mathpazo}                   
\usepackage{microtype}          
\usepackage{setspace}               
\setstretch{1.2}                          

\usepackage[Lenny]{fncychap}
\usepackage{titlesec}

\titleformat*{\section}{\Large\bfseries\sffamily}       
\titleformat*{\subsection}{\large\bfseries\sffamily}    
\titleformat*{\subsubsection}{\normalsize\bfseries\sffamily} 

\titleformat{\chapter}[display]{\sf\bfseries\LARGE}{\vspace{-10ex}
	\filleft\MakeUppercase{\chaptertitlename}~\Huge\thechapter}{4ex}
	{\titlerule\vspace{2ex}\filright}[\vspace{2ex}\titlerule]

\usepackage{fancyhdr}
\pagestyle{fancy}
\fancyfoot[LE,RO]{\tiny Master of Science in Data Science and Artificial Intelligence}
\rhead{}
\lfoot{\footnotesize\tiny\textbf{Gwade Steve}}
\rfoot{\footnotesize\tiny\textbf{MTTL and Application to Malaria Detection}}

\usepackage[english]{babel}
\usepackage{verbatim}
\usepackage{url}
\usepackage{lipsum}
\usepackage{csquotes}

\usepackage{calc,amsfonts,amssymb,amsthm,latexsym}
\usepackage{tocloft}
\usepackage[centertags]{amsmath}

\newcommand{\listequationsname}{List of Equations}
\newlistof{myequations}{equ}{\listequationsname}

\setlength{\cftmyequationsnumwidth}{1cm} 

\newcommand{\myequations}[1]{%
	\addcontentsline{equ}{myequations}{\protect\numberline{\theequation}#1}}

\usepackage{fancybox,wrapfig,exscale,colortbl,tcolorbox,frcursive,empheq,multicol,varwidth,lipsum}
\usepackage{pgf,tikz}
\usetikzlibrary{arrows,calc}
\usepackage{eso-pic,bclogo}
\usepackage[all]{xy}

\usepackage{hyperref}
\hypersetup{
	colorlinks=true,
	linkcolor=black,
	citecolor=teal,
	urlcolor=magenta,
	filecolor=orange,
	pdfauthor={Gwade Steve},
	pdftitle={Multi-Task Transfer Learning and Application to Malaria Detection}
}

% BIBLIOGRAPHY / REFERENCES
%\usepackage[backend=biber, style=numeric, sorting=none]{biblatex} 
\usepackage[backend=biber, style=authoryear, maxcitenames=1, maxbibnames=5, sorting=nyt]{biblatex}
\addbibresource{references.bib}

\DeclareFieldFormat{year}{\textcolor{teal}{#1}}

\DeclareNameFormat{author}{%
	\textbf{\namepartfamily}%
	\ifthenelse{\value{listcount}<\value{liststop}}{\addcomma\space}{}%
}

\setlength\bibitemsep{0.1em} 
\setlength{\bibhang}{1em}
\DeclareFieldFormat[article]{journaltitle}{\mkbibemph{#1}}

\usepackage{acro}
\usepackage{longtable} 
\usepackage{multicol}

\usepackage[ruled,vlined]{algorithm2e}
\SetKwInput{KwInput}{Input}
\SetKwInput{KwOutput}{Output}

% ACRONYMS LIST
\DeclareAcronym{AI}{short=AI,long=Artificial Intelligence}
\DeclareAcronym{AMP}{short=AMP,long=Automatic Mixed Precision}
\DeclareAcronym{ANN}{short=ANN,long=Artificial Neural Network}
\DeclareAcronym{AUC}{short=AUC,long=Area Under the Curve}
\DeclareAcronym{AUROC}{short=AUROC,long=Area Under the Receiver Operating Characteristic Curve}
\DeclareAcronym{API}{short=API,long=Application Programming Interface}
\DeclareAcronym{BCE}{short=BCE,long=Binary Cross-Entropy}
\DeclareAcronym{BiLSTM}{short=BiLSTM,long=Bidirectional Long Short-Term Memory}
\DeclareAcronym{BERT}{short=BERT,long=Bidirectional Encoder Representations from Transformers}
\DeclareAcronym{BN}{short=BN,long=Batch Normalization}
\DeclareAcronym{CNN}{short=CNN,long=Convolutional Neural Network}
\DeclareAcronym{CV}{short=CV,long=Computer Vision}
\DeclareAcronym{CT}{short=CT,long=Computed Tomography}
\DeclareAcronym{DA}{short=DA,long=Domain Adaptation}
\DeclareAcronym{DBN}{short=DBN,long=Deep Belief Network}
\DeclareAcronym{DNN}{short=DNN,long=Deep Neural Network}
\DeclareAcronym{DL}{short=DL,long=Deep Learning}
\DeclareAcronym{DR}{short=DR,long=Diabetic Retinopathy}
\DeclareAcronym{ECG}{short=ECG,long=Electrocardiogram}
\DeclareAcronym{EEG}{short=EEG,long=Electroencephalogram}
\DeclareAcronym{ELMo}{short=ELMo,long=Embeddings from Language Models}
\DeclareAcronym{FC}{short=FC, long=Fully Connected}
\DeclareAcronym{FP}{short=FP,long=False Positive}
\DeclareAcronym{FN}{short=FN,long=False Negative}
\DeclareAcronym{F1}{short=F1,long=F1-Score}
\DeclareAcronym{FPN}{short=FPN,long=Feature Pyramid Network}
\DeclareAcronym{FPR}{short=FPR,long=False Positive Rate}
\DeclareAcronym{GAN}{short=GAN,long=Generative Adversarial Network}
\DeclareAcronym{GRU}{short=GRU,long=Gated Recurrent Unit}
\DeclareAcronym{GPU}{short=GPU,long=Graphics Processing Unit}
\DeclareAcronym{IoU}{short=IoU,long=Intersection over Union}
\DeclareAcronym{KNN}{short=KNN,long=k-Nearest Neighbors}
\DeclareAcronym{LBP}{short=LBP,long=Local Binary Pattern}
\DeclareAcronym{LDA}{short=LDA,long=Linear Discriminant Analysis}
\DeclareAcronym{LIME}{short=LIME,long=Local Interpretable Model-agnostic Explanations}
\DeclareAcronym{LSTM}{short=LSTM,long=Long Short-Term Memory}
\DeclareAcronym{LoRA}{short=LoRA,long=Low-Rank Adaptation}
\DeclareAcronym{MAE}{short=MAE,long=Mean Absolute Error}
\DeclareAcronym{mAP}{short=mAP,long=mean Average Precision}
\DeclareAcronym{MSE}{short=MSE,long=Mean Squared Error}
\DeclareAcronym{ML}{short=ML,long=Machine Learning}
\DeclareAcronym{MTL}{short=MTL,long=Multi-Task Learning}
\DeclareAcronym{MTTL}{short=MTTL,long=Multi-Task Transfer Learning}
\DeclareAcronym{MRI}{short=MRI,long=Magnetic Resonance Imaging}
\DeclareAcronym{NLP}{short=NLP,long=Natural Language Processing}
\DeclareAcronym{NLM}{short=NLM,long=National Library of Medecine}
\DeclareAcronym{NIH}{short=NIH,long=National Institutes of Health}
\DeclareAcronym{PCA}{short=PCA,long=Principal Component Analysis}
\DeclareAcronym{PDF}{short=PDF,long=Probability Density Function}
\DeclareAcronym{PEFT}{short=PEFT,long=Parameter-Efficient Fine-Tuning}
\DeclareAcronym{PSNR}{short=PSNR,long=Peak Signal-to-Noise Ratio}
\DeclareAcronym{RBC}{short=RBC,long=Red Blood Cell}
\DeclareAcronym{RDT}{short=RDT,long=Rapid Diagnostic Test}
\DeclareAcronym{ReLU}{short=ReLU,long=Rectified Linear Unit}
\DeclareAcronym{ResNet}{short=ResNet,long=Residual Network}
\DeclareAcronym{RoI}{short=RoI,long=Region of Interest}
\DeclareAcronym{RPN}{short=RPN,long=Region Proposal Network}
\DeclareAcronym{ROC}{short=ROC,long=Receiver Operating Characteristic}
\DeclareAcronym{SGD}{short=SGD,long=Stochastic Gradient Descent}
\DeclareAcronym{STL}{short=STL,long=Single-Task Learning}
\DeclareAcronym{SVM}{short=SVM,long=Support Vector Machine}
\DeclareAcronym{TL}{short=TL,long=Transfer Learning}
\DeclareAcronym{TP}{short=TP,long=True Positive}
\DeclareAcronym{TN}{short=TN,long=True Negative}
\DeclareAcronym{TPR}{short=TPR,long=True Positive Rate}
\DeclareAcronym{TPU}{short=TPU,long=Tensor Processing Unit}
\DeclareAcronym{VGG}{short=VGG,long=Visual Geometry Group Network}
\DeclareAcronym{ViT}{short=ViT,long=Vision Transformer}
\DeclareAcronym{WBC}{short=WBC,long=White Blood Cell}
\DeclareAcronym{WHO}{short=WHO,long=World Health Organization}
\DeclareAcronym{XAI}{short=XAI,long=Explainable Artificial Intelligence}
\DeclareAcronym{YOLO}{short=YOLO,long=You Only Look Once}
\DeclareAcronym{DenseNet}{short=DenseNet,long=Densely Connected Convolutional Network}
\DeclareAcronym{Incep}{short=Incep,long=Inception Network}
\DeclareAcronym{UNet}{short=UNet,long=U-Net Convolutional Network}
\DeclareAcronym{Optuna}{short=Optuna,long=Hyperparameter Optimization Framework}
\DeclareAcronym{FasterRCNN}{short=FasterRCNN,long=Faster Region-Based Convolutional Neural Network}
\DeclareAcronym{MMDetection}{short=MMDetection,long=OpenMMLab Detection Toolbox}
\DeclareAcronym{COCO}{short=COCO,long=Common Objects in Context Dataset}
\DeclareAcronym{ImageNet}{short=ImageNet,long=ImageNet Large Scale Visual Recognition Dataset}
\DeclareAcronym{GANomaly}{short=GANomaly,long=Generative Adversarial Anomaly Detection}
\DeclareAcronym{R-CNN}{short=R-CNN,long=Region-Based Convolutional Neural Network}
\DeclareAcronym{SSD}{short=SSD,long=Single Shot Detector}
\DeclareAcronym{ViViT}{short=ViViT,long=Video Vision Transformer}
\DeclareAcronym{MixUp}{short=MixUp,long=Data Augmentation Technique}
\DeclareAcronym{CutMix}{short=CutMix,long=Data Augmentation Technique}
\DeclareAcronym{SWIN}{short=SWIN,long=Shifted Window Transformer}
\DeclareAcronym{SAM}{short=SAM,long=Segment Anything Model}
\DeclareAcronym{CLIP}{short=CLIP,long=Contrastive Language-Image Pretraining}
\DeclareAcronym{MAE-MLM}{short=MAE-MLM,long=Masked Autoencoder with Masked Language Modeling}
\DeclareAcronym{SwinUNet}{short=SwinUNet,long=Swin-UNet Network for Segmentation}
\DeclareAcronym{GradCAM}{short=GradCAM,long=Gradient-weighted Class Activation Mapping}
\DeclareAcronym{Dice}{short=Dice,long=Dice Similarity Coefficient}
\DeclareAcronym{IoBB}{short=IoBB,long=Intersection over Bounding Box}
\DeclareAcronym{Ablation}{short=Ablation,long=Ablation Study}
\DeclareAcronym{mDice}{short=mDice,long=Mean Dice Score}
\DeclareAcronym{FocalLoss}{short=FocalLoss,long=Focal Loss Function}
\DeclareAcronym{DiceLoss}{short=DiceLoss,long=Dice Loss Function}
\DeclareAcronym{AdamW}{short=AdamW,long=Adam Weight Decay Optimizer}
\DeclareAcronym{SGDR}{short=SGDR,long=Stochastic Gradient Descent with Restarts}
\DeclareAcronym{TTA}{short=TTA,long=Test Time Augmentation}
\DeclareAcronym{EMA}{short=EMA,long=Exponential Moving Average}
\DeclareAcronym{LR}{short=LR,long=Learning Rate}
\DeclareAcronym{CSAM}{short=CSAM,long=Channel Spatial Attention Module}
\DeclareAcronym{PSP}{short=PSP,long=Pyramid Scene Parsing Network}
\DeclareAcronym{DeepLabV3}{short=DeepLabV3,long=DeepLabV3 Semantic Segmentation Network}
\DeclareAcronym{MS-SSIM}{short=MS-SSIM,long=Multi-Scale Structural Similarity Index}
\DeclareAcronym{HPA}{short=HPA,long=Human Protein Atlas}
\DeclareAcronym{TCGA}{short=TCGA,long=The Cancer Genome Atlas}
\DeclareAcronym{FISH}{short=FISH,long=Fluorescence In Situ Hybridization}
\DeclareAcronym{PBS}{short=PBS,long=Phosphate Buffered Saline}
\DeclareAcronym{HE}{short=HE,long=Hematoxylin and Eosin Staining}
\DeclareAcronym{IFA}{short=IFA,long=Immunofluorescence Assay}
\DeclareAcronym{qPCR}{short=qPCR,long=Quantitative Polymerase Chain Reaction}
\DeclareAcronym{NGS}{short=NGS,long=Next-Generation Sequencing}
\DeclareAcronym{CIFAR}{short=CIFAR,long=CIFAR Image Classification Dataset}
\DeclareAcronym{MNIST}{short=MNIST,long=Modified National Institute of Standards and Technology Dataset}
\DeclareAcronym{BloodSmear}{short=BloodSmear,long=Peripheral Blood Smear}
\DeclareAcronym{PBSM}{short=PBSM,long=Peripheral Blood Smear Microscopy}
\DeclareAcronym{ParasiteDensity}{short=PD,long=Parasite Density Measurement}
\DeclareAcronym{LifeCycleStage}{short=LCS,long=Parasite Life-Cycle Stage}
\DeclareAcronym{DataAug}{short=DataAug,long=Data Augmentation}
\DeclareAcronym{RandAug}{short=RandAug,long=Random Augmentation}
\DeclareAcronym{SegmMask}{short=SegmMask,long=Segmentation Mask}
\DeclareAcronym{TrainSet}{short=TrainSet,long=Training Dataset}
\DeclareAcronym{ValSet}{short=ValSet,long=Validation Dataset}
\DeclareAcronym{TestSet}{short=TestSet,long=Testing Dataset}
\DeclareAcronym{MLOps}{short=MLOps,long=Machine Learning Operations}
\DeclareAcronym{Docker}{short=Docker,long=Docker Container Platform}
\DeclareAcronym{K8s}{short=K8s,long=Kubernetes}
\DeclareAcronym{CI}{short=CI,long=Continuous Integration}
\DeclareAcronym{CD}{short=CD,long=Continuous Deployment}
\DeclareAcronym{MLFlow}{short=MLFlow,long=Machine Learning Lifecycle Management Platform}
\DeclareAcronym{OpticalFlow}{short=OF,long=Optical Flow}
\DeclareAcronym{SIFT}{short=SIFT,long=Scale-Invariant Feature Transform}
\DeclareAcronym{SURF}{short=SURF,long=Speeded Up Robust Features}
\DeclareAcronym{HOG}{short=HOG,long=Histogram of Oriented Gradients}
\DeclareAcronym{ORB}{short=ORB,long=Oriented FAST and Rotated BRIEF}
\DeclareAcronym{YOLOv5}{short=YOLOv5,long=You Only Look Once Version 5}
\DeclareAcronym{YOLOv8}{short=YOLOv8,long=You Only Look Once Version 8}
\DeclareAcronym{FPNLite}{short=FPNLite,long=Lightweight Feature Pyramid Network}
\DeclareAcronym{CIF}{short=CIF,long=Canadian Institute for Forestry} % Example of extra
\DeclareAcronym{TPM}{short=TPM,long=Transcripts Per Million}
\DeclareAcronym{GNN}{short=GNN,long=Graph Neural Network}
\DeclareAcronym{MPNN}{short=MPNN,long=Message Passing Neural Network}
\DeclareAcronym{EdgeAI}{short=EdgeAI,long=Edge Artificial Intelligence}
\DeclareAcronym{TinyML}{short=TinyML,long=Tiny Machine Learning}
\DeclareAcronym{FPGA}{short=FPGA,long=Field-Programmable Gate Array}
\DeclareAcronym{ASIC}{short=ASIC,long=Application-Specific Integrated Circuit}
\DeclareAcronym{RL}{short=RL,long=Reinforcement Learning}
\DeclareAcronym{PPO}{short=PPO,long=Proximal Policy Optimization}
\DeclareAcronym{SAC}{short=SAC,long=Soft Actor-Critic}
\DeclareAcronym{DQN}{short=DQN,long=Deep Q-Network}
\DeclareAcronym{HER}{short=HER,long=Hindsight Experience Replay}
\DeclareAcronym{MCTS}{short=MCTS,long=Monte Carlo Tree Search}
\DeclareAcronym{OCT}{short=OCT,long=Optical Coherence Tomography}
\DeclareAcronym{fMRI}{short=fMRI,long=functional Magnetic Resonance Imaging}
\DeclareAcronym{DTI}{short=DTI,long=Diffusion Tensor Imaging}
\DeclareAcronym{EEGNet}{short=EEGNet,long=Deep Learning Network for EEG Signals}
\DeclareAcronym{EMG}{short=EMG,long=Electromyography}
\DeclareAcronym{PET}{short=PET,long=Positron Emission Tomography}
\DeclareAcronym{SPECT}{short=SPECT,long=Single Photon Emission Computed Tomography}
\DeclareAcronym{CAD}{short=CAD,long=Computer-Aided Diagnosis}
\DeclareAcronym{RIS}{short=RIS,long=Radiology Information System}
\DeclareAcronym{PACS}{short=PACS,long=Picture Archiving and Communication System}
\DeclareAcronym{DICOM}{short=DICOM,long=Digital Imaging and Communications in Medicine}
\DeclareAcronym{HL7}{short=HL7,long=Health Level 7 Standard}
\DeclareAcronym{FHIR}{short=FHIR,long=Fast Healthcare Interoperability Resources}
\DeclareAcronym{LIFEx}{short=LIFEx,long=Radiomics Feature Extraction Software}
\DeclareAcronym{PyRadiomics}{short=PyRad,long=Python Radiomics Library}
\DeclareAcronym{VAE}{short=VAE,long=Variational Autoencoder}
\DeclareAcronym{CVAE}{short=CVAE,long=Conditional Variational Autoencoder}
\DeclareAcronym{cGAN}{short=cGAN,long=Conditional Generative Adversarial Network}
\DeclareAcronym{InfoGAN}{short=InfoGAN,long=Information Maximizing GAN}
\DeclareAcronym{SRGAN}{short=SRGAN,long=Super-Resolution GAN}
\DeclareAcronym{SR}{short=SR,long=Super-Resolution}
\DeclareAcronym{ESRGAN}{short=ESRGAN,long=Enhanced SRGAN}
\DeclareAcronym{MUNIT}{short=MUNIT,long=Multimodal Unsupervised Image-to-Image Translation}
\DeclareAcronym{CycleGAN}{short=CycleGAN,long=Cycle-Consistent GAN}
\DeclareAcronym{Pix2Pix}{short=Pix2Pix,long=Paired Image-to-Image Translation GAN}
\DeclareAcronym{UDA}{short=UDA,long=Unsupervised Domain Adaptation}
\DeclareAcronym{FDA}{short=FDA,long=Fourier Domain Adaptation}
\DeclareAcronym{WD}{short=WD,long=Weight Decay}
\DeclareAcronym{SWAG}{short=SWAG,long=Stochastic Weight Averaging Gaussian}
\DeclareAcronym{LaplaceApprox}{short=LapApprox,long=Laplace Approximation}
\DeclareAcronym{MC-Dropout}{short=MCD,long=Monte Carlo Dropout}
\DeclareAcronym{KD}{short=KD,long=Knowledge Distillation}
\DeclareAcronym{SOTA}{short=SOTA,long=State-of-the-Art}
\DeclareAcronym{PoissonLoss}{short=PoissonLoss,long=Poisson Negative Log-Likelihood Loss}
\DeclareAcronym{Bbox}{short=Bbox,long=Bounding Box}
\DeclareAcronym{Heatmap}{short=Heatmap,long=Localization Heatmap}
\DeclareAcronym{WSI}{short=WSI,long=Whole Slide Imaging}
\DeclareAcronym{CLAHE}{short=CLAHE,long=Contrast Limited Adaptive Histogram Equalization}


\acsetup{
	make-links = true,
	first-style = long-short,
	list/name = {List of Acronyms},
	list/heading = none,
	list/sort = true,
	list/template = description,
	pages/display = first
}

\newtheorem{exo}{{Exercice}}[chapter]
\newtheorem{remark}{{Remarque}}[chapter]
\newtheorem{definition}{Définition}[chapter]
\newtheorem{example}{{Exemple}}[chapter]
%% -------------------------------
% PAGE GEOMETRY
% -------------------------------
\usepackage[a4paper,tmargin=2.0cm,bmargin=2.0cm,rmargin=2cm,lmargin=2cm]{geometry}

\usepackage{float}
\usepackage{subcaption}

% --- STANDARD PACKAGES ---
\usepackage{amssymb, amsfonts}
\usepackage{graphicx}
\usepackage{caption}
\usepackage{subcaption}

\usepackage{algpseudocode}
\usepackage{csquotes}
\usepackage{xcolor}
\definecolor{thesisblue}{RGB}{20, 68, 106}
\definecolor{thesisgray}{RGB}{80, 80, 80}
\usepackage{lmodern}
\usepackage{array}
\usepackage{booktabs}  % For professional-looking tables
\usepackage{tabularx}

\usepackage{enumitem}  % For customized lists (e.g., itemize, enumerate)
\usepackage{colortbl}
\usepackage{lscape}    % Optional: For landscape tables if the table is extremely wide

\usepackage{pgfplots}  % For plotting functions
\usepgfplotslibrary{groupplots}
\pgfplotsset{compat=1.18} 
\usepackage{float}

\usepackage{etoolbox}
\usepackage{tabularray}

% --- TIKZ PACKAGES & LIBRARIES ---
\usepackage{tikz}
\usetikzlibrary{
	positioning,
	fit,
	arrows.meta,
	shapes.geometric,
	calc,
	matrix,
	patterns,
	shadows,
	backgrounds,
	decorations.pathreplacing,
	shapes.misc
}

% --- CUSTOM TIKZ STYLES (plotneuralnet inspired) ---
\tikzset{
	% For 3D-like convolutional layers
	cuboid/.style={
		shape=trapezium,
		trapezium left angle=70,
		trapezium right angle=-70,
		shape border rotate=90,
		minimum width=2.2cm,
		minimum height=1cm,
		draw, thick,
		align=center,
		font=\small
	},
	% Default cuboid color (Frozen)
	cuboidfill/.style={
		fill=blue!20
	},
	% Style for trainable components (sets cuboid color to orange)
	trainable/.style={
		fill=orange!30
	},
	% Style for generic blocks
	block/.style={
		rectangle, draw, thick,
		minimum width=3cm, minimum height=1.5cm, align=center,
		rounded corners=3pt
	},
	% Standard block for operations
	op/.style={
		circle, draw, thick, fill=gray!10, minimum size=0.8cm
	},
	% Main arrows for data flow
	arrow/.style={
		-Stealth, thick, draw=black!80
	},
	% Orange LoRA arrows
	loraarrow/.style={
		-Stealth, thick, draw=orange!80!black
	},
	% Dashed highlight around LoRA path
	lorabox/.style={
		draw=orange!70!black, dashed, thick,
		inner sep=6pt, rounded corners
	},
	% Title for a diagram section
	blocktitle/.style={
		font=\sffamily\bfseries\small, text=black!90
	}, 
	outnode/.style={
		rectangle, draw, fill=blue!10, rounded corners, minimum height=1cm
	}
}

% -------------------------------
% FONTS & TYPOGRAPHY
% -------------------------------
\renewcommand{\familydefault}{\rmdefault} 
\usepackage{mathpazo}                   
\usepackage{microtype}          
\usepackage{setspace}               
\setstretch{1.2}                          

% -------------------------------
% CHAPTER & SECTION STYLING
% -------------------------------
\usepackage[Lenny]{fncychap}
\usepackage{titlesec}
\titleformat*{\section}{\Large\bfseries\sffamily}       
\titleformat*{\subsection}{\large\bfseries\sffamily}    
\titleformat*{\subsubsection}{\large\bfseries\sffamily} 
\titleformat{\chapter}[display]{\sf\bfseries\LARGE}{\vspace{-10ex}
	\filleft\MakeUppercase{\chaptertitlename}~
	\Huge\thechapter}{4ex}{\titlerule\vspace{2ex}\filright}[\vspace{2ex}\titlerule]

% -------------------------------
% HEADERS & FOOTERS
% -------------------------------
\usepackage{fancyhdr}
\pagestyle{fancy}
\fancyfoot[LE,RO]{\tiny Master of Science in Data Science and Artificial Intelligence}
\rhead{}
\lfoot{\footnotesize\tiny\textbf{Gwade Steve}}
\rfoot{\footnotesize\tiny\textbf{MTTL and Application to Malaria Detection}}

% -------------------------------
% INPUT ENCODING & LANGUAGE
% -------------------------------
\usepackage[utf8]{inputenc}
\usepackage[english]{babel}
\usepackage{verbatim}
\usepackage{url}
\usepackage{lipsum}
\usepackage{csquotes}

% -------------------------------
% MATH & SYMBOLS
% -------------------------------
\usepackage{calc,amsfonts,amssymb,amsthm,latexsym}
\usepackage{tocloft}
\usepackage[centertags]{amsmath}

% -------------------------------
% LIST OF EQUATIONS
% -------------------------------
\newcommand{\listequationsname}{List of Equations}
\newlistof{myequations}{equ}{\listequationsname}

% Command to add an equation caption to the list
\newcommand{\myequations}[1]{%
	\addcontentsline{equ}{myequations}{\protect\numberline{\theequation}#1}}


% -------------------------------
% GRAPHICS & OTHER PACKAGES
% -------------------------------
\usepackage{fancybox,wrapfig,exscale,colortbl,tcolorbox,frcursive,empheq,multicol,varwidth,lipsum}
\usepackage{pgf,tikz}
\usetikzlibrary{arrows,calc}
\usepackage{eso-pic,bclogo}
\usepackage[all]{xy}

% -------------------------------
% HYPERLINKS
% -------------------------------
\usepackage{hyperref}
\hypersetup{
	colorlinks=true,
	linkcolor=black,
	citecolor=teal,
	urlcolor=magenta,
	filecolor=orange,
	pdfauthor={Gwade Steve},
	pdftitle={Multi-Task Transfer Learning – Malaria Detection}
}

% -------------------------------
% BIBLIOGRAPHY / REFERENCES
% -------------------------------
\usepackage[backend=biber, style=numeric, sorting=none]{biblatex} 
\addbibresource{references.bib}

% -------------------------------
% ABBREVIATIONS & ALGORITHMS
% -------------------------------
\usepackage{acro}
\usepackage{longtable} 
\usepackage{multicol}

% Algorithms
\usepackage[ruled,vlined]{algorithm2e}
\SetKwInput{KwInput}{Input}
\SetKwInput{KwOutput}{Output}

% All abbreviations
% ACRONYMS LIST
\DeclareAcronym{AI}{short=AI,long=Artificial Intelligence}
\DeclareAcronym{AMP}{short=AMP,long=Automatic Mixed Precision}
\DeclareAcronym{ANN}{short=ANN,long=Artificial Neural Network}
\DeclareAcronym{AUC}{short=AUC,long=Area Under the Curve}
\DeclareAcronym{AUROC}{short=AUROC,long=Area Under the Receiver Operating Characteristic Curve}
\DeclareAcronym{API}{short=API,long=Application Programming Interface}
\DeclareAcronym{BCE}{short=BCE,long=Binary Cross-Entropy}
\DeclareAcronym{BiLSTM}{short=BiLSTM,long=Bidirectional Long Short-Term Memory}
\DeclareAcronym{BERT}{short=BERT,long=Bidirectional Encoder Representations from Transformers}
\DeclareAcronym{BN}{short=BN,long=Batch Normalization}
\DeclareAcronym{CNN}{short=CNN,long=Convolutional Neural Network}
\DeclareAcronym{CV}{short=CV,long=Computer Vision}
\DeclareAcronym{CT}{short=CT,long=Computed Tomography}
\DeclareAcronym{DA}{short=DA,long=Domain Adaptation}
\DeclareAcronym{DBN}{short=DBN,long=Deep Belief Network}
\DeclareAcronym{DNN}{short=DNN,long=Deep Neural Network}
\DeclareAcronym{DL}{short=DL,long=Deep Learning}
\DeclareAcronym{DR}{short=DR,long=Diabetic Retinopathy}
\DeclareAcronym{ECG}{short=ECG,long=Electrocardiogram}
\DeclareAcronym{EEG}{short=EEG,long=Electroencephalogram}
\DeclareAcronym{ELMo}{short=ELMo,long=Embeddings from Language Models}
\DeclareAcronym{FC}{short=FC, long=Fully Connected}
\DeclareAcronym{FP}{short=FP,long=False Positive}
\DeclareAcronym{FN}{short=FN,long=False Negative}
\DeclareAcronym{F1}{short=F1,long=F1-Score}
\DeclareAcronym{FPN}{short=FPN,long=Feature Pyramid Network}
\DeclareAcronym{FPR}{short=FPR,long=False Positive Rate}
\DeclareAcronym{GAN}{short=GAN,long=Generative Adversarial Network}
\DeclareAcronym{GRU}{short=GRU,long=Gated Recurrent Unit}
\DeclareAcronym{GPU}{short=GPU,long=Graphics Processing Unit}
\DeclareAcronym{IoU}{short=IoU,long=Intersection over Union}
\DeclareAcronym{KNN}{short=KNN,long=k-Nearest Neighbors}
\DeclareAcronym{LBP}{short=LBP,long=Local Binary Pattern}
\DeclareAcronym{LDA}{short=LDA,long=Linear Discriminant Analysis}
\DeclareAcronym{LIME}{short=LIME,long=Local Interpretable Model-agnostic Explanations}
\DeclareAcronym{LSTM}{short=LSTM,long=Long Short-Term Memory}
\DeclareAcronym{LoRA}{short=LoRA,long=Low-Rank Adaptation}
\DeclareAcronym{MAE}{short=MAE,long=Mean Absolute Error}
\DeclareAcronym{mAP}{short=mAP,long=mean Average Precision}
\DeclareAcronym{MSE}{short=MSE,long=Mean Squared Error}
\DeclareAcronym{ML}{short=ML,long=Machine Learning}
\DeclareAcronym{MTL}{short=MTL,long=Multi-Task Learning}
\DeclareAcronym{MTTL}{short=MTTL,long=Multi-Task Transfer Learning}
\DeclareAcronym{MRI}{short=MRI,long=Magnetic Resonance Imaging}
\DeclareAcronym{NLP}{short=NLP,long=Natural Language Processing}
\DeclareAcronym{NLM}{short=NLM,long=National Library of Medecine}
\DeclareAcronym{NIH}{short=NIH,long=National Institutes of Health}
\DeclareAcronym{PCA}{short=PCA,long=Principal Component Analysis}
\DeclareAcronym{PDF}{short=PDF,long=Probability Density Function}
\DeclareAcronym{PEFT}{short=PEFT,long=Parameter-Efficient Fine-Tuning}
\DeclareAcronym{PSNR}{short=PSNR,long=Peak Signal-to-Noise Ratio}
\DeclareAcronym{RBC}{short=RBC,long=Red Blood Cell}
\DeclareAcronym{RDT}{short=RDT,long=Rapid Diagnostic Test}
\DeclareAcronym{ReLU}{short=ReLU,long=Rectified Linear Unit}
\DeclareAcronym{ResNet}{short=ResNet,long=Residual Network}
\DeclareAcronym{RoI}{short=RoI,long=Region of Interest}
\DeclareAcronym{RPN}{short=RPN,long=Region Proposal Network}
\DeclareAcronym{ROC}{short=ROC,long=Receiver Operating Characteristic}
\DeclareAcronym{SGD}{short=SGD,long=Stochastic Gradient Descent}
\DeclareAcronym{STL}{short=STL,long=Single-Task Learning}
\DeclareAcronym{SVM}{short=SVM,long=Support Vector Machine}
\DeclareAcronym{TL}{short=TL,long=Transfer Learning}
\DeclareAcronym{TP}{short=TP,long=True Positive}
\DeclareAcronym{TN}{short=TN,long=True Negative}
\DeclareAcronym{TPR}{short=TPR,long=True Positive Rate}
\DeclareAcronym{TPU}{short=TPU,long=Tensor Processing Unit}
\DeclareAcronym{VGG}{short=VGG,long=Visual Geometry Group Network}
\DeclareAcronym{ViT}{short=ViT,long=Vision Transformer}
\DeclareAcronym{WBC}{short=WBC,long=White Blood Cell}
\DeclareAcronym{WHO}{short=WHO,long=World Health Organization}
\DeclareAcronym{XAI}{short=XAI,long=Explainable Artificial Intelligence}
\DeclareAcronym{YOLO}{short=YOLO,long=You Only Look Once}
\DeclareAcronym{DenseNet}{short=DenseNet,long=Densely Connected Convolutional Network}
\DeclareAcronym{Incep}{short=Incep,long=Inception Network}
\DeclareAcronym{UNet}{short=UNet,long=U-Net Convolutional Network}
\DeclareAcronym{Optuna}{short=Optuna,long=Hyperparameter Optimization Framework}
\DeclareAcronym{FasterRCNN}{short=FasterRCNN,long=Faster Region-Based Convolutional Neural Network}
\DeclareAcronym{MMDetection}{short=MMDetection,long=OpenMMLab Detection Toolbox}
\DeclareAcronym{COCO}{short=COCO,long=Common Objects in Context Dataset}
\DeclareAcronym{ImageNet}{short=ImageNet,long=ImageNet Large Scale Visual Recognition Dataset}
\DeclareAcronym{GANomaly}{short=GANomaly,long=Generative Adversarial Anomaly Detection}
\DeclareAcronym{R-CNN}{short=R-CNN,long=Region-Based Convolutional Neural Network}
\DeclareAcronym{SSD}{short=SSD,long=Single Shot Detector}
\DeclareAcronym{ViViT}{short=ViViT,long=Video Vision Transformer}
\DeclareAcronym{MixUp}{short=MixUp,long=Data Augmentation Technique}
\DeclareAcronym{CutMix}{short=CutMix,long=Data Augmentation Technique}
\DeclareAcronym{SWIN}{short=SWIN,long=Shifted Window Transformer}
\DeclareAcronym{SAM}{short=SAM,long=Segment Anything Model}
\DeclareAcronym{CLIP}{short=CLIP,long=Contrastive Language-Image Pretraining}
\DeclareAcronym{MAE-MLM}{short=MAE-MLM,long=Masked Autoencoder with Masked Language Modeling}
\DeclareAcronym{SwinUNet}{short=SwinUNet,long=Swin-UNet Network for Segmentation}
\DeclareAcronym{GradCAM}{short=GradCAM,long=Gradient-weighted Class Activation Mapping}
\DeclareAcronym{Dice}{short=Dice,long=Dice Similarity Coefficient}
\DeclareAcronym{IoBB}{short=IoBB,long=Intersection over Bounding Box}
\DeclareAcronym{Ablation}{short=Ablation,long=Ablation Study}
\DeclareAcronym{mDice}{short=mDice,long=Mean Dice Score}
\DeclareAcronym{FocalLoss}{short=FocalLoss,long=Focal Loss Function}
\DeclareAcronym{DiceLoss}{short=DiceLoss,long=Dice Loss Function}
\DeclareAcronym{AdamW}{short=AdamW,long=Adam Weight Decay Optimizer}
\DeclareAcronym{SGDR}{short=SGDR,long=Stochastic Gradient Descent with Restarts}
\DeclareAcronym{TTA}{short=TTA,long=Test Time Augmentation}
\DeclareAcronym{EMA}{short=EMA,long=Exponential Moving Average}
\DeclareAcronym{LR}{short=LR,long=Learning Rate}
\DeclareAcronym{CSAM}{short=CSAM,long=Channel Spatial Attention Module}
\DeclareAcronym{PSP}{short=PSP,long=Pyramid Scene Parsing Network}
\DeclareAcronym{DeepLabV3}{short=DeepLabV3,long=DeepLabV3 Semantic Segmentation Network}
\DeclareAcronym{MS-SSIM}{short=MS-SSIM,long=Multi-Scale Structural Similarity Index}
\DeclareAcronym{HPA}{short=HPA,long=Human Protein Atlas}
\DeclareAcronym{TCGA}{short=TCGA,long=The Cancer Genome Atlas}
\DeclareAcronym{FISH}{short=FISH,long=Fluorescence In Situ Hybridization}
\DeclareAcronym{PBS}{short=PBS,long=Phosphate Buffered Saline}
\DeclareAcronym{HE}{short=HE,long=Hematoxylin and Eosin Staining}
\DeclareAcronym{IFA}{short=IFA,long=Immunofluorescence Assay}
\DeclareAcronym{qPCR}{short=qPCR,long=Quantitative Polymerase Chain Reaction}
\DeclareAcronym{NGS}{short=NGS,long=Next-Generation Sequencing}
\DeclareAcronym{CIFAR}{short=CIFAR,long=CIFAR Image Classification Dataset}
\DeclareAcronym{MNIST}{short=MNIST,long=Modified National Institute of Standards and Technology Dataset}
\DeclareAcronym{BloodSmear}{short=BloodSmear,long=Peripheral Blood Smear}
\DeclareAcronym{PBSM}{short=PBSM,long=Peripheral Blood Smear Microscopy}
\DeclareAcronym{ParasiteDensity}{short=PD,long=Parasite Density Measurement}
\DeclareAcronym{LifeCycleStage}{short=LCS,long=Parasite Life-Cycle Stage}
\DeclareAcronym{DataAug}{short=DataAug,long=Data Augmentation}
\DeclareAcronym{RandAug}{short=RandAug,long=Random Augmentation}
\DeclareAcronym{SegmMask}{short=SegmMask,long=Segmentation Mask}
\DeclareAcronym{TrainSet}{short=TrainSet,long=Training Dataset}
\DeclareAcronym{ValSet}{short=ValSet,long=Validation Dataset}
\DeclareAcronym{TestSet}{short=TestSet,long=Testing Dataset}
\DeclareAcronym{MLOps}{short=MLOps,long=Machine Learning Operations}
\DeclareAcronym{Docker}{short=Docker,long=Docker Container Platform}
\DeclareAcronym{K8s}{short=K8s,long=Kubernetes}
\DeclareAcronym{CI}{short=CI,long=Continuous Integration}
\DeclareAcronym{CD}{short=CD,long=Continuous Deployment}
\DeclareAcronym{MLFlow}{short=MLFlow,long=Machine Learning Lifecycle Management Platform}
\DeclareAcronym{OpticalFlow}{short=OF,long=Optical Flow}
\DeclareAcronym{SIFT}{short=SIFT,long=Scale-Invariant Feature Transform}
\DeclareAcronym{SURF}{short=SURF,long=Speeded Up Robust Features}
\DeclareAcronym{HOG}{short=HOG,long=Histogram of Oriented Gradients}
\DeclareAcronym{ORB}{short=ORB,long=Oriented FAST and Rotated BRIEF}
\DeclareAcronym{YOLOv5}{short=YOLOv5,long=You Only Look Once Version 5}
\DeclareAcronym{YOLOv8}{short=YOLOv8,long=You Only Look Once Version 8}
\DeclareAcronym{FPNLite}{short=FPNLite,long=Lightweight Feature Pyramid Network}
\DeclareAcronym{CIF}{short=CIF,long=Canadian Institute for Forestry} % Example of extra
\DeclareAcronym{TPM}{short=TPM,long=Transcripts Per Million}
\DeclareAcronym{GNN}{short=GNN,long=Graph Neural Network}
\DeclareAcronym{MPNN}{short=MPNN,long=Message Passing Neural Network}
\DeclareAcronym{EdgeAI}{short=EdgeAI,long=Edge Artificial Intelligence}
\DeclareAcronym{TinyML}{short=TinyML,long=Tiny Machine Learning}
\DeclareAcronym{FPGA}{short=FPGA,long=Field-Programmable Gate Array}
\DeclareAcronym{ASIC}{short=ASIC,long=Application-Specific Integrated Circuit}
\DeclareAcronym{RL}{short=RL,long=Reinforcement Learning}
\DeclareAcronym{PPO}{short=PPO,long=Proximal Policy Optimization}
\DeclareAcronym{SAC}{short=SAC,long=Soft Actor-Critic}
\DeclareAcronym{DQN}{short=DQN,long=Deep Q-Network}
\DeclareAcronym{HER}{short=HER,long=Hindsight Experience Replay}
\DeclareAcronym{MCTS}{short=MCTS,long=Monte Carlo Tree Search}
\DeclareAcronym{OCT}{short=OCT,long=Optical Coherence Tomography}
\DeclareAcronym{fMRI}{short=fMRI,long=functional Magnetic Resonance Imaging}
\DeclareAcronym{DTI}{short=DTI,long=Diffusion Tensor Imaging}
\DeclareAcronym{EEGNet}{short=EEGNet,long=Deep Learning Network for EEG Signals}
\DeclareAcronym{EMG}{short=EMG,long=Electromyography}
\DeclareAcronym{PET}{short=PET,long=Positron Emission Tomography}
\DeclareAcronym{SPECT}{short=SPECT,long=Single Photon Emission Computed Tomography}
\DeclareAcronym{CAD}{short=CAD,long=Computer-Aided Diagnosis}
\DeclareAcronym{RIS}{short=RIS,long=Radiology Information System}
\DeclareAcronym{PACS}{short=PACS,long=Picture Archiving and Communication System}
\DeclareAcronym{DICOM}{short=DICOM,long=Digital Imaging and Communications in Medicine}
\DeclareAcronym{HL7}{short=HL7,long=Health Level 7 Standard}
\DeclareAcronym{FHIR}{short=FHIR,long=Fast Healthcare Interoperability Resources}
\DeclareAcronym{LIFEx}{short=LIFEx,long=Radiomics Feature Extraction Software}
\DeclareAcronym{PyRadiomics}{short=PyRad,long=Python Radiomics Library}
\DeclareAcronym{VAE}{short=VAE,long=Variational Autoencoder}
\DeclareAcronym{CVAE}{short=CVAE,long=Conditional Variational Autoencoder}
\DeclareAcronym{cGAN}{short=cGAN,long=Conditional Generative Adversarial Network}
\DeclareAcronym{InfoGAN}{short=InfoGAN,long=Information Maximizing GAN}
\DeclareAcronym{SRGAN}{short=SRGAN,long=Super-Resolution GAN}
\DeclareAcronym{SR}{short=SR,long=Super-Resolution}
\DeclareAcronym{ESRGAN}{short=ESRGAN,long=Enhanced SRGAN}
\DeclareAcronym{MUNIT}{short=MUNIT,long=Multimodal Unsupervised Image-to-Image Translation}
\DeclareAcronym{CycleGAN}{short=CycleGAN,long=Cycle-Consistent GAN}
\DeclareAcronym{Pix2Pix}{short=Pix2Pix,long=Paired Image-to-Image Translation GAN}
\DeclareAcronym{UDA}{short=UDA,long=Unsupervised Domain Adaptation}
\DeclareAcronym{FDA}{short=FDA,long=Fourier Domain Adaptation}
\DeclareAcronym{WD}{short=WD,long=Weight Decay}
\DeclareAcronym{SWAG}{short=SWAG,long=Stochastic Weight Averaging Gaussian}
\DeclareAcronym{LaplaceApprox}{short=LapApprox,long=Laplace Approximation}
\DeclareAcronym{MC-Dropout}{short=MCD,long=Monte Carlo Dropout}
\DeclareAcronym{KD}{short=KD,long=Knowledge Distillation}
\DeclareAcronym{SOTA}{short=SOTA,long=State-of-the-Art}
\DeclareAcronym{PoissonLoss}{short=PoissonLoss,long=Poisson Negative Log-Likelihood Loss}
\DeclareAcronym{Bbox}{short=Bbox,long=Bounding Box}
\DeclareAcronym{Heatmap}{short=Heatmap,long=Localization Heatmap}
\DeclareAcronym{WSI}{short=WSI,long=Whole Slide Imaging}
\DeclareAcronym{CLAHE}{short=CLAHE,long=Contrast Limited Adaptive Histogram Equalization}


\acsetup{
	make-links = true,
	first-style = long-short,
	list/name = {List of Acronyms},
	list/heading = none,
	list/sort = true,
	list/template = description,
	pages/display = first
}

% -------------------------------
% THEOREMS
% -------------------------------
\newtheorem{exo}{{Exercice}}[chapter]
\newtheorem{remark}{{Remarque}}[chapter]
\newtheorem{definition}{Définition}[chapter]
\newtheorem{example}{{Exemple}}[chapter]

\renewcommand{\cftchapleader}{\cftdotfill{\cftdotsep}}

\makeatletter
\patchcmd{\@chapter}{\addcontentsline{toc}{chapter}{#1}}{}{}{}
\patchcmd{\@section}{\addcontentsline{toc}{section}{#1}}{}{}{}
\patchcmd{\@subsection}{\addcontentsline{toc}{subsection}{#1}}{}{}{}
\patchcmd{\@subsubsection}{\addcontentsline{toc}{subsubsection}{#1}}{}{}{}
\makeatother

\begin{document}

% TITLE PAGE
\begin{titlepage}
\centering
\fontfamily{lmss}\selectfont 
\begin{minipage}{0.45\linewidth} 
    \raggedright 
    \includegraphics[height=2.8cm]{Logos/UD.png}
\end{minipage} \hfill 
\begin{minipage}{0.45\linewidth} 
    \raggedleft 
    \includegraphics[height=2.8cm]{Logos/ENSPD LOGO.jpeg}
\end{minipage}

\vspace{-3.5cm}
{\scriptsize
\begin{center}
\textbf{REPUBLIC OF CAMEROON} \\
Peace – Work – Fatherland \\
\textcolor{gray}{$\diamond\diamond\diamond\diamond\diamond$} \\
\textbf{MINISTRY OF HIGHER EDUCATION} \\
\textcolor{gray}{$\diamond\diamond\diamond\diamond\diamond$} \\
\textbf{THE UNIVERSITY OF DOUALA} \\
\textcolor{gray}{$\diamond\diamond\diamond\diamond\diamond$} \\
\textbf{National Higher Polytechnic School of Douala (NHPSD)}
\end{center}
}

\vspace{1cm}
{\large Laboratory of Computer Engineering,}\\
\vspace{0.1cm}
{\large Data Science and Artificial Intelligence (LCEDSAI)}\\[0.5cm]

\begin{center}
	\tcbset{colback=white, colframe=black, boxrule=0.5pt, arc=3pt}
	\tcbox{\textbf{Academic Year: 2024–2025}}
\end{center}

\rule{\linewidth}{1.2pt}\\[0.6cm]
{\Huge \bfseries Multi-Task Transfer Learning}\\[0.3cm]
{\LARGE And Application to Malaria Detection}\\[0.6cm]
\rule{\linewidth}{1.2pt}\\

\vspace{0.4cm}
{\normalsize Presented by}\\
\vspace{0.2cm}
{\LARGE \textbf{Gwade Steve}}\\

\vspace{1cm}
{\large Submitted in partial fulfillment of the requirements}\\
{\large for the degree of}\\[0.8cm]

{\Large \textbf{Master of Science}}\\
\vspace{0.1cm}
{\large in Data Science and Artificial Intelligence}\\

\vspace{0.25cm}

\begin{center}
\textbf{Supervised by:} \\
Prof. Dr. habil. P. Njionou Sadjang \\[0.4cm]
\textbf{Co-supervised by:} \\
Dr. Armielle Noulapeu
\end{center}

\vfill
{\large National Higher Polytechnic School of Douala}\\
{\large University of Douala, Cameroon}\\[1.2cm]
{\large \today}
\end{titlepage}

\pagestyle{fancy} 
\pagenumbering{roman}

% -------------------------------
% DECLARATION
% -------------------------------
\chapter*{Declaration}
\thispagestyle{plain}
\addcontentsline{toc}{chapter}{Declaration}
\thispagestyle{plain}
\noindent I, \textbf{Gwade Steve}, hereby solemnly declare that this Master’s thesis entitled \textit{“Multi-Task Transfer Learning and Application to Malaria Detection”} is the result of my own research and investigations carried out under the guidance and supervision of Prof. Dr. habil. P. Njionou Sadjang and Dr. Armielle Noulapeu at the National Higher Polytechnic School of Douala (NHPSD), University of Douala, Cameroon.\\

\noindent I declare that, to the best of my knowledge, this work has not been submitted, either in whole or in part, for any other degree, diploma, or certification at any other academic institution.\\

\noindent I further declare that all sources of information, ideas, and data used in this thesis have been fully acknowledged. Any contributions from other researchers, authors, or institutions are clearly referenced in accordance with academic conventions.\\

\noindent I accept full responsibility for the integrity and authenticity of the research and the results presented herein. I affirm that any errors or omissions are entirely my own and do not reflect the work of my supervisors or any other individual.\\

\vspace{3cm}
\begin{flushright}
\begin{tabular}{l}
Signed: \underline{\hspace{8cm}} \\[1cm]
Date: \underline{\hspace{8cm}} \\[1cm]
Place: \underline{\hspace{8cm}}
\end{tabular}
\end{flushright}

% -------------------------------
% ABSTRACT & ACKNOWLEDGMENTS
% -------------------------------

\chapter*{Acknowledgments}
\thispagestyle{plain}
\addcontentsline{toc}{chapter}{Acknowledgments}
\thispagestyle{plain}

\noindent 
First, I place my gratitude to God for the strength, health, and guidance granted to me throughout this path.\\

\noindent 
I would like to express my deepest appreciation to my supervisor, \textbf{Prof. Dr. habil. PATRICK NJIONOU SADJANG}. Thank you for your trust, your patience, and the wisdom you shared at every stage of this research. Your mentorship was vital in shaping not just this work, but my growth as an engineer and as a researcher.\\

\noindent 
A special thanks goes to my co-supervisor, \textbf{Dr. ARMIELLE NOULAPEU}, for her sharp eye and honest feedback. Her critiques significantly elevated the quality of this work. I am also indebted to \textbf{LIONEL OWONO}, \textbf{ALEX BRUNO KENFACK}, \textbf{YVAN NGNEUNMEU} and \textbf{Dr. HIOL NICOLAS} for taking the time to review my drafts and offering insights that helped me see things more clearly.\\

\noindent 
To my colleagues and labmates, thank you for the conversations and debates that helped refine the ideas presented here. I also want to give a specific shout-out to my friends \textbf{NYEMB NDJEM EONE ANDRE KEVIN} and \textbf{NJIMEYUP HAROLD FRANCOIS} for their support and feedback.\\

\noindent 
Most importantly, my heart goes out to my family. To my mother, \textbf{NGO HIOLL VERONIQUE}, and my father, \textbf{GWADE ELIAS}, thank you for your endless love and for the sacrifices you made to get me here. Finally, to my girlfriend, thank you for your understanding, and for keeping me motivated during the hardest moments of this journey.

\chapter*{Abstract}
\thispagestyle{plain}
\addcontentsline{toc}{chapter}{Abstract}
\thispagestyle{plain}

\noindent
Automated malaria diagnosis remains a major challenge for researchers. While many deep learning methods exist, most focus on simplified goals like binary classification or just finding infected cells in a blood smear. These specialized models often achieve high scores on benchmarks, but they struggle in real diagnostic settings that require identifying cells, locating parasites, and measuring infection levels all at once.\\

\noindent This thesis proposes a \textbf{Multi-Task Transfer Learning (MTTL)} framework designed to perform a complete diagnostic analysis from single thin blood smear images. The system uses Low-Rank Adaptation (LoRA) and partial fine-tuning to learn four tasks together: multi-class detection, cell segmentation, infection localization, and instance-level classification.

\bigskip

\noindent
Our experiments show that standard Single-Task Learning (STL) works well for simple detection but fails as the complexity increases. We observed the F1 score drop from $0.82$ in simple one-class tests to $0.38$ in three-class settings. In contrast, our best MTTL model, using RoI Classification, achieved a 103\% relative improvement in detecting infected cells. It also reduced the error in parasitemia estimation from $3.30\%$ down to $1.08\%$, a 67\% reduction, while producing fewer false alarms and sharper localization results.

\bigskip

\noindent
These findings suggest that combining Transfer Learning, Multi-Task Learning, and Parameter-Efficient Fine-Tuning creates a strong foundation for complex medical imaging tasks. This framework improves reliability and makes results easier to interpret, offering a practical strategy for developing diagnostic models that can be adapted to other biomedical fields. The complete implementation and trained models have been released publicly.

\bigskip

\noindent
\textbf{Keywords:} Malaria Diagnosis, Multi-Task Learning, Transfer Learning, Computational Pathology, Deep Learning, Blood Smear Analysis, Object Detection, LoRA, Fine-Tuning

\bigskip

\noindent
\textbf{Code Repository:} \href{https://github.com/GwadeSteve/MTTL-Research-Malaria}{https://github.com/GwadeSteve/MTTL-Research-Malaria}

\chapter*{Summary}
\thispagestyle{plain}
\addcontentsline{toc}{chapter}{Summary}
\thispagestyle{plain}

\noindent \textbf{Context and Motivation.}  
Malaria remains one of the most persistent infectious diseases in the world and continues to affect millions of people each year, particularly in tropical regions such as Sub-Saharan Africa. Microscopic examination of thin blood smears is still considered the reference method for diagnosis because it allows direct visualization of parasites. However, this process is slow, requires highly skilled technicians, and can produce inconsistent results in low-resource environments. In recent years, machine learning methods have been explored to support and automate malaria diagnosis. Despite promising progress, most models have been developed for very specific objectives such as binary detection of infected cells or parasite classification, without considering the complete diagnostic process.\\

\noindent \textbf{Problem Statement and Objectives.}  
This observation reveals an important research gap, the lack of unified computational models able to handle the full complexity of malaria diagnosis within a single framework. In practice, diagnosis involves several interrelated steps such as identifying infected cells, segmenting their contours, locating the parasite, and estimating the level of infection. Existing approaches usually address these tasks separately, which limits their robustness when applied to new data or varied clinical conditions. The objective of this work is therefore to design, implement, and evaluate a comprehensive diagnostic framework based on \textbf{Multi-Task Transfer Learning (MTTL)}. The goal is to develop a system that learns several related tasks at the same time and produces reliable and complete diagnostic predictions from a single blood smear image.\\

\noindent \textbf{Methodology.}  
We begin by defining a mathematical framework that combines three existing concepts: \textbf{Transfer Learning (TL)}, \textbf{Multi-Task Learning (MTL)}, and \textbf{Parameter-Efficient Fine-Tuning (PEFT)}. This theoretical basis provides a structured way to adapt pre-trained models to multiple interdependent tasks. Based on this formulation, we implemented a hybrid approach using Low-Rank Adaptation (LoRA) together with partial fine-tuning of the backbone network. The proposed system learns four related tasks at once: cell detection, binary cell segmentation, parasite localization, and instance-level cell classification. Experiments were conducted using the public NLM thin falciparum blood smear dataset in both single-task and multi-task configurations, making it possible to directly compare the performance of \textbf{Single-Task Learning (STL)} and the proposed \textbf{MTTL} approach.\\

\noindent \textbf{Results and Analysis.}  
The results clearly show the advantage of the multi-task strategy. Single-task models perform well on simple detection but lose stability when the problem becomes more complex. For example, the F1 score for detecting infected cells decreases from $0.82$ in the single-class case to $0.38$ when three classes are considered. The multi-task model, regularized by auxiliary tasks, remains much more stable. We also benchmarked our system against \textbf{YOLOv8-Small}, a standard model for real-time detection. Our framework achieved better clinical metrics, improving the infected-class F1-score by \textbf{34.7\%} and reducing the parasitemia estimation error by \textbf{71.4\%}. Visual inspection further confirms that MTTL produces cleaner detection maps, fewer false positives on healthy slides, and accurate localization.\\

\noindent \textbf{Conclusion and Perspectives.}  
These findings confirm that combining Transfer Learning, Multi-Task Learning, and Parameter-Efficient Fine-Tuning forms a strong foundation for complex medical image analysis. The proposed MTTL model improves both diagnostic accuracy and interpretability by linking related visual cues across tasks. Although this study focuses on malaria detection, the same framework could be extended to other biomedical imaging domains that require the joint learning of related objectives. Future work may include validation on additional datasets, evaluation under domain shift conditions, and a more detailed theoretical study of task transferability and generalization.\\

\noindent
\textbf{Keywords:} Malaria Diagnosis, Multi-Task Learning, Transfer Learning, Deep Learning, Parameter-Efficient Fine-Tuning, LoRA, Computational Pathology\\

\noindent
\textbf{Code Repository:} \href{https://github.com/GwadeSteve/MTTL-Research-Malaria}{https://github.com/GwadeSteve/MTTL-Research-Malaria}

% TOC
\newpage
\tableofcontents
\cleardoublepage

\chapter*{List of Acronyms}
\addcontentsline{toc}{chapter}{List of Acronyms}
{\small
	\setlength{\columnsep}{1cm} 
	\begin{multicols*}{2}
		\raggedright
		\let\olddescription\description
		\renewenvironment{description}{
			\olddescription
			\setlength{\itemsep}{0pt}   
			\setlength{\parsep}{0pt}    
			\setlength{\topsep}{0pt}    
			\setlength{\partopsep}{0pt} 
		}{\endlist}
		\printacronyms
	\end{multicols*}
}
\cleardoublepage

% LIST OF FIGURES
\chapter*{List of Figures}
\addcontentsline{toc}{chapter}{List of Figures}
\vspace*{-2.5cm}
\begingroup
\normalsize
\setlength{\cftfignumwidth}{2.5em}      
\setlength{\cftfigindent}{0pt}          
\setlength{\cftbeforefigskip}{0pt}      
\setlength{\cftaftertoctitleskip}{0pt}  
\renewcommand{\listfigurename}{}        
\listoffigures
\endgroup
\cleardoublepage

% LIST OF TABLES
\chapter*{List of Tables}
\addcontentsline{toc}{chapter}{List of Tables}
\vspace*{-2.5cm}  
\begingroup
\normalsize
\setlength{\cfttabnumwidth}{2.5em}       
\setlength{\cfttabindent}{0pt}           
\setlength{\cftbeforetabskip}{0pt}       
\setlength{\cftaftertoctitleskip}{0pt}   
\renewcommand{\listtablename}{}          
\listoftables
\endgroup
\cleardoublepage

% LIST OF EQUATIONS 
\chapter*{List of Equations}
\addcontentsline{toc}{chapter}{List of Equations}
\vspace*{-2.5cm} 
\begingroup
\normalsize
\setlength{\cftmyequationsnumwidth}{2.5em}  
\setlength{\cftmyequationsindent}{0pt}      
\setlength{\cftbeforemyequationsskip}{0pt}  
\renewcommand{\listequationsname}{}         
\listofmyequations
\endgroup
\cleardoublepage

% LIST OF ALGORITHMS
% The algorithm2e package uses a different command for its list
%\listofalgorithms
%\addcontentsline{toc}{chapter}{List of Algorithms}
%\cleardoublepage

% START OF MAIN CONTENT -
\pagenumbering{arabic}
\setcounter{page}{1}

% -------------------------------
% CHAPTERS
% -------------------------------

% INTRODUCTION
\chapter*{Introduction}
\addcontentsline{toc}{chapter}{Introduction}
\label{chap:intro}

Malaria remains a persistent global public health challenge. Transmitted by infected female \textit{Anopheles mosquitoes}, this parasitic disease places a disproportionate burden on low and middle-income countries, particularly in sub-Saharan Africa. The 2024 \textit{World Malaria Report} estimates there were nearly \textbf{263 million} cases and over \textbf{597,000 deaths} in 2023, with children under five and pregnant women bearing the highest risk \parencite{WHO2024GlobalMalariaReport}. Beyond direct health consequences, malaria slows economic development and strains healthcare systems, perpetuating cycles of poverty \parencite{Sachs2002MacroeconomicsHealth}. Despite decades of eradication efforts including insecticide-treated nets and preventive therapies the disease continues to pose a major obstacle to global health.\\

\noindent
Effective control relies on early, reliable, and widely available diagnosis. Missed or delayed diagnoses lead to incorrect treatment, increased mortality, and continued transmission \parencite{WHO2024GlobalMalariaReport}. However, diagnostic consistency remains a significant hurdle. Microscopy is highly accurate but demands trained technicians, stable electricity, and equipped laboratories resources often scarce in rural areas \parencite{Wongsrichanalai2007}. Rapid diagnostic tests (RDTs) offer an accessible alternative, yet their performance is inconsistent. Field studies in Cameroon highlight this variability, some tests show high specificity but low sensitivity \parencite{Ngalame2025}, others demonstrate the reverse, leading to high false-positive rates \parencite{Missoup2025}. Furthermore, factors such as storage conditions and parasite genetic variations, including \textit{pfhrp2/3} deletions, jeopardize RDT reliability \parencite{Gatton2020,Prosser2021}.\\

\noindent
These limitations urges the need for diagnostic tools that balance accuracy, speed, and field suitability. Digital microscopy and AI-based image analysis show promise in reducing reliance on specialized infrastructure while making precise diagnosis accessible \parencite{Poostchi2018,Maturana2022,WHO2024GlobalMalariaReport}.

\section*{The Malaria Diagnostic Challenge}
\label{sec:malaria_challenge}
Traditional diagnosis relies on microscopy and \acp{RDT}, both of which face operational constraints. Microscopy (Figure~\ref{fig:intro_malaria_diag_methods}) is the gold standard \parencite{WHO2015MalariaMicroscopyQA}. When performed by experts, it identifies parasite species and infection severity. However, its efficacy declines in real-world settings due to technician shortages, variable slide preparation, and the time-intensive nature of manual examination.\\

\begin{figure}[htbp]
	\centering
	\begin{subfigure}[b]{0.48\textwidth}
		\centering
		\includegraphics[width=\linewidth, height=4cm, keepaspectratio]{Images/MalariaMicroscope.jpeg}
		\caption{Manual microscopy of a blood smear.}
		\label{fig:microscopy_malaria}
	\end{subfigure}
	\hfill 
	\begin{subfigure}[b]{0.48\textwidth}
		\centering
		\includegraphics[width=\linewidth, height=4cm, keepaspectratio]{Images/RDT.jpeg}
		\caption{Rapid Diagnostic Test (RDT).}
		\label{fig:rdt_malaria}
	\end{subfigure}
	\caption{Common methods for malaria diagnosis.}
	\label{fig:intro_malaria_diag_methods}
\end{figure}

\noindent
Conversely, \acp{RDT} are cost-effective and easy to deploy in underserved regions. Yet, they lack the ability to quantify parasitemia or identify specific species and may fail to detect low-density infections or specific strains \parencite{Eyong2022Plasmodium, Jacobs2014}. The resulting \acp{FN} and \acp{FP} contribute to untreated cases or drug resistance through misuse. Combined with supply chain interruptions and high seasonal caseloads, these limitations leave large populations vulnerable to inaccurate diagnosis.

\section*{The Potential of AI in Medical Imaging}
\label{sec:potential_AI}

Advances in \ac{AI}, particularly \ac{CV} and \ac{DL}, offer new pathways for medical imaging. Models based on \acp{CNN} and transformer architectures can analyze data with consistency and precision that rivals human experts. Systems based on these architectures have reached near expert performance in tasks such as detecting tumors in X-rays, grading retinal diseases in fundus photographs \parencite{Gulshan2016DevelopmentValidationDLAlgorithm}, and classifying skin lesions from dermoscopic images \parencite{Esteva2017DermatologistlevelClassificationSkin}. In the context of malaria, \ac{AI} offers distinct advantages over traditional methods.\\

\noindent
First, \ac{AI} models can automatically detect parasites in blood smear images, reducing dependency on human expertise and accelerating turnaround times. Second, unlike standard \acp{RDT}, \acp{AI} can classify species and estimate parasite density metrics essential for appropriate treatment. Third, these systems can be embedded into low-cost digital microscopes or mobile applications, extending reach to low-resource environments. While currently limited to proof-of-concept studies \parencite{Rajaraman2018NIHMalariaDataset}, \ac{AI} offers scalability and adaptability, models can be retrained to recognize new strains or regional variations. By combining automated analysis with telemedicine, \ac{AI} can serve as a powerful complementary tool to strengthen malaria control programs.

\section*{Research Problem and Motivation}
\label{sec:problem_motivation}
While \ac{DL} and \ac{TL} have shown promise in detecting \textit{parasitized cells}, current systems often fall short of clinical utility. Most approaches focus on binary classification (infected vs. uninfected) or parasitized cell detection, whereas a reliable diagnosis requires locating parasites, distinguishing cell types (e.g., \acp{RBC}, \acp{WBC}), estimating parasitemia and even distinguishing parasite specie. The current literature of \ac{STL} approaches do not deliver this comprehensive information from a single smear using a single model.\\

\noindent
\acl{MTL}, a paradigm designed to learn related tasks simultaneously, has been largely overlooked in this domain. We go even further and propose that combining \ac{MTL} with the prior knowledge of \ac{TL} creates a superior hybrid strategy. To our knowledge, this combined paradigm \textbf{\ac{MTTL}} has not been formally applied to malaria diagnosis. This thesis addresses two central problems:

\begin{enumerate}
	\item \textbf{Practical Problem:} Existing works in the literature address only isolated aspects of diagnosis (e.g., infected cell detection, binary classification), providing clinicians with partial outputs rather than a diagnostic system.
	\item \textbf{Theoretical Problem:} There is no clear formulation in current literature unifying \ac{TL}, \ac{MTL}, and modern \ac{PEFT} techniques for medical image analysis. Our work closes this gap by proposing a unified formulation and using malaria detection as a representative case.
\end{enumerate}

\section*{Objectives and Research Questions}
\label{sec:objectives_rq}
The aim of this research is to design and empirically validate a \ac{MTTL} framework for automated malaria diagnosis.

\begin{itemize}
	\item \textbf{Main Objective:} Design a \ac{MTTL} approach and validate its effectiveness in malaria diagnosis.  
	\item \textbf{Specific Objectives:}
	\begin{itemize}
		\item Formalize and propose a mathematical formulation of \ac{MTTL}.  
		\item Apply the proposed framework to malaria diagnosis tasks.  
		\item Compare the performance of \ac{MTTL} against \ac{STL} baselines.  
		\item Evaluate the contribution of auxiliary tasks to identify optimal configurations.  
	\end{itemize}
\end{itemize}
\noindent \textbf{Research Question:} \textit{Does an \ac{MTTL} approach provide superior efficacy and robustness compared to \ac{STL} for automated malaria diagnosis?} \\

\noindent This research question guided our work and led us to the conclusions drawn at the end of this work.

\section*{Hypotheses}
\label{sec:hypotheses}
\noindent\textbf{Main Hypothesis}:
\begin{quote}
	We hypothesize that initializing a shared backbone with relevant prior knowledge and jointly learning related tasks forces the model to learn a more powerful, generalizable shared representation, improving performance on individual tasks compared to independent training.
\end{quote}

\noindent\textbf{Supporting Hypotheses}
\begin{itemize}
	\item \textbf{H1 (\ac{MTTL} vs. \ac{STL})}: The \ac{MTTL} paradigm will outperform \ac{STL} baselines in diagnostic completeness and generalization, specifically in multi-class cell detection.
	\item \textbf{H2 (Auxiliary Task Synergy)}: Auxiliary tasks will act as effective regularizers.
	\item \textbf{H3 (Robustness)}: \ac{MTTL} models will demonstrate higher robustness on specialized subsets (Infected-Only and Healthy-Only) compared to \ac{STL} baselines.
\end{itemize}

\section*{Contributions and Thesis Outline}
\label{sec:contributions_outline}

This work contributes to medical imaging and \ac{ML} by providing:
\begin{itemize}
	\item \textbf{An \ac{MTTL} Framework:} A mathematical formulation combining \ac{TL}, \ac{MTL}, and \ac{PEFT} for medical image analysis.
	\item \textbf{A Diagnostic System:} The design and validation of a comprehensive, multi-task system for malaria diagnosis.
	\item \textbf{Empirical Evidence:} A rigorous study demonstrating the superiority of the \ac{MTTL} paradigm over \ac{STL} for complex diagnostics.
\end{itemize}

\noindent
The rest of this document is structured as follows. \textbf{Chapter \ref{chap:theory}} establishes the theoretical foundations of \ac{DL}, \acp{CNN}, \ac{TL}, \ac{MTL}, \ac{PEFT}, and the current state of literature for AI applied to medical imaging and malaria specifically. \textbf{Chapter \ref{chap:framework}} presents our core contribution, the mathematical framework for \ac{MTTL}. \textbf{Chapter \ref{chap:methodology}} details the methodological application to malaria diagnosis, the experimental implementation, including dataset and architecture. \textbf{Chapter \ref{chap:results}} reports the empirical findings and comparative analysis. Finally, \textbf{Chapter \ref{chap:discussion}} interprets the implications and limitations of the study, and \textbf{Chapter \ref{chap:conclusion}} summarizes findings and outlines future directions. 
\cleardoublepage

%LITERATURE REVIEW
\chapter{Theoretical Foundations and Literature Review}
\label{chap:theory}
Deep Learning (\ac{DL}), a branch of Machine Learning (\ac{ML}), has revolutionized how machines analyze complex data, especially visual information like medical images \parencite{Litjens2017SurveyDeepLearningMedical}. Its fundamental strength, as highlighted by reviews such as \textcite{alzubaidi2021review}, is the ability to learn representations of features directly from raw data, obviating the need for manual, domain-specific feature engineering. This chapter covers the foundational principles of \ac{DL} and Convolutional Neural Networks (\acp{CNN}). We will define their core components and mathematical foundations, describe the learning process, and discuss the principal challenges encountered when applying these models to medical data. This foundation is essential to understand why advanced learning paradigms like Transfer and Multitask Learning are not merely advantageous, but often necessary.

\section{Deep Neural Networks as Composite Functions}
\label{sec:deep_network_composition}

\ac{DL} models are a class of \acp{ANN} distinguished by their depth, typically referring to a significant number of layers. Mathematically, a deep network $h(x; \theta)$ is a composite function that maps an input $x$ to an output through a series of $L$ nested transformations.\\

\noindent
This process can be defined iteratively. Let the input be the zeroth-layer activation, $A^{(0)} = x$. The activation $A^{(i)}$ of the $i$-th layer is computed by applying a transformation $h^{(i)}$ (parameterized by $\theta_i$) to the activation of the preceding layer:
\begin{equation}
	A^{(i)} = h^{(i)}(A^{(i-1)}; \theta_i) \quad \text{for } i = 1, \dots, L
	\myequations{Iterative Layer Transformation}
	\label{eq:dnn_layer_transform}
\end{equation}

\noindent
The final output of the network is the activation of the last layer, $y = A^{(L)}$. This unrolled, iterative definition results in the fully composite function:
\begin{equation}
	h(x; \theta) = h^{(L)}(h^{(L-1)}(\dots h^{(1)}(x; \theta_1) \dots; \theta_{L-1}); \theta_L)
	\myequations{Deep Network Composite Function}
	\label{eq:dnn_forward_pass}
\end{equation}

\noindent
The complete set of model parameters, $\theta = \{\theta_1, \dots, \theta_L\}$, is learned from data via an optimization process. The hierarchical, multi-layered structure enables the model to discover and represent intricate structures within the data at multiple, increasing levels of abstraction.

\subsection{Hierarchical Representations}
\label{ssec:dl_learned_features_power}
Traditional \ac{ML} workflows depended on handcrafted features (e.g., SIFT \parencite{Lowe2004DistinctiveImageFeatures}, \ac{HOG} \parencite{Dalal2005HistogramsOrientedGradients}, \ac{LBP} \parencite{Ojala2002MultiresolutionGrayScaleRotation}), which required significant domain expertise and were often brittle. \ac{DL} models, and \acp{CNN} in particular, automate this process of feature discovery \parencite{LeCun2015DeepLearning}. The hierarchical composition of functions enables a layer-wise abstraction of features:
\begin{itemize}
	\item \textbf{Early Layers:} Learn to detect primitive features such as edges, corners, and color gradients.
	\item \textbf{Intermediate Layers:} Combine these primitives into more complex motifs, textures, and parts of objects.
	\item \textbf{Later Layers:} Assemble motifs into high-level, task
	relevant representations of entire objects or concepts.
\end{itemize}
Automated learning of a feature hierarchy is the primary advantage of deep learning models, especially in medical imaging where the relevant visual patterns may be too subtle for human-engineered features to capture effectively.

\begin{figure}[htbp]
	\centering
	\begin{tikzpicture}[
		scale=0.9, every node/.style={scale=0.9, align=center, font=\footnotesize},
		% Define styles for different layer types
		block_style/.style={
			rectangle, draw, rounded corners, minimum height=1cm, text width=2cm, thick, inner sep=4pt},
		arrow_style/.style={-Stealth, thick, shorten >=2pt, shorten <=2pt}
		]
		
		% Nodes arranged horizontally
		\node[block_style, fill=blue!10] (input) {Input Image};
		\node[block_style, fill=green!10, right=1cm of input] (layer1) {Simple Features};
		\node[block_style, fill=green!20, right=1cm of layer1] (layer2) {Combined Features};
		\node[block_style, fill=green!30, right=1cm of layer2] (layer3) {Abstract Features};
		\node[block_style, fill=red!10, right=1cm of layer3] (output) {Prediction};
		
		% Arrows connecting the layers
		\draw[arrow_style] (input.east) -- (layer1.west);
		\draw[arrow_style] (layer1.east) -- (layer2.west);
		\draw[arrow_style] (layer2.east) -- (layer3.west);
		\draw[arrow_style] (layer3.east) -- (output.west);
		
		% Label for direction of abstraction
		\node[above=0.6cm of layer2, font=\small\bfseries, text width=7cm] {Increasing Feature Abstraction $\longrightarrow$};
		
	\end{tikzpicture}
	\caption{Hierarchical feature learning in a deep network.}
	\label{fig:hierarchical_features_simple}
\end{figure} 

\subsection{Convolutional Neural Networks}
\label{ssec:dl_cnns_visual_specialists}
\acp{CNN} are a specialized class of neural networks designed for processing grid-like data, such as images \parencite{LeCun2015DeepLearning}. Their architecture is inspired by the human visual cortex and incorporates strong \textbf{inductive biases} that make them exceptionally effective for visual tasks:
\begin{itemize}
	\item \textbf{Local Receptive Fields:} Neurons are connected only to small, local regions of the input, exploiting the spatial locality of features in images.
	\item \textbf{Parameter Sharing:} The same set of weights (a filter or kernel) is applied across all spatial locations of the input. This drastically reduces the number of learnable parameters and builds in \textbf{translation equi-variance} the ability to detect a feature regardless of where it appears in the image.
	\item \textbf{Spatial Sub-sampling:} Pooling layers progressively reduce the spatial dimensions of the data, creating representations that are more robust to small shifts and distortions.
\end{itemize}
Notable \ac{CNN} architectures that build upon these principles include LeNet-5 \parencite{LeCun1998GradientBasedLearning}, AlexNet \parencite{Krizhevsky2012ImageNetClassification}, VGGNet \parencite{Simonyan2014VeryDeepConvolutional}, GoogLeNet \parencite{Szegedy2015GoingDeeperConvolutions}, ResNet \parencite{He2016DeepResidualLearning}, and U-Net \parencite{RonnebergerFB15}. More recently, Vision Transformers (\acp{ViT}) \parencite{Dosovitskiy2020ImageWorth16x16Words} have emerged as a powerful alternative, which uses attention mechanisms instead of convolutions.
\begin{figure}[htbp]
	\centering
	
	\begin{subfigure}[b]{0.48\textwidth}
		\centering
		\includegraphics[width=\linewidth, height=4cm, keepaspectratio]{Images/LeNet5.png}
		\caption{LeNet-5 \parencite{LeCun1998GradientBasedLearning}}
		\label{fig:lenet_arch}
	\end{subfigure}
	\hfill
	\begin{subfigure}[b]{0.48\textwidth} 
		\centering
		\includegraphics[width=\linewidth, height=4cm, keepaspectratio]{Images/AlexNet-Architecture.png}
		\caption{AlexNet \parencite{Krizhevsky2012ImageNetClassification}}
		\label{fig:alexnet_arch}
	\end{subfigure}
	
	\vspace{1cm}
	
	\begin{subfigure}[b]{0.48\textwidth}
		\centering
		\includegraphics[width=\linewidth, height=4cm, keepaspectratio]{Images/ResidualNet.jpeg}
		\caption{ResNet (Residual Block) \parencite{He2016DeepResidualLearning}}
		\label{fig:resnet_block_arch}
	\end{subfigure}
	\hfill
	\begin{subfigure}[b]{0.48\textwidth}
		\centering
		\includegraphics[width=\linewidth, height=4cm, keepaspectratio]{Images/The-3D-Unet-model.png}
		\caption{U-Net \parencite{RonnebergerFB15}}
		\label{fig:unet_arch}
	\end{subfigure}
	
	\caption{Some notable \ac{CNN} architectures.}
	\label{fig:cnn_architectures_gallery}
\end{figure}

\section{CNN Architecture Components}
\label{sec:dl_core_cnn_components}
A modern \ac{CNN} is constructed by composing a sequence of fundamental layers.
\begin{figure}[H]
	\centering
	\begin{tikzpicture}[
		scale=0.85, every node/.style={scale=0.85, font=\footnotesize, align=center},
		data_flow/.style={rectangle, draw, minimum height=1.2cm, minimum width=1.2cm, fill=gray!10, inner sep=2pt},
		op_block/.style={rectangle, draw, fill=blue!10, minimum height=0.8cm, text width=1.8cm, rounded corners=1pt, inner sep=3pt},
		fc_block/.style={rectangle, draw, fill=red!10, minimum height=1cm, minimum width=1cm, rounded corners=1pt, inner sep=3pt},
		arrow_style/.style={-Stealth, thick, shorten >=1pt, shorten <=1pt}
		]
		
		\node[data_flow, fill=blue!20, minimum width=1.5cm, label={[font=\tiny]below:Image}] (input_data) {};
		\draw (input_data.north east) -- ++(0.2,0.2) -- ++(0,-1.2*0.85) -- (input_data.south east); 
		\draw (input_data.north east) ++(0.2,0.2) -- ++(-1.5*0.85,0) -- (input_data.north west);
		
		\node[op_block, right=0.7cm of input_data, fill=green!20] (conv1_op) {Conv+ReLU};
		\node[data_flow, right=0.2cm of conv1_op, minimum width=1cm, minimum height=1cm, label={[font=\tiny]below:Maps 1}] (maps1_data) {};
		\draw (maps1_data.north east) -- ++(0.15,0.15) -- ++(0,-1.0*0.85) -- (maps1_data.south east);
		\draw (maps1_data.north east) ++(0.15,0.15) -- ++(-1.0*0.85,0) -- (maps1_data.north west);
		\node[op_block, right=0.2cm of maps1_data, fill=orange!30] (pool1_op) {Pool};
		
		\node[data_flow, right=0.2cm of pool1_op, minimum width=0.7cm, minimum height=0.7cm, label={[font=\tiny]below:Maps 2}] (maps2_data) {};
		\draw (maps2_data.north east) -- ++(0.1,0.1) -- ++(0,-0.7*0.85) -- (maps2_data.south east);
		\draw (maps2_data.north east) ++(0.1,0.1) -- ++(-0.7*0.85,0) -- (maps2_data.north west);
		
		\node[right=0.5cm of maps2_data] (dots) {\dots};
		
		\node[op_block, fill=purple!20, right=0.5cm of dots, text width=1cm] (flatten_op) {Flatten};
		\node[data_flow, right=0.2cm of flatten_op, minimum width=0.3cm, minimum height=1.5cm, label={[font=\tiny]below:Vector}] (vector_data) {}; 
		\node[fc_block, fill=red!20, right=0.5cm of vector_data] (fc_op) {FC};
		\node[data_flow, fill=yellow!30, right=0.2cm of fc_op, minimum width=0.5cm, label={[font=\tiny]below:Output}] (output_data) {};
		
		\draw[arrow_style] (input_data) -- (conv1_op);
		\draw[arrow_style] (conv1_op) -- (maps1_data);
		\draw[arrow_style] (maps1_data) -- (pool1_op);
		\draw[arrow_style] (pool1_op) -- (maps2_data);
		\draw[arrow_style] (maps2_data) -- (dots);
		\draw[arrow_style] (dots) -- (flatten_op);
		\draw[arrow_style] (flatten_op) -- (vector_data);
		\draw[arrow_style] (vector_data) -- (fc_op);
		\draw[arrow_style] (fc_op) -- (output_data);
		
	\end{tikzpicture}
	\caption{Basic building blocks of a \ac{CNN}}
	\label{fig:cnn_building_blocks}
\end{figure}

\subsection*{Convolutional Layers}
\label{ssec:dl_conv_layers_detail}
This is the primary building block of a \ac{CNN}, it works by sliding a small, learnable filter over an input volume. At each position, it computes a dot product between the filter's weights and the corresponding input patch. This process, illustrated in Figure~\ref{fig:convolution_OP}, generates a feature map that highlights specific patterns such as a vertical edge or a particular texture detected by the filter.
\begin{figure}[H]
	\centering
	\begin{tikzpicture}[
		scale=0.8, every node/.style={scale=0.8, font=\footnotesize, align=center},
		cell/.style={rectangle, draw, minimum size=0.7cm, inner sep=2pt},
		filter_cell/.style={rectangle, draw=blue, fill=blue!10, minimum size=0.7cm, inner sep=2pt},
		output_cell/.style={rectangle, draw=red, fill=red!10, minimum size=0.7cm, inner sep=2pt},
		op_symbol/.style={circle, draw, fill=yellow!30, minimum size=0.5cm, inner sep=1pt, font=\Large},
		arrow_style/.style={-Stealth, thick, shorten >=1pt, shorten <=1pt}
		]
		
		\matrix (input_patch) [matrix of nodes, nodes={cell}, column sep=-\pgflinewidth, row sep=-\pgflinewidth, anchor=center] at (0,0) {
			\pgfmathrandominteger{\val}{0}{9}\val & \pgfmathrandominteger{\val}{0}{9}\val & \pgfmathrandominteger{\val}{0}{9}\val \\
			\pgfmathrandominteger{\val}{0}{9}\val & \pgfmathrandominteger{\val}{0}{9}\val & \pgfmathrandominteger{\val}{0}{9}\val \\
			\pgfmathrandominteger{\val}{0}{9}\val & \pgfmathrandominteger{\val}{0}{9}\val & \pgfmathrandominteger{\val}{0}{9}\val \\
		};
		\node[above=0.2cm of input_patch] {Input Patch};
		
		\matrix (filter) [matrix of nodes, nodes={filter_cell}, column sep=-\pgflinewidth, row sep=-\pgflinewidth, anchor=center] at (3,1) { 
			\pgfmathrandominteger{\w}{0}{1}\w & \pgfmathrandominteger{\w}{0}{1}\w \\
			\pgfmathrandominteger{\w}{0}{1}\w & \pgfmathrandominteger{\w}{0}{1}\w \\
		};
		\node[above=0.2cm of filter] {Filter (Kernel)};
		
		\begin{scope}[on background layer]
			\node [fit=(input_patch-1-1) (input_patch-2-2), draw=blue, thick, fill=blue!5, inner sep=-1pt, dashed] {};
		\end{scope}
		
		\node[op_symbol] (operation) at (2.5, -1.5) {$\otimes$}; 
		\node[below=0.1cm of operation, text width=2.5cm, font=\tiny] {Element-wise Product + Sum + Bias};
		
		\node[output_cell] (output_value) at (5, -1.5) {Result};
		\node[above=0.2cm of output_value] {Output Element};
		
		\draw[arrow_style] (input_patch-2-2.south) .. controls +(0,-0.5) and +(-0.5,0.5) .. (operation.west);
		\draw[arrow_style] (filter.south) .. controls +(0,-0.5) and +(0.5,0.5) .. (operation.east);
		\draw[arrow_style] (operation.east) -- (output_value.west);
		
		\draw[->, blue!70!black, dashed, shorten >=2pt, shorten <=2pt] (filter.east) ++(0.2,0) -- ++(0.7,0) node[right, font=\tiny, black] {Filter slides};
		
	\end{tikzpicture}
	\caption{Principle of the convolution operation.}
	\label{fig:convolution_OP}
\end{figure}

\subsection*{Pooling Layers}
\label{ssec:dl_pooling_layers_detail}
Pooling layers perform non-linear down-sampling to reduce the spatial dimensions of feature maps. This decreases computational load, reduces the number of parameters, and provides a degree of translation invariance. The two most common forms are:
\begin{itemize}
	\item \textbf{Max Pooling:} Selects the maximum element from a local region $R$:\\
	\hspace*{\fill} $\text{MaxOut}(R) = \max_{(p,q) \in R} (a_{pq})$ \hspace*{\fill}
	
	\item \textbf{Average Pooling:} Computes the average of elements in region $R$:\\
	\hspace*{\fill} $\text{AvgOut}(R) = \frac{1}{|R|} \sum_{(p,q) \in R} a_{pq}$ \hspace*{\fill}
\end{itemize}
Pooling is usually applied with a fixed window size (e.g., 2x2) and stride (e.g., 2).
\begin{figure}[htbp]
	\centering
	\begin{tikzpicture}[
		scale=0.9, every node/.style={scale=0.9, font=\footnotesize, align=center},
		cell/.style={rectangle, draw, minimum size=0.7cm, rounded corners=2pt, inner sep=2pt},
		input_cell/.style={cell, fill=gray!10},
		max_val_cell/.style={cell, fill=yellow!60, font=\bfseries\large},
		output_cell/.style={cell, fill=red!20, font=\bfseries\Large},
		window_box/.style={draw=blue!70, very thick, dashed, rounded corners=2pt},
		arrow_style/.style={-Stealth, thick, blue!70},
		stride_arrow/.style={->, very thick, dashed, gray!70}
		]
		
		\node[above=0.2cm, font=\small] (input_label) at (1,2.5) {Input (4x4)};
		
		\matrix (input_map) [matrix of nodes, nodes={input_cell},
		column sep=-\pgflinewidth, row sep=-\pgflinewidth,
		anchor=center] at (1,1) {
			1 & 2 & 6 & 3 \\
			3 & 5 & 2 & 1 \\
			1 & 2 & 2 & 1 \\
			7 & 3 & 4 & 8 \\
		};
		
		\node at (input_map-2-2) [max_val_cell] {5};
		\node at (input_map-1-3) [max_val_cell] {6};
		\node at (input_map-4-1) [max_val_cell] {7};
		\node at (input_map-4-4) [max_val_cell] {8};
		
		\node[above=0.2cm, font=\small] (output_label) at (5,2.5) {Output (2x2)};
		
		\matrix (output_map) [matrix of nodes, nodes={output_cell},
		column sep=-\pgflinewidth, row sep=-\pgflinewidth,
		anchor=center] at (5,1) {
			5 & 6 \\
			7 & 8 \\
		};
		
		\node [window_box, fit=(input_map-1-1) (input_map-2-2)] (w1) {};
		\node [window_box, fit=(input_map-1-3) (input_map-2-4)] (w2) {};
		\node [window_box, fit=(input_map-3-1) (input_map-4-2)] (w3) {};
		\node [window_box, fit=(input_map-3-3) (input_map-4-4)] (w4) {};
		
		\draw[arrow_style] (w1) -- (output_map-1-1);
		\draw[arrow_style] (w2) -- (output_map-1-2);
		\draw[arrow_style] (w3) -- (output_map-2-1);
		\draw[arrow_style] (w4) -- (output_map-2-2);
		
		\draw[stride_arrow] ([xshift=0.35cm]input_map-1-1.east) -- ([xshift=0.35cm]input_map-1-3.west) node[midway, above, font=\tiny, blue!70] {};
		\draw[stride_arrow] ([yshift=-0.35cm]input_map-1-1.south) -- ([yshift=-0.35cm]input_map-3-1.north) node[midway, left, font=\tiny, blue!70, rotate=90] {};
		
	\end{tikzpicture}
	\caption{Max pooling with 2x2 windows and stride s=2 from input to output.}
	\label{fig:max_pooling}
\end{figure}

\subsection*{Activation Functions}
\label{ssec:dl_activation_functions_table_ultra_minimal_v2}
An activation function $\sigma(\cdot)$ is applied element-wise to the output of a linear operation (like convolution or a fully-connected layer \ref{ssec:dl_fc_output_layers}). It introduces non-linearity, which is critical for the network to learn complex mappings. The choice of activation function significantly impacts model performance and training dynamics. Common choices include the Sigmoid, Hyperbolic Tangent (Tanh), Rectified Linear Unit (\ac{ReLU}), and Leaky \ac{ReLU} functions, each with distinct properties.

\subsection*{Fully Connected Layers and Output Layers}
\label{ssec:dl_fc_output_layers}
After several stages of convolution and pooling, the high-level feature maps are flattened into a vector and passed to one or more \ac{FC} layers. In a \ac{FC} layer, every input neuron is connected to every output neuron. The final \ac{FC} layer produces the model's output logits, $z$. A final activation function transforms these logits into a desired output format, such as class probabilities using the Softmax function for multi-class classification:
\[ P(y=j|x) = \text{Softmax}(z_j) = \frac{e^{z_j}}{\sum_{k=1}^{C} e^{z_k}} \]
Here, $z_j$ is the logit for class $j$ out of $C$ total classes.

\subsection*{Supervised Learning in CNNs}
\label{sec:dl_supervised_learning_in_cnns}
\acp{CNN} are most commonly trained through supervised learning. Given a dataset of labeled examples $\mathcal{D} = \{(x_i, y_i)\}_{i=1}^N$, where $x_i$ is an input image and $y_i$ is its corresponding label, the goal is to learn the network parameters $\theta$ that best map inputs to outputs.\\

\noindent The training process is usually framed as an optimization problem where the objective is to find the set of parameters $\theta^*$ that minimizes the \textbf{empirical risk}, the average loss over the training dataset, often with an added regularization term $R(\theta)$:
\[ \theta^* = \arg\min_{\theta} \frac{1}{N} \sum_{i=1}^{N} \mathcal{L}(f(x_i; \theta), y_i) + \lambda R(\theta) \]
Here, $\hat{y}_i=f(x_i; \theta)$ is the model's prediction for input $x_i$, and the \textbf{loss function} $\mathcal{L}(\cdot, \cdot)$ quantifies the difference between the prediction $\hat{y}_i$ and the true label $y_i$. Common loss functions include:
\begin{itemize}
	\item \textbf{Cross-Entropy Loss:} For classification tasks. Binary Cross-Entropy (\ac{BCE}) for two-class problems:
	\[ \mathcal{L}_{BCE} = - [y \log(\hat{y}) + (1-y) \log(1-\hat{y})] \]
	\item \textbf{Mean Squared Error (MSE):} For regression tasks:
	\[ \mathcal{L}_{MSE} = \frac{1}{2}(\hat{y} - y)^2 \]
	\item \textbf{Dice Loss:} Commonly used for segmentation tasks to measure overlap between predicted ($P$) and ground truth ($G$) masks: 
	\[ \mathcal{L}_{Dice} = 1 - \frac{2 |P \cap G|}{|P| + |G|} \]
\end{itemize}

\section{Limitations of Training CNNs from Scratch}
\label{sec:dl_limitations_from_scratch}

Although Convolutional Neural Networks (\acp{CNN}) have shown remarkable success, training them from random initialization poses significant challenges, especially in medical imaging.

\subsection*{Data Scarcity}
\label{ssec:dl_data_dependency_scarcity_issue}
Deep models are \textbf{data-hungry}, requiring large, diverse annotated datasets. In medical domains, acquiring such data is often impractical due to costly, time-intensive expert labeling, privacy regulations (e.g., HIPAA, GDPR), and the rarity of certain conditions, which also induces severe class imbalance. Consequently, many medical datasets are orders of magnitude smaller than standard computer vision benchmarks, making training from scratch unreliable.

\subsection*{Overfitting and Limited Generalization}
\label{ssec:dl_overfitting_generalization_problem}
High-capacity models trained on small datasets are prone to \textbf{overfitting}, memorizing training samples rather than learning generalizable patterns. Techniques like regularization (L1/L2 \parencite{Hoerl1970Ridge,Tibshirani1996Lasso}), Dropout \parencite{Srivastava2014Dropout}, and Batch Normalization \parencite{Ioffe2015BatchNormalization} help but cannot fully compensate for scarce data.

\subsection*{Domain Shift}
\label{ssec:dl_domain_shift_sensitivity_issue}
Supervised learning assumes $P_{train}(X,Y) = P_{test}(X,Y)$. In clinical practice, this rarely holds (\textbf{domain shift} \parencite{Kouw2019ReviewDomainAdaptation}) due to differences in imaging devices, acquisition protocols, sample preparation, and patient populations. Domain sensitivity limits the deployability of models trained purely on local datasets.

\subsection*{Class Imbalance}
\label{ssec:dl_class_imbalance_challenge}
Medical datasets often exhibit extreme imbalance, where rare but clinically critical cases are vastly outnumbered. Standard losses bias training toward majority classes, degrading minority-class performance. Remedies include weighted losses (e.g., Focal Loss \parencite{Lin2017Focal}), resampling (e.g., SMOTE \parencite{Chawla2002SMOTE}), and robust metrics (e.g., \acp{AUC}, \ac{ROC} \parencite{Fawcett2006ROC}).

\noindent These limitations scarce data, overfitting, domain shift, and class imbalance make training CNNs from scratch impractical in real-world medical applications like malaria detection. This motivates advanced paradigms designed to enhance robustness and generalization, notably:

\begin{itemize}
	\item \textbf{Transfer Learning (\ac{TL}):} Leveraging knowledge from large-scale datasets to improve performance with limited labeled data (Chapter~\ref{sec:tl_leveraging_knowledge}).
	\item \textbf{Multitask Learning (\ac{MTL}):} Learning related tasks simultaneously to improve efficiency and generalization (Chapter~\ref{sec:mtl_simultaneous_tasks}).
\end{itemize}

\section{Transfer Learning Paradigm}
\label{sec:tl_leveraging_knowledge}
\acf{TL} is a paradigm where knowledge learned in a source domain $(D_S, T_S)$ is transferred to a target domain $(D_T, T_T)$ to improve performance on a target task. This approach is particularly useful in medical imaging where large annotated datasets are difficult to obtain due to privacy concerns and the high cost of expert labeling. General benchmarks like ImageNet \parencite{Deng2009ImageNetAL} provide millions of examples compared to medical data which is often limited. TL allows models to perform effectively in these low resource settings by reusing pre-learned features rather than training from scratch.

\subsection{Definition and Typology}
\label{ssec:tl_def_typo}
The most widely cited formal definition of \ac{TL} is provided by \textcite{pan2009survey}, who defined it in terms of a \textbf{domain} and a \textbf{task}:  
\begin{itemize}
	\item A \textbf{domain} $D$ consists of a feature space $\mathcal{X}$ and a marginal probability distribution $P(X)$, where $X = \{x_1, \dots, x_n\} \in \mathcal{X}$.
	\item A \textbf{task} $T$ is composed of a label space $\mathcal{Y}$ and a predictive function $f(\cdot)$ learned from the training data.
\end{itemize}

\noindent Given a source domain $D_S$ with task $T_S$, and a target domain $D_T$ with task $T_T$, the goal of \textbf{transfer learning} is to improve the learning of the target predictive function $f_T(\cdot)$ in $D_T$ using knowledge from $D_S$ and $T_S$, under the condition that $D_S \neq D_T$ or $T_S \neq T_T$.\\ 

\subsubsection*{Typology of Transfer Learning}
\ac{TL} can be categorized along several complementary dimensions. Zhuang et al. \parencite{zhuang2020comprehensive} proposed a more structured and modern classification of transfer learning methods. They categorized these methods along several complementary dimensions, reflecting the evolution of the field, especially with the rise of \ac{DL}:  

\begin{itemize}
	\item \textbf{By label availability between source and target:}  
	\begin{itemize}
		\item \textit{Inductive TL:} Target domain has labeled data. The source may or may not be labeled. 
		\item \textit{Transductive TL:} Target domain is unlabeled, but labeled source data is available. 
		\item \textit{Unsupervised TL:} Neither source nor target domains contain labels, and transfer relies on unsupervised representation learning.
	\end{itemize}
	
	\item \textbf{By feature space consistency:}  
	\begin{itemize}
		\item \textit{Homogeneous TL:} Source and target share the same feature space (e.g., RGB pixel space), though distributions differ. This is the most common case in medical imaging.  
		\item \textit{Heterogeneous TL:} Feature spaces differ (e.g., transferring between 2D fundus photography and 3D OCT scans). Specialized methods are required to align the modalities.  
	\end{itemize}
	
	\item \textbf{By what is transferred:}  
	\begin{itemize}
		\item \textit{Instance-based TL:} Source samples are reweighted or reused directly in the target training process.  
		\item \textit{Feature-based TL:} Shared or mapped feature representations are constructed to connect domains (e.g., domain-invariant embeddings).  
		\item \textit{Parameter-based TL:} Source trained parameters are reused as initialization or partially shared (e.g., fine-tuning CNN backbones).  
		\item \textit{Relational-based TL:} Structural knowledge such as inter-class relations or graph structures is transferred. 
	\end{itemize}
\end{itemize}

\subsection{Common Transfer Learning Techniques in Deep Learning}
\label{ssec:tl_common_techniques_revised} 
Several practical techniques facilitate parameter-based and feature-based \ac{TL} within \ac{DL}. 
\subsubsection*{Fine-tuning Pre-trained Models} 
\label{sssec:tl_fine_tuning_pre_trained_methods} Fine-tuning is the most common parameter-based \ac{TL} technique. It adapts a model pre-trained on a large source dataset like \textcite{Deng2009ImageNetAL} for a target task by continuing the training process on the target data. The primary strategies, shown in Figure~\ref{fig:tl_finetuning_strategies_legend_minimal} are:
\begin{itemize} 
	\item \textbf{Full Fine-tuning:} All layers of the pre-trained network are updated using a small learning rate. This is suitable when the target dataset is sufficiently large and similar to the source. 
	\item \textbf{Feature Extraction (Frozen Layers):} The convolutional base of the pre-trained model is frozen, acting as a fixed feature extractor. Only a new, task-specific classifier head is trained. This is ideal for very small target datasets to prevent overfitting. \item \textbf{Partial Fine-tuning:} A hybrid approach where early layers (capturing generic features like edges) are frozen, while later layers (capturing more abstract features) are fine-tuned along with the new head. This balances feature reuse and task-specific adaptation. 
\end{itemize}

\begin{figure}[htbp]
	\centering
	\resizebox{0.8\textwidth}{!}{
		
\begin{tikzpicture}[
	scale=0.8, every node/.style={scale=0.8, font=\footnotesize},
	layer_block/.style={rectangle, draw=black, minimum width=1.5cm, minimum height=0.6cm, text width=1.4cm, rounded corners=1pt, inner sep=2pt},
	frozen/.style={layer_block, fill=blue!20, pattern=north east lines, pattern color=blue!40},
	finetune/.style={layer_block, fill=orange!30},
	new_head/.style={layer_block, fill=red!30}
	]
	\node[frozen] (leg_frozen) at (0,0) {};
	\node[right=0.2cm of leg_frozen, anchor=west] {Frozen Layers (Pre-trained, Weights Fixed)};
	\node[finetune] (leg_finetune) at (0,-1) {};
	\node[right=0.2cm of leg_finetune, anchor=west] {Fine-tuned Layers (Pre-trained, Weights Updated)};
	\node[new_head] (leg_newhead) at (0,-2) {};
	\node[right=0.2cm of leg_newhead, anchor=west] {New Head (Trainable, From Scratch)};
\end{tikzpicture}

\vspace{0.5cm} % Space between legend and main diagram

% --- Main Diagram ---
\begin{tikzpicture}[
	scale=0.9, every node/.style={scale=0.9, font=\small, align=center},
	% Layer block styles from legend
	layer_block/.style={rectangle, draw=black, minimum width=1.2cm, minimum height=0.5cm, text width=1cm, rounded corners=1pt, inner sep=2pt, font=\tiny},
	frozen/.style={layer_block, fill=blue!20, pattern=north east lines, pattern color=blue!40},
	finetune/.style={layer_block, fill=orange!30},
	new_head/.style={layer_block, fill=red!30, minimum width=1cm, text width=0.8cm}, % Slightly smaller head
	% Other styles
	io_label/.style={font=\small\bfseries, text=black!70},
	strategy_label/.style={font=\bfseries\small, text width=3.5cm, align=center, minimum height=0.8cm},
	arrow/.style={-Stealth, thick, shorten >=1pt, shorten <=1pt, draw=gray!80}
	]
	
	% --- Positioning Coordinates ---
	\coordinate (input_pos) at (6,3.5); % Centered input
	\coordinate (s1_col) at (0,0);    % Column for Strategy 1
	\coordinate (s2_col) at (6,0);    % Column for Strategy 2
	\coordinate (s3_col) at (12,0);   % Column for Strategy 3
	\coordinate (output_pos) at (6,-4.5); % Centered output
	
	% --- Common Input ---
	\node[io_label] (input_node) at (input_pos) {Input Image};
	
	% === Strategy 1: Feature Extraction ===
	\node[strategy_label] (s1_title) at ($(s1_col) + (0,1.5)$) {Feature Extraction};
	\node[frozen] (s1_l1) at (s1_col) {L1};
	\node[frozen, below=0.1cm of s1_l1] (s1_l2) {L2};
	\node[frozen, below=0.1cm of s1_l2] (s1_l3) {\vdots};
	\node[frozen, below=0.1cm of s1_l3] (s1_ln) {LN};
	\node[new_head, below=0.2cm of s1_ln] (s1_head) {Head};
	\draw[arrow] (s1_l1) -- (s1_l2); \draw[arrow] (s1_l2) -- (s1_l3); \draw[arrow] (s1_l3) -- (s1_ln); \draw[arrow] (s1_ln) -- (s1_head);
	\draw[arrow] (input_node) -- (s1_l1.north);
	\draw[arrow] (s1_head.south) -- (output_pos);
	
	% === Strategy 2: Partial Fine-tuning ===
	\node[strategy_label] (s2_title) at ($(s2_col) + (0,1.5)$) {Partial Fine-tuning};
	\node[frozen]   (s2_l1) at (s2_col) {L1};
	\node[frozen,   below=0.1cm of s2_l1] (s2_l2) {L2};
	\node[finetune, below=0.1cm of s2_l2] (s2_l3) {\vdots}; % Transition point
	\node[finetune, below=0.1cm of s2_l3] (s2_ln) {LN};
	\node[new_head, below=0.2cm of s2_ln] (s2_head) {Head};
	\draw[arrow] (s2_l1) -- (s2_l2); \draw[arrow] (s2_l2) -- (s2_l3); \draw[arrow] (s2_l3) -- (s2_ln); \draw[arrow] (s2_ln) -- (s2_head);
	\draw[arrow] (input_node) -- (s2_l1.north);
	\draw[arrow] (s2_head.south) -- (output_pos);
	
	% === Strategy 3: Full Fine-tuning ===
	\node[strategy_label] (s3_title) at ($(s3_col) + (0,1.5)$) {Full Fine-tuning};
	\node[finetune] (s3_l1) at (s3_col) {L1};
	\node[finetune, below=0.1cm of s3_l1] (s3_l2) {L2};
	\node[finetune, below=0.1cm of s3_l2] (s3_l3) {\vdots};
	\node[finetune, below=0.1cm of s3_l3] (s3_ln) {LN};
	\node[new_head, below=0.2cm of s3_ln] (s3_head) {Head};
	\draw[arrow] (s3_l1) -- (s3_l2); \draw[arrow] (s3_l2) -- (s3_l3); \draw[arrow] (s3_l3) -- (s3_ln); \draw[arrow] (s3_ln) -- (s3_head);
	\draw[arrow] (input_node) -- (s3_l1.north);
	\draw[arrow] (s3_head.south) -- (output_pos);
	
	% --- Common Output ---
	\node[io_label] (output_node) at (output_pos) {Target Prediction};
	
\end{tikzpicture}
	

	}
	\caption{Conceptual overview of three common fine-tuning strategies.}
	\label{fig:tl_finetuning_strategies_legend_minimal}
\end{figure}

\subsubsection*{Learning Rate Management}
\label{sssec:tl_learning_rate_management}
Effective fine-tuning requires careful learning rate management to avoid catastrophic forgetting of valuable pre-trained features. A common practice is using \textbf{differential learning rates}, where different parts of the network have different update speeds. Let $\theta = \{\theta_1, \dots, \theta_L\}$ be the parameters of the model's layers. The update rule for layer $i$ becomes:
\[ \theta_i \leftarrow \theta_i - \eta_i \nabla_{\theta_i} \mathcal{L} \]
where $\eta_i$ is the learning rate for the $i$-th layer. Typically, earlier layers are assigned a smaller learning rate than later layers ($\eta_i < \eta_j$ for $i < j$), allowing the model to preserve general features while adapting task-specific ones more aggressively.

\subsubsection*{\acf{PEFT}}
\label{sssec:tl_adapter_modules_efficiency}
\acf{PEFT} adapts large pretrained models by training only a small set of extra parameters while freezing the majority, enabling efficient transfer with minimal storage and compute \parencite{Houlsby2019ParameterEfficientTransferLearning}. A key example is \textbf{adapter modules}, precisely \textbf{Low-Rank Adaptation (LoRA)} \parencite{Hu2021LoRA} which inserts trainable low-rank matrices into the weight update path. For an input $\mathbf{z}$ (Fig.~\ref{fig:lora_module}):
\begin{equation}
	\mathbf{z}' = \mathbf{z} + \frac{\alpha}{r} W_B W_A \mathbf{z}.
	\myequations{LoRa Update equation.}
	\label{eq:lora_update}
\end{equation}
This keeps parameter growth minimal while maintaining strong performance.

\subsection{General Benefits and Limitations of Transfer Learning}
\label{sec:tl_general_benefits_limitations}
\ac{TL} provides clear advantages for model development, particularly in data-scarce domains, but it also comes with important caveats that must be considered.

\subsubsection*{Key Benefits}
\begin{itemize}
	\item \textbf{Improved Performance with Limited Data:} Using pre-trained models allows achieving high performance on a target task with substantially fewer labeled samples than training from scratch, as the model already captures generalizable feature representations.
	\item \textbf{Faster Convergence:} Fine-tuning starts from a well-initialized weight set, reducing the number of epochs needed to reach optimal performance and speeding up training.
	\item \textbf{Better Generalization:} Features learned from large, diverse source datasets (e.g., ImageNet) often provide robust representations that act as a regularizer, lowering the risk of overfitting on small target datasets.
	\item \textbf{Access to Advanced Architectures:} Transfer learning enables the use of state-of-the-art architectures without the computational cost of training them from scratch.
\end{itemize}

\subsubsection*{Limitations and Challenges}
\begin{itemize}
	\item \textbf{Negative Transfer:} If the source and target domains/tasks are too dissimilar, transferred knowledge can bias the model and degrade performance compared to training from scratch \parencite{wang2019characterizing}.
	\item \textbf{Domain Shift:} Differences in data distribution (e.g., natural images vs. medical microscopy) may limit transfer effectiveness, as source features might be irrelevant or misleading for the target task \parencite{Kouw2019ReviewDomainAdaptation}.
	\item \textbf{Inherited Source Bias:} Any biases present in the source dataset such as demographic or equipment-related can propagate to the target model, reducing robustness to out-of-distribution samples.
	\item \textbf{Fine-tuning Complexity:} Selecting which layers to freeze, choosing learning rates, or deciding on partial vs. full fine-tuning often requires extensive experimentation \parencite{raffel2020exploring}.
\end{itemize}

\noindent Overall, \ac{TL} is a powerful tool for accelerating learning and improving generalization, but it must be applied carefully, with attention to domain alignment and task relevance. 

\section{Multitask Learning Paradigm}
\label{sec:mtl_simultaneous_tasks}
\ac{MTL} is a paradigm in which a single model is trained to perform multiple related tasks simultaneously. The central idea, introduced in early work by \textcite{Caruana1997MTL}, is that learning tasks in parallel with a shared representation allows the model to exploit commonalities across tasks, improving generalization. This joint learning acts as a form of inductive transfer, providing regularization that reduces overfitting, particularly in data scarce settings \parencite{Ruder2017OverviewMTL}.

\subsection{Definition and Formulation}
\label{sec:mtl_definition_math}
In an \ac{MTL} setting, we consider $K$ related tasks $\{T_k\}_{k=1}^K$, each with dataset 
$\mathcal{D}_k = \{(x_i^{(k)}, y_i^{(k)})\}_{i=1}^{N_k}$ and loss $\mathcal{L}_k$. 
A model is designed with both shared parameters $\theta_{\text{sh}}$ and task-specific parameters 
$\{\theta_k\}_{k=1}^K$. The overall objective is a weighted sum of task losses:

\begin{equation}
	\mathcal{L}_{\text{MTL}}(\theta_{\text{sh}}, \{\theta_k\}_{k=1}^K) 
	= \sum_{k=1}^{K} \lambda_k \, \mathcal{L}_k \big(f(x; \theta_{\text{sh}}, \theta_k), y_i^{(k)}\big)
	\myequations{Multitask Learning optimization formulation.}
	\label{eq:mtl_loss}
\end{equation}
where $\lambda_k$ balances the contribution of each task. The challenge is to optimize both the parameters and the task weights such that knowledge sharing through $\theta_{sh}$ promotes positive transfer rather than interference.

\subsection{MTL Architectural Approaches}
\label{ssec:mtl_architectural_approaches}

Architectural design is central to how \ac{MTL} takes advantage of shared representations. The two most widely adopted approaches are \textbf{hard parameter sharing} and \textbf{soft parameter sharing}:  

\begin{itemize}
	\item \textbf{Hard Parameter Sharing:} A shared encoder (backbone) captures common features, followed by task-specific heads. This forces the backbone to learn representations useful across all tasks and provides strong regularization.
	\item \textbf{Soft Parameter Sharing:} Each task has its own parameters, but similarity between task models is encouraged through explicit regularization (e.g., penalizing large distances between parameter sets).
\end{itemize}

\begin{figure}[htbp]
	\centering
	\begin{subfigure}[t]{0.47\textwidth}
		\centering
		\includegraphics[width=\textwidth]{Images/HardParmSharing.png}
		\caption{Hard parameter sharing in MTL.}
		\label{fig:mtl_hard_sharing}
	\end{subfigure}
	\hfill
	\begin{subfigure}[t]{0.47\textwidth}
		\centering
		\includegraphics[width=\textwidth]{Images/SoftSharing.png}
		\caption{Soft parameter sharing in MTL.}
		\label{fig:mtl_soft_sharing}
	\end{subfigure}
	\caption{Main architectural approaches for MTL.}
	\label{fig:mtl_sharing_comparison}
\end{figure}

More advanced architectures mentioned by \textcite{Ruder2017OverviewMTL}, such as Deep Relationship Networks, Cross-stitch Networks, and Sluice Networks, offer more nuanced mechanisms for controlling knowledge sharing between tasks.

\subsection{Optimization Challenges}
\label{sec:mtl_optimization_challenges}
Unlike \acf{STL}, \ac{MTL} introduces unique training difficulties:

\begin{itemize}
	\item \textbf{Task Balancing:} Uniform task weights ($\lambda_k=1$) often lead to dominance by tasks with larger losses. Dynamic schemes, such as uncertainty weighting \parencite{Kendall2018MultiTaskLearningUncertainty} or GradNorm \parencite{Chen2018GradNorm}, adaptively rescale contributions during training.
	\item \textbf{Negative Transfer and Gradient Conflicts:} Sharing across weakly related tasks may harm performance \parencite{wang2019characterizing}. Conflicting gradients from different tasks can pull shared parameters in opposite directions. Recent methods mitigate this by modifying or projecting gradients to reduce destructive interference \parencite{Yu2020GradientSurgery}.
\end{itemize}

\subsection{General Benefits and Limitations of Multitask Learning}
\label{sec:mtl_general_benefits_limitations}

\subsubsection*{Key Benefits}
\begin{itemize}
	\item \textbf{Implicit Data Augmentation:} Joint training exposes the shared backbone to more varied data distributions, improving representation robustness.
	\item \textbf{Regularization:} Shared parameters constrain the model, reducing overfitting and improving generalization.
	\item \textbf{Feature Relevance:} Auxiliary tasks can guide attention toward informative features that a primary task might underutilize.
	\item \textbf{Efficiency:} A single \ac{MTL} model supports multiple tasks in one forward pass, saving inference cost relative to training separate models.
\end{itemize}

\subsubsection*{Limitations and Challenges}
\begin{itemize}
	\item \textbf{Negative Transfer:} The main risk, arising when tasks are insufficiently related, degrading performance relative to single-task baselines.
	\item \textbf{Task Balancing Complexity:} Optimal weighting of $\lambda_k$ is non-trivial, and poor choices can destabilize training.
	\item \textbf{Architecture Design:} Determining the right degree of sharing between tasks requires careful experimentation.
	\item \textbf{Dependence on Task Relatedness:} Benefits assume relatedness between tasks. If this assumption fails, \ac{MTL} may be counterproductive.
\end{itemize}

\section{Prior Work in Malaria Detection}
\label{sec:prior_work_malaria}
This section reviews prior research in malaria blood smear analysis with a particular emphasis on advanced AI paradigms. We first discuss how transfer learning and related methods have been applied to malaria detection across different imaging modalities. We then examine the emerging but underexplored potential of \acf{MTL} and \acf{MTTL}, which defines our own research focus.

\subsection{Advanced AI Paradigms for Malaria Detection}
\label{ssec:advanced_ai_malaria}
Transfer learning (\ac{TL}) has been the most widely adopted paradigm in malaria detection, largely due to its effectiveness in overcoming limited annotated datasets. Pre-trained convolutional neural networks (CNNs) such as VGG \parencite{Simonyan2014VeryDeepConvolutional}, ResNet \parencite{he2016deep}, DenseNet \parencite{Huang2017DenseNet}, Inception \parencite{Szegedy2015GoogLeNet}, and EfficientNet \parencite{Tan2019EfficientNet} appear frequently in the literature, with ImageNet as the primary source domain \parencite{Deng2009ImageNetAL}. Fine-tuning these models on malaria datasets has consistently yielded high accuracy in the narrow task of binary classification between parasitized and uninfected cells. Table~\ref{tab:tl_studies_summary} summarizes representative works, spanning thin blood smears, thick smears, and smartphone-based microscopy.

\begin{center} 
	\footnotesize 
	\begin{longtable}{@{} 
			>{\raggedright\arraybackslash}p{0.22\textwidth}  % Study (Ref.)
			>{\centering\arraybackslash}p{0.05\textwidth}     % Year 
			>{\raggedright\arraybackslash}p{0.28\textwidth}  % Task
			>{\raggedright\arraybackslash}p{0.20\textwidth}  % Method
			>{\raggedright\arraybackslash}p{0.20\textwidth}@{}}
		\caption{Selected TL Applications in Medical Imaging and Malaria Detection}\label{tab:tl_studies_summary}\\
		\toprule
		\textbf{Study} & \textbf{Year} & \textbf{Task (Dataset/Type)} & \textbf{Method/Model} & \textbf{Key Result/Note} \\ \midrule
		\endfirsthead
		
		\multicolumn{5}{c}{{\bfseries\tablename\ \thetable{}} -- continued from previous page} \\
		\toprule
		\textbf{Study} & \textbf{Year} & \textbf{Task (Dataset/Type)} & \textbf{Method/Model} & \textbf{Key Result/Note} \\ \midrule
		\endhead
		
		\multicolumn{5}{r}{{\textit{Continued on next page}}} \\
		\endfoot
		
		\bottomrule
		\endlastfoot
		
		% --- 2016 ---
		%\multicolumn{5}{l}{\textbf{2016}} \\
		%\addlinespace[0.2em]
		
		%\textcite{Gulshan2016DevelopmentValidationDLAlgorithm} & 2016 & Diabetic Retinopathy Grading (Fundus Images) & InceptionV3 TL & Ophthalmologist-level Acc. \\
		%\addlinespace[0.3em]
		
		% --- 2017 ---
		%\multicolumn{5}{l}{\textbf{2017}} \\
		%\addlinespace[0.2em]
		
		%\textcite{Rajpurkar2017CheXNet} & 2017 & Pneumonia Det. (ChestX-ray14) & DenseNet-121 TL & Radiologist-level Perf. \\
		%\addlinespace[0.3em]
		%\textcite{Esteva2017DermatologistlevelClassificationSkin} & 2017 & Skin Cancer Classif. (Dermoscopy) & InceptionV3 TL & Dermatologist-level Acc. \\
		%\addlinespace[0.3em]
		%\textcite{Bejnordi2017DiagnosticAssessmentDeep} & 2017 & Breast Cancer Metastasis Det. (CAMELYON16 \ac{WSI}) & GoogLeNet TL & High Acc. lymph node \\
		%\addlinespace[0.3em]
		
		% --- 2018 ---
		\multicolumn{5}{l}{\textbf{2018}} \\
		\addlinespace[0.2em]
		
		\rowcolor{gray!15}
		\textcite{Rajaraman2018NIHMalariaDataset} & 2018 & Malaria: Binary Cell Classif. (NIH, Thin Giemsa) & VGG, ResNet, etc. TL & ResNet-50: 95.9\% Acc. \\
		\addlinespace[0.3em]
		
		% --- 2021 ---
		\multicolumn{5}{l}{\textbf{2021}} \\
		\addlinespace[0.2em]
		
		\rowcolor{gray!15}
		\textcite{Abdurahman2021ModifiedYOLO} & 2021 & Malaria: Parasite Det. (Makerere U., Thick N/S) & Modified YOLOv3/v4 & YOLO for thick smears \\
		\addlinespace[0.3em]
		\rowcolor{gray!15}
		\textcite{DeepBasedDeconv} & 2021 & Malaria: RBC Det. \& Extraction (NLM Thin Smears) & Modified CFPNet-M (Distance Transform Regression) & 92.2\% Cell Count Acc. \\
		\addlinespace[0.3em]
		
		% --- 2022 ---
		\multicolumn{5}{l}{\textbf{2022}} \\
		\addlinespace[0.2em]
		
		\rowcolor{gray!15}
		\textcite{Banerjee2022Falcon} & 2022 & Malaria: Binary Cell Classif. (NIH, Thin N/S) & Falcon DCNN, TL & Custom DCNN+TL; 95.2\% Acc. \\
		\addlinespace[0.3em]
		\rowcolor{gray!15}
		\textcite{LiMa2022ResidualAttentionSVM} & 2022 & Malaria: Binary Cell Classif. (NIH, Thin N/S) & Residual Attn. CNN + SVM & Hybrid CNN-SVM; 99.7\% Acc. \\
		\addlinespace[0.3em]
		\rowcolor{gray!15}
		\textcite{AlonsoRamirez2022CNNRecurrent} & 2022 & Malaria: Binary Cell Classif. (NLM NIH, Thin N/S) & CNN + LSTM/BiLSTM & CNN-RNN hybrid; 99.89\% Acc. \\
		\addlinespace[0.3em]
		\rowcolor{gray!15}
		\textcite{Jameela2022DeepLearning} & 2022 & Malaria: Cell Classif. (NIH, Thin Smear) & ResNet50, ResNet34, VGG-16, VGG-19 TL/FT & Best VGG-19 Acc. \\
		\addlinespace[0.3em]
		
		% --- 2023 ---
		\multicolumn{5}{l}{\textbf{2023}} \\
		\addlinespace[0.2em]
		
		\rowcolor{gray!15}
		\textcite{sukumarran_automated_2023} & 2023 & Malaria Det. (MP-IDB/ Giemsa Thin) & YOLOv4, Faster R-CNN, SSD300 & YOLOv4: Prec. 83\%, Rec. 95\%, F1 89\%, mAP 93.87\% \\
		\addlinespace[0.3em]
		\rowcolor{gray!15}
		\textcite{Khan2023CNNBTSystems} & 2023 & Malaria: Binary Cell Classif. (NLM NIH, Thin N/S) & Inception-ResNet TL & 95\% Acc. \\
		\addlinespace[0.3em]
		\rowcolor{gray!15}
		\textcite{Goni2023DELMMalaria} & 2023 & Malaria: Binary Cell Classif. (Kaggle NIH derived, Thin N/S) & Light CNN + DELM & CNN feat. + DELM; 99.66\% Acc. \\
		\addlinespace[0.3em]
		\rowcolor{gray!15}
		\textcite{Silka2023MalariaDetection} & 2023 & Malaria: Cell Det. (BBBC041, Thin Giemsa) & Custom Semantic Seg. CNN & 99.68\% Acc. (Det. via Seg.); 97.1\% pixel Acc. \\
		\addlinespace[0.3em]
		
		% --- 2024 ---
		\multicolumn{5}{l}{\textbf{2024}} \\
		\addlinespace[0.2em]
		
		\rowcolor{gray!15}
		\textcite{HoyosHoyos2024YOLOv8Malaria} & 2024 & Malaria: Parasite/Leukocyte Det. (NIH, Thin N/S) & YOLOv8 & 95\% Acc. (Parasite) \\
		\addlinespace[0.3em]
		\rowcolor{gray!15}
		\textcite{Ramos2024TransferLearning} & 2024 & Malaria: Plasmodium vivax ROI Classif. (BBBC+FIOCRUZ, Thin Giemsa) & DenseNet201, MobileNetV2, VGG19, etc. TL & DenseNet201: 99.41\% AUC \\
		\addlinespace[0.3em]
		\rowcolor{gray!15}
		\textcite{Ohdar2024MultiSpecies} & 2024 & Malaria: Multi-Species Classif. (MP-IDB, Thin N/S) & Xception, EfficientNet, MobileNet, DenseNet121 Ensemble TL & 98.9\% Acc. (4 species) \\
		\addlinespace[0.3em]
		\rowcolor{gray!15}
		\textcite{Boit2024MalariaDetection} & 2024 & Malaria: RBC Classif. (NIH, Thin Giemsa) & EDRI (Hybrid DL Model) & 97.68\% Acc. \\
		\addlinespace[0.3em]
		
		% --- 2025 ---
		\multicolumn{5}{l}{\textbf{2025}} \\
		\addlinespace[0.2em]
		
		\rowcolor{gray!15}
		\textcite{RamosBriceno2025MalariaSpecies} & 2025 & Malaria: Species Classif. (NLM, Thick Giemsa) & Custom CNN (7-ch. input, preprocessing) & 99.51\% Acc. (P. falciparum, P. vivax, uninfected WBC) \\
		\addlinespace[0.3em]
		\rowcolor{gray!15}
		\textcite{SoraCardenas2025MalariaThickSmear} & 2025 & Malaria: Parasite/Leukocyte Det. \& Classif. (Thick Romanowsky) & SVM (Quality), OTSU (Leukocyte), Custom CNN (Parasite) & Parasite Det. F1: 82.10\%; Stage Classif. F1: 83-88\% \\
		\addlinespace[0.3em]
		\rowcolor{gray!15}
		\textcite{Mmileng2025ConvNeXtMalaria} & 2025 & Malaria: Parasite Classif. (NLM Thin Smear) & ConvNeXt Tiny (V1/V2 mod.) TL/DA & ConvNeXt V2 Tiny: 98.1\% Acc. \\
		\addlinespace[0.3em]
		\rowcolor{gray!15}
		\textcite{Nasra2025AutomatedMalaria} & 2025 & Malaria: Parasite Det. (Lacuna, Thick/Thin) & Optimized CNN Architecture & Approx. 98\% Acc.\\
	\end{longtable}
\end{center}

%\begin{center}
%	\parbox{0.9\linewidth}{
%		\scriptsize
%		\textbf{Note:} Gray background indicates malaria studies. Key datasets: NIH \parencite{Rajaraman2018NIHMalariaDataset}, Lacuna \parencite{LacunaMalariaDatasets} MP-IDB \cite{MP-IDB}.
%		\vspace{0.3em}
%		
%		\textbf{Abbreviations:} Acc.: Accuracy; Attn.: Attention; Classif.: Classification; CNN: Convolutional Neural Network; Det.: Detection; F1: F1-Score; FT: Fine-Tuning; mAP: Mean Average Precision; N/S: Not Specified; NLM: National Library of Medicine; Perf.: Performance; Prec.: Precision; RBC: Red Blood Cell; Rec.: Recall; ROI: Region of Interest; Seg.: Segmentation; TL: Transfer Learning; WBC: White Blood Cell; WSI: Whole Slide Imaging.
%	}
%\end{center}
%\vspace{\baselineskip}

\noindent
Beyond fine-tuning, several works explored ensemble strategies \parencite{Nayak2022EnsembleAI, Bibin2017MalariaDBN}, hybrid pipelines mixing CNN features with classifiers such as SVMs \parencite{LiMa2022ResidualAttentionSVM}, and object detection based on YOLO variants \parencite{Abdurahman2021ModifiedYOLO, HoyosHoyos2024YOLOv8Malaria}. Data augmentation is also widely used because annotated samples are limited. Although these methods mark progress, important limitations remain. Model outputs are often partial and do not cover all steps needed for clinical diagnosis, and performance tends to drop when exposed to variations in clinics, staining quality, or imaging devices. Reproducibility is also weakened by inconsistent reporting of smear types, dataset curation, or preprocessing choices.\\

\noindent
Transfer learning has been helpful for automated malaria classification, yet most studies focus on the binary separation between infected and uninfected cells. A complete diagnosis requires more information, including species, life stage, and parasitemia. Since standard TL approaches usually target a single prediction objective, there is space for techniques that better reflect the multi-step nature of real diagnosis.

\subsection{Multitask and Multitask Transfer Learning Relevance}
\label{ssec:mtl_mttl_malaria}

\noindent
Multitask learning (\ac{MTL}) and it's extension multitask transfer learning (MTTL) offer a way to learn several related outputs together. Tasks such as species identification, life-stage prediction, and parasitemia estimation share visual cues, so a shared backbone with transferred features can capture these relations more effectively than training independent models. Despite this fit, no systematic use of MTL or MTTL has been reported for malaria imaging. In other medical imaging areas, \ac{MTL} and \ac{MTTL} have shown clear value. \textcite{Yadav2024_MT_TL_ActiveMedSeg} used multitask transfer learning within an active learning setup for segmentation and reported better annotation efficiency. \textcite{Huang2022MTLABS3Net} introduced a multitask system combining organ segmentation with anatomical prior prediction and obtained strong semi-supervised performance. \textcite{Kim2023CrossTaskAttention} presented a cross-task attention mechanism to improve multi-task learning across diverse medical applications. \textcite{Chandani2023Investigating} and \textcite{Dheer2023Exploiting} studied MTTL approaches for segmentation, showing that shared representations improve accuracy. \textcite{Schoenpflug2023_MTL_CRC} applied multitask learning to colorectal cancer histology slides to segment tissues and detect tumors, while \textcite{Li2023_MTL_NSCLC} used MTL to classify histologic subtypes and tumor grade in non-small cell lung cancer CT scans. In robotic surgery, \textcite{Islam2021StMTLSpatioTemporal} developed a spatio-temporal multitask model predicting instrument paths and surgeon scanpaths. Applications outside medicine, including NLP and multi-view autonomous driving \parencite{Zhang2019_MultiviewTL_MTL, raffel2020exploring}, also report gains in efficiency and robustness.\\

\noindent
Together, these studies show that multitask and multitask transfer learning can capture shared structure across tasks, improve generalization, and avoid training separate models. This opportunity remains mostly unexplored in malaria diagnosis.

\subsection{Observed Gaps in the Literature}
\label{ssec:literature_gaps}

\noindent
Even with recent progress, key gaps still limit clinical use when malaria diagnosis is viewed as a full multi-step procedure. Main issues include:

\begin{itemize}
	\item \textbf{Incomplete outputs:} Most studies classify cells as either infected or uninfected, which does not provide species, life stage, or parasitemia needed for treatment decisions.
	\item \textbf{Lack of multitask approaches:} No work has applied MTL or MTTL to malaria imaging even though this mirrors the workflow of microscopic diagnosis.
	\item \textbf{Theoretical limitations:} Existing research does not offer a formal multitask transfer framework adapted to malaria or similar diagnostic processes.
	\item \textbf{Reporting inconsistencies:} Details like smear type, staining protocol, and preprocessing are often missing, which complicates reproducibility.
	\item \textbf{Dataset constraints:} Public datasets are small and homogeneous, reducing the ability of models to generalize across clinics or devices.\\
\end{itemize}

\noindent Prior work confirms the strengths of transfer learning but remains narrow in scope. The absence of multitask methods, combined with limited datasets and incomplete diagnostic outputs, highlights the need for more comprehensive approaches. This motivates the introduction of a multitask transfer learning framework for malaria diagnosis.
\cleardoublepage

%MTTL FRAMEWORK
\chapter{Multi-Task Transfer Learning Framework}
\label{chap:framework}

This chapter presents the theoretical contribution of this work, a mathematical framework for \acf{MTTL} built on foundational concepts from \textcite{pan2009survey}, \textcite{Caruana1997MTL}, and \textcite{Ruder2017OverviewMTL}, we first establish the necessary mathematical preliminaries. From there, we formulate Single-Task, Transfer, and Multi-Task Learning as distinct optimization problems. This leads to our proposed \ac{MTTL} formulation, which integrates pre-trained knowledge and parameter-efficient fine-tuning into a joint adaptation process across multiple related tasks.

\section{Mathematical Preliminaries and Notation}
\noindent A \textbf{domain} $\mathcal{D}$ is a tuple defining the feature space and its underlying data distribution:
\begin{equation}
	\mathcal{D} = \{\mathcal{X}, P(\mathbf{x})\}
	\myequations{Domain Definition.}
\end{equation}
where $\mathcal{X}$ is the input feature space and $P(\mathbf{x})$ is the marginal probability distribution over $\mathcal{X}$.

\noindent A \textbf{task} $\mathcal{T}$ defines a predictive objective within a domain:
\begin{equation}
	\mathcal{T} = \{\mathcal{Y}, f(\cdot)\}, \quad \text{where } f: \mathcal{X} \to \mathcal{Y}
	\myequations{Task Definition.}
\end{equation}
where $\mathcal{Y}$ is the label space and $f(\cdot)$ is the true, unknown target function.

\noindent A \textbf{model} is a parametric function $h(\mathbf{x}; \theta)$ with parameters $\theta \in \Theta$, designed to approximate $f(\cdot)$. For modern deep architectures, the model is a composition of a feature-extracting \textbf{backbone} $\phi(\cdot; \theta_\phi)$ and a task-specific \textbf{head} $g(\cdot; \theta_g)$:
\begin{equation}
	h(\mathbf{x}; \theta) = g(\phi(\mathbf{x}; \theta_\phi); \theta_g), \quad \text{where } \theta = \{\theta_\phi, \theta_g\}
	\myequations{Model Decomposition.}
\end{equation}

\noindent Given a dataset $D = \{(\mathbf{x}_i, y_i)\}_{i=1}^N$ drawn from a joint distribution $P(\mathbf{x}, y)$, learning in a supervised setting is framed as finding the optimal parameters $\theta^*$ that minimize the \textbf{empirical risk} over a loss function $\mathcal{L}(\cdot, \cdot)$:
\begin{equation}
	\theta^{*} = \arg\min_{\theta \in \Theta} \frac{1}{N} \sum_{i=1}^N \mathcal{L}(h(\mathbf{x}_i; \theta), y_i)
	\myequations{Empirical Risk Minimization.}
	\label{eq:erm}
\end{equation}

\noindent
Using the established notation, we now formally define the optimization objectives for STL, TL, MTL, and MTTL.\\

\noindent In the standard \acf{STL} paradigm, a model with randomly initialized parameters $\theta = \{\theta_\phi, \theta_g\}$ is trained from scratch to learn a single task $\mathcal{T}$ using its corresponding dataset $D$. The objective is a direct application of Equation~\ref{eq:erm}:
\begin{equation}
	(\theta_\phi^*, \theta_g^*) = \arg\min_{\theta_\phi, \theta_g} \frac{1}{N} \sum_{i=1}^N \mathcal{L}(g(\phi(\mathbf{x}_i; \theta_\phi); \theta_g), y_i)
	\myequations{STL Optimization Objective.}
\end{equation}

\noindent \acf{TL} uses knowledge acquired from a source task $\mathcal{T}_S = \{\mathcal{Y}_S, f_S(\cdot)\}$ on a source domain $\mathcal{D}_S = \{\mathcal{X}_S, P_S(\mathbf{x})\}$ to a target task $\mathcal{T}_T = \{\mathcal{Y}_T, f_T(\cdot)\}$ on a target domain $\mathcal{D}_T = \{\mathcal{X}_T, P_T(\mathbf{x})\}$. A pre-trained backbone $\phi(\cdot; \theta_\phi^*)$ from $\mathcal{T}_S$ is used. A new head $g_T(\cdot; \theta_{g_T})$ is trained for $\mathcal{T}_T$.
\begin{equation}
	\theta_T^* = \arg\min_{\theta_T} \frac{1}{N_T} \sum_{i=1}^{N_T} \mathcal{L}_T(g_T(\phi(\mathbf{x}_i; \theta_\phi^*); \theta_{g_T}), y_i)
	\myequations{Transfer Learning Optimization Objective.}
\end{equation}
where $\theta_T = \{\theta_\phi^*, \theta_{g_T}\}$ and parameters $\theta_\phi^*$ may be frozen or fine-tuned.

\noindent \acf{MTL} on the other hand trains a single model for multiple related tasks $\{\mathcal{T}_k = \{\mathcal{Y}_k, f_k(\cdot)\}\}_{k=1}^M$ on a shared domain $\mathcal{D} = \{\mathcal{X}, P(\mathbf{x})\}$. It employs a shared backbone $\phi(\cdot; \theta_\phi)$ and task-specific heads $\{g_k(\cdot; \theta_{g_k})\}_{k=1}^M$. The model $h(\mathbf{x}; \theta)$ outputs for all the tasks\\
$\{g_1(\phi(\mathbf{x}; \theta_\phi); \theta_{g_1}), \dots, g_M(\phi(\mathbf{x}; \theta_\phi); \theta_{g_M})\}$.
\begin{equation}
	\theta^* = \arg\min_{\theta} \frac{1}{N} \sum_{i=1}^N \sum_{k=1}^M \alpha_k \mathcal{L}_k(g_k(\phi(\mathbf{x}_i; \theta_\phi); \theta_{g_k}), y_{i,k})
	\myequations{Multi-task Learning Optimization Objective.}
\end{equation}
where $\theta = \{\theta_\phi, \{\theta_{g_k}\}_{k=1}^M\}$ and $\alpha_k$ are task weighting coefficients. All parameters in $\theta$ are typically trainable.

\section{MTTL Definition and Typology}
\label{sec:mttl_def}
By combining the principles of Transfer Learning (TL) and Multi-Task Learning (MTL), we describe Multi-Task Transfer Learning (MTTL) as a unified strategy that combines the strengths of both. The idea is to use shared representations learned from pre-trained models and train multiple related tasks together. We define \ac{MTTL} as follows:

\paragraph{Formal Definition:}
\textbf{Multi-Task Transfer Learning (MTTL)} is a learning paradigm designed to solve a set of $M$ target tasks $\{\mathcal{T}_k\}_{k=1}^M$ by optimizing a joint objective function that is conditioned on the parameters $\theta_{\phi_S}^*$ of a backbone model pre-trained on a source task $\mathcal{T}_S$. A specific MTTL method is an instantiation of this paradigm, defined by its \textbf{Typology}, a selected \textbf{Fine-Tuning Strategy}, and a \textbf{Task-Balancing Strategy}.

\paragraph{Typology}
MTTL can be categorized along multiple axes, we propose a typology based on the flow of knowledge and tasks, which is particularly relevant to medical imaging:
\begin{itemize}
	\item \textbf{Source-to-Target Cardinality:} This describes the relationship between the number of source and target tasks. Common settings include \textit{One-to-Many} (e.g., ImageNet pre-training for multiple diagnostic tasks), \textit{Many-to-Many}, and \textit{Many-to-One}. Our framework is formulated in the \textit{One-to-Many} setting.
	\item \textbf{Fine-Tuning Formulation:} This specifies how the pre-trained knowledge is utilized and adapted during the multi-task optimization phase. Key strategies, which will be formalized in Section~\ref{ssec:derivations_finetuning}, include Frozen Backbone, Full Fine-Tuning, and Parameter-Efficient Fine-Tuning (PEFT).
\end{itemize}

\section{The MTTL Problem Formulation}
We formulate our MTTL problem in a \textit{One-to-Many} setting, where a single source task $\mathcal{T}_S$ provides a pre-trained backbone, which is then fine-tuned on $K$ target tasks $\{\mathcal{T}_k\}_{k=1}^K$.

\paragraph{Stage 1: Source Representation Learning.}
Let $h_S(\cdot; \theta_S) = g_S(\phi(\cdot; \theta_\phi); \theta_{g_S})$ be the source model. We obtain the optimal source backbone parameters $\theta_\phi^*$ by solving the standard STL problem on the source data:
\begin{equation}
	\theta_\phi^* = \left( \arg\min_{\theta_\phi, \theta_{g_S}} \hat{R}_S(\theta_\phi, \theta_{g_S}) \right)_{\theta_\phi}
	\myequations{Source Backbone Parameter Extraction.}
\end{equation}
where $(\cdot)_{\theta_\phi}$ denotes projection onto the parameter space of the backbone.

\paragraph{Stage 2: Target Multi-Task Optimization.}
We transfer the backbone $\phi(\cdot; \theta_\phi^*)$ and attach $K$ new heads $\{g_k(\cdot; \theta_{g_k})\}$. The foundation of our framework is the following master objective function for the target stage:
\begin{equation}
	\min_{\Theta_{\text{train}}} \quad \mathcal{J}(\Theta_{\text{train}}, \{\lambda_k\}) = \sum_{k=1}^K \lambda_k \hat{R}_k(\Theta_{\text{train}}) + \Omega_{\text{reg}}(\Theta_{\text{train}})\\
	\myequations{MTTL Master Objective Function.}
	\label{eq:mttl_master_objective}
\end{equation}

\noindent
Finally, to define a concrete learning problem or in other words to formalize the learning objective, we specify two key components the \textbf{Fine-Tuning Strategy} to determine the set of trainable parameters, regularization scope for the shared backbone ($\Theta_{\text{train}}, \Omega_{\text{reg}}$) and the \textbf{Task-Balancing Strategy} ($\{\lambda_k\}$), which controls the relative weighting of individual task losses.

\subsection{Derivations by Fine-Tuning Strategy}
\label{ssec:derivations_finetuning}
The fine-tuning strategy defines the set of trainable parameters $\Theta_{\text{train}}$ and the regularization term $\Omega_{\text{reg}}$ in \eqref{eq:mttl_master_objective}. This choice specifies which parameters are updated and how they are constrained.

\paragraph{Formulation 1: Frozen Backbone (Feature Extraction).}
The backbone $\phi(\cdot; \theta_\phi^*)$ is used as a fixed feature extractor. The trainable parameters consist only of the new heads.
\begin{align*}
	\text{Let} \quad \Theta_{\text{train}} &:= \{\theta_{g_k}\}_{k=1}^M \quad \text{and} \quad \Omega_{\text{reg}} := 0 \\
	\implies \mathcal{J} &= \sum_{k=1}^M \lambda_k \hat{R}_k(\theta_\phi^*, \theta_{g_k})
\end{align*}

\paragraph{Formulation 2: Full Fine-Tuning.}
All layers of the pre-trained backbone are adapted. We define this as learning a dense update matrix $\Delta\theta_\phi$ for the initial backbone weights $\theta_\phi^*$.
\begin{align*}
	\text{Let} \quad \Theta_{\text{train}} &:= \{\Delta\theta_\phi, \{\theta_{g_k}\}_{k=1}^M\} \quad \text{and} \quad \Omega_{\text{reg}} := \gamma ||\Delta\theta_\phi||_2^2 \\
	\implies \mathcal{J} &= \sum_{k=1}^M \lambda_k \hat{R}_k(\theta_\phi^* + \Delta\theta_\phi, \theta_{g_k}) + \gamma ||\Delta\theta_\phi||_2^2
\end{align*}

\paragraph{Formulation 3: Partial Fine-Tuning.}
A subset of the backbone layers, typically the deeper, more task-specific ones, are unfrozen and fine-tuned. Let the backbone parameters be partitioned into a frozen set $\theta_{\phi, \text{frozen}}^*$ and a trainable set $\theta_{\phi, \text{tune}}^*$. We learn an update $\Delta\theta_{\phi, \text{tune}}$.
\begin{align*}
	\text{Let} \quad \Theta_{\text{train}} &:= \{\Delta\theta_{\phi, \text{tune}}, \{\theta_{g_k}\}_{k=1}^M\} \quad \text{and} \quad \Omega_{\text{reg}} := \gamma ||\Delta\theta_{\phi, \text{tune}}||_2^2 \\
	\implies \mathcal{J} &= \sum_{k=1}^M \lambda_k \hat{R}_k(\{\theta_{\phi, \text{frozen}}^*, \theta_{\phi, \text{tune}}^* + \Delta\theta_{\phi, \text{tune}}\}, \theta_{g_k}) + \gamma ||\Delta\theta_{\phi, \text{tune}}||_2^2
\end{align*}

\paragraph{Formulation 4: Parameter-Efficient Fine-Tuning (PEFT).}
This approach, exemplified by LoRA \parencite{Hu2021LoRA}, keeps the entire backbone $\theta_\phi^*$ frozen but injects a small set of new, trainable parameters $\theta_\psi$ (the adapters).
\begin{align*}
	\text{Let} \quad \Theta_{\text{train}} &:= \{\theta_\psi, \{\theta_{g_k}\}_{k=1}^M\} \quad \text{and} \quad \Omega_{\text{reg}} := 0 \\
	\implies \mathcal{J} &= \sum_{k=1}^M \lambda_k \hat{R}_k(\phi_{\text{adapted}}(\cdot; \theta_\phi^*, \theta_\psi), \theta_{g_k})
\end{align*}

\paragraph{Formulation 5: Hybrid Tuning (Our Approach).}
This strategy, central to our experimental investigation, combines Partial Fine-Tuning with PEFT. A subset of the backbone layers is fully fine-tuned while the remaining frozen layers are adapted using PEFT. 
\begin{align*}
	\text{Let} \quad \Theta_{\text{train}} &:= \{\Delta\theta_{\phi, \text{tune}}, \theta_\psi, \{\theta_{g_k}\}_{k=1}^M\} \quad \text{and} \quad \Omega_{\text{reg}} := \gamma ||\Delta\theta_{\phi, \text{tune}}||_2^2 \\
	\implies \mathcal{J} &= \sum_{k=1}^M \lambda_k \hat{R}_k(\phi_{\text{adapted}}(\cdot; \{\theta_{\phi, \text{frozen}}^*, \theta_{\phi, \text{tune}}^* + \Delta\theta_{\phi, \text{tune}}\}, \theta_\psi), \theta_{g_k}) + \gamma ||\Delta\theta_{\phi, \text{tune}}||_2^2
\end{align*}

\noindent The choice of which backbone layers to unfreeze is critical. Based on established principles of transfer learning \parencite{yosinski2014transferable}, earlier layers of a CNN learn generic, low-level features (e.g., edges, textures), while later layers learn more abstract, task-specific features. To preserve the generic features while still tuning the high-level representations to our specific domain, we chose to unfreeze only the final residual block (\texttt{layer4}) of the ResNet-50 backbone. The remaining, earlier layers were kept frozen but were adapted using LoRA. This hybrid approach aims to achieve the best of both worlds deep feature adaptation with maximal parameter efficiency.

\subsection{Derivations by Task-Balancing Strategy}
\label{ssec:derivations_balancing}
This component defines the task weights $\{\lambda_k\}$ and can be applied to any of the fine-tuning formulations above.

\paragraph{Formulation A: Manual Weighting.}
The weights are set as fixed hyperparameters, $\lambda_k = c_k$. A common constraint is to enforce a convex combination such that
\[
\lambda_k \ge 0 \quad \text{and} \quad \sum_{k=1}^M \lambda_k = 1
\]

\paragraph{Formulation B: Uncertainty Weighting.}
This approach re-frames the objective based on maximizing a multi-task Gaussian likelihood, where each task's uncertainty $\sigma_k^2$ is a learnable parameter \parencite{Kendall2018MultiTaskLearningUncertainty}. This corresponds to setting $\lambda_k = (2\sigma_k^2)^{-1}$ and adding an uncertainty regularization term. The master objective becomes a joint minimization over model and uncertainty parameters:
\begin{equation}
	\min_{\Theta_{\text{train}}, \{\sigma_k\}} \quad \sum_{k=1}^M \left( \frac{1}{2\sigma_k^2} \hat{R}_k(\Theta_{\text{train}}) + \log \sigma_k \right) + \Omega_{\text{reg}}(\Theta_{\text{train}})
	\myequations{MTTL Objective with Uncertainty Weighting.}
\end{equation}

\paragraph{Formulation C: Gradient Normalization (GradNorm).}
This method \parencite{Chen2018GradNorm} dynamically tunes $\{\lambda_k(t)\}$ at each training step $t$ to equalize learning speeds across tasks. Let $\theta_{sh} \subseteq \Theta_{\text{train}}$ be the trainable shared parameters. The weights $\{\lambda_k\}$ are updated to minimize a gradient loss $\mathcal{L}_{\text{grad}}(t, \{\lambda_k\})$ that measures the disparity in gradient magnitudes relative to each task's training rate.
\cleardoublepage

%METHODOLOGY
\chapter{Methodological Application to Malaria Diagnosis}
\label{chap:methodology}
The previous chapter established the theoretical framework for Multi-Task Transfer Learning. In this chapter, we apply the defined framework to the practical challenge of automated malaria diagnosis from Giemsa stained thin blood smears. We begin by defining the specific MTTL objective function used in this study, including the selected fine-tuning and task balancing strategies. Then we describe the data curation and preprocessing steps required to ensure model stability. This is followed by an overview of the model architecture, which features a shared backbone adapted via Low Rank Adaptation (LoRA) and specialized decoder heads. Finally, we outline the experimental protocols designed to benchmark the system against Single-Task Learning (STL) baselines and evaluate its performance across different data subsets.

\section{Instantiated MTTL Objective for Malaria Diagnosis}
\label{sec:instantiated_mttl_objective}

Our framework adopts a \textbf{One-to-Many} topology, using a single source model pre-trained on ImageNet to learn multiple tasks relevant to malaria diagnosis. To achieve this, we combine two specific strategies:

\begin{itemize}
	\item \textbf{Fine-Tuning Strategy:} We employ both \textbf{Parameter-Efficient Fine-Tuning (PEFT)} using LoRA adapters and a \textbf{Hybrid Tuning} approach. This enables a comparison between highly efficient adaptation and a more comprehensive fine-tuning of deeper backbone layers to balance performance against computational cost.
	\item \textbf{Task-Balancing Strategy:} We utilize \textbf{Uncertainty Weighting} to manage the losses of heterogeneous tasks. This automated approach allows the model to dynamically learn the optimal contribution of each task during training, avoiding the difficulties of manual hyperparameter tuning.
\end{itemize}

\noindent
These choices define the final optimization problem addressed in our experiments. \\

\noindent Let $\theta_\phi^*$ represent the parameters of the pre-trained source backbone. By integrating the chosen fine-tuning strategy with Uncertainty Weighting, the objective function is expressed as:

\begin{equation}
	\min_{\Theta_{\text{train}}, \{\theta_{g_k}\}, \{\sigma_k\}} \quad 
	\sum_{k=1}^M \left( \frac{1}{2\sigma_k^2}\,\hat{R}_k\big(\Theta_{\text{train}},\theta_{g_k}\big) + \log \sigma_k \right)
	+ \Omega_{\text{reg}}(\Theta_{\text{train}})
	\label{eq:mtl_objective_single_clean}
\end{equation}

\noindent
where the empirical risk for task $k$ is defined as:

\begin{equation}
	\hat{R}_k\big(\Theta_{\text{train}},\theta_{g_k}\big)
	\;=\;
	\frac{1}{N_k}\sum_{i=1}^{N_k}
	\mathcal{L}_k\!\Big( 
	g_k\big(\phi_{\text{adapted}}(\mathbf{x}_i;\theta_\phi^*,\Theta_{\text{train}});\theta_{g_k}\big),\; y_{ik}
	\Big).
	\label{eq:empirical_risk_clean}
\end{equation}

\noindent
Substituting \eqref{eq:empirical_risk_clean} into \eqref{eq:mtl_objective_single_clean} yields the explicit double-sum formulation:

\begin{equation}
	\min_{\Theta_{\text{train}}, \{\theta_{g_k}\}, \{\sigma_k\}} \quad
	\sum_{k=1}^M \left[
	\frac{1}{2\sigma_k^2 N_k}\sum_{i=1}^{N_k}
	\mathcal{L}_k\!\Big( 
	g_k\big(\phi_{\text{adapted}}(\mathbf{x}_i;\theta_\phi^*,\Theta_{\text{train}});\theta_{g_k}\big),\; y_{ik}
	\Big)
	+ \log \sigma_k
	\right]
	+ \Omega_{\text{reg}}(\Theta_{\text{train}})
	\myequations{Instantiated MTTL Objective.}
	\label{eq:final_objective_double_sum_clean}
\end{equation}

\section{Data Curation and Preprocessing Pipeline}
\label{sec:data_curation_and_pipeline}
\subsection{Dataset}
\label{ssec:dataset_description}

We utilize the publicly available \textbf{Malaria Dataset from the \acf{NLM}} \cite{NLMMalariaDataset}, curated by the Lister Hill National Center for Biomedical Communications \parencite{Rajaraman2018NIHMalariaDataset, KassimClustering}. The dataset contains Giemsa-stained thin blood smear images from \textbf{193 patients} (148 \textit{P. falciparum}-infected and 45 healthy), captured using a smartphone attached to a light microscope. For this study, we focus on three primary annotations essential for diagnosis:
\begin{itemize}
	\item \textbf{Parasitized:} \acp{RBC} containing one or more malaria parasites.
	\item \textbf{Uninfected:} Healthy red blood cells.
	\item \textbf{\acp{WBC}:} Leukocytes, a critical negative class that the system must distinguish from parasites.
\end{itemize}

\noindent As shown in Table~\ref{tab:dataset_class_distribution}, the data reflects the reality of clinical samples with severe \textbf{class imbalance}. Uninfected cells vastly outnumber parasitized ones, while white blood cells are extremely rare. This disparity motivates the loss-balancing and data-sampling strategies used in our experiments in order to fix this.

\begin{table}[htbp]
	\centering
	\caption{Class Distribution of Annotated Cells in the Dataset.}
	\label{tab:dataset_class_distribution}
	\begin{tabular}{@{}lr@{}}
		\toprule
		\textbf{Cell Class} & \textbf{Total Annotations} \\
		\midrule
		Uninfected & 155,640 \\
		Parasitized & 6,810 \\
		White Blood Cell & 220 \\
		\bottomrule
	\end{tabular}
\end{table}

\noindent To ensure unbiased evaluation, we partitioned the dataset into training (80\%), validation (10\%), and test (10\%) sets at the \textbf{patient level}. This prevents data leakage by guaranteeing that the model never encounters images from the same patient in both training and testing.

\subsection{Image Preprocessing Pipeline}
\label{ssec:preprocessing_pipeline}
Microscopy images frequently vary in staining, illumination, and acquisition hardware. To mitigate these domain shifts, we designed a sequential preprocessing pipeline. Figure~\ref{fig:dataset_samples} displays raw samples, while Figure~\ref{fig:preprocessed_samples} shows the corresponding results after processing.

\paragraph{Contrast Enhancement (CLAHE).}  
To highlight intracellular structures like ring-stage parasites, we applied \textbf{Contrast Limited Adaptive Histogram Equalization (CLAHE)} \parencite{Pizer1987AdaptiveHistogramEqualization}. Images were converted to the LAB color space, and CLAHE was applied only to the Luminance channel to improve contrast while preserving color fidelity.

\paragraph{Resizing.}  
All images were standardized to $512 \times 512$ pixels using Lanczos interpolation. We favored this method over padding to ensure the entire image region contributes useful features, accepting a minor trade-off in aspect ratio preservation.

\paragraph{Data Augmentation and Normalization.}  
We applied on-the-fly augmentations during training, including random flips, $90^{\circ}$ rotations, color adjustments, and noise simulation. Finally, pixel intensities were normalized using standard ImageNet statistics to ensure compatibility with the pretrained backbone:
\begin{equation}
	\mu = (0.485,\; 0.456,\; 0.406), \quad 
	\sigma = (0.229,\; 0.224,\; 0.225).
\end{equation}
\noindent Normalization was applied only during training and inference, not for the visualization grids (Figures~\ref{fig:dataset_samples} and~\ref{fig:preprocessed_samples}).

\begin{figure}[htbp]
	\centering	
	\begin{subfigure}[b]{0.32\textwidth}
		\centering
		\includegraphics[width=\textwidth, height=3cm, keepaspectratio]{Images/Samples/sample1.jpg}
		\caption{Sample 1}
	\end{subfigure}
	\hfill
	\begin{subfigure}[b]{0.32\textwidth}
		\centering
		\includegraphics[width=\textwidth, height=3cm, keepaspectratio]{Images/Samples/sample2.jpg}
		\caption{Sample 2}
	\end{subfigure}
	\hfill
	\begin{subfigure}[b]{0.32\textwidth}
		\centering
		\includegraphics[width=\textwidth, height=3cm, keepaspectratio]{Images/Samples/sample3.jpg}
		\caption{Sample 3}
	\end{subfigure}
	
	\vspace{1mm}
	
	\begin{subfigure}[b]{0.32\textwidth}
		\centering
		\includegraphics[width=\textwidth, height=3cm, keepaspectratio]{Images/Samples/sample4.jpg}
		\caption{Sample 4}
	\end{subfigure}
	\hfill
	\begin{subfigure}[b]{0.32\textwidth}
		\centering
		\includegraphics[width=\textwidth, height=3cm, keepaspectratio]{Images/Samples/sample5.jpg}
		\caption{Sample 5}
	\end{subfigure}
	\hfill
	\begin{subfigure}[b]{0.32\textwidth}
		\centering
		\includegraphics[width=\textwidth, height=3cm, keepaspectratio]{Images/Samples/sample6.jpg}
		\caption{Sample 6}
	\end{subfigure}
	
	\vspace{1mm}
	
	\begin{subfigure}[b]{0.32\textwidth}
		\centering
		\includegraphics[width=\textwidth, height=3cm, keepaspectratio]{Images/Samples/sample7.jpg}
		\caption{Sample 7}
	\end{subfigure}
	\hfill
	\begin{subfigure}[b]{0.32\textwidth}
		\centering
		\includegraphics[width=\textwidth, height=3cm, keepaspectratio]{Images/Samples/sample8.jpg}
		\caption{Sample 8}
	\end{subfigure}
	\hfill
	\begin{subfigure}[b]{0.32\textwidth}
		\centering
		\includegraphics[width=\textwidth, height=3cm, keepaspectratio]{Images/Samples/sample9.jpg}
		\caption{Sample 9}
	\end{subfigure}
	
	\caption{Original Samples from NLM Thin Falciparum dataset.}
	\label{fig:dataset_samples}
\end{figure}

\begin{figure}[htbp]
	\centering	
	\begin{subfigure}[b]{0.32\textwidth}
		\centering
		\includegraphics[width=\textwidth, height=3cm]{Images/Samples_Preprocessed/sample1.jpg}
		\caption{Sample 1}
	\end{subfigure}
	\begin{subfigure}[b]{0.32\textwidth}
		\centering
		\includegraphics[width=\textwidth, height=3cm]{Images/Samples_Preprocessed/sample2.jpg}
		\caption{Sample 2}
	\end{subfigure}
	\begin{subfigure}[b]{0.32\textwidth}
		\centering
		\includegraphics[width=\textwidth, height=3cm]{Images/Samples_Preprocessed/sample3.jpg}
		\caption{Sample 3}
	\end{subfigure}
	
	\vspace{1mm} 
	
	\begin{subfigure}[b]{0.32\textwidth}
		\centering
		\includegraphics[width=\textwidth, height=3cm]{Images/Samples_Preprocessed/sample4.jpg}
		\caption{Sample 4}
	\end{subfigure}
	\begin{subfigure}[b]{0.32\textwidth}
		\centering
		\includegraphics[width=\textwidth, height=3cm]{Images/Samples_Preprocessed/sample5.jpg}
		\caption{Sample 5}
	\end{subfigure}
	\begin{subfigure}[b]{0.32\textwidth}
		\centering
		\includegraphics[width=\textwidth, height=3cm]{Images/Samples_Preprocessed/sample6.jpg}
		\caption{Sample 6}
	\end{subfigure}
	
	\vspace{1mm} 
	
	\begin{subfigure}[b]{0.32\textwidth}
		\centering
		\includegraphics[width=\textwidth, height=3cm]{Images/Samples_Preprocessed/sample7.jpg}
		\caption{Sample 7}
	\end{subfigure}
	\begin{subfigure}[b]{0.32\textwidth}
		\centering
		\includegraphics[width=\textwidth, height=3cm]{Images/Samples_Preprocessed/sample8.jpg}
		\caption{Sample 8}
	\end{subfigure}
	\begin{subfigure}[b]{0.32\textwidth}
		\centering
		\includegraphics[width=\textwidth, height=3cm]{Images/Samples_Preprocessed/sample9.jpg}
		\caption{Sample 9}
	\end{subfigure}
	
	\caption{Preprocessed version of the images from Figure~\ref{fig:dataset_samples}.}
	\label{fig:preprocessed_samples}
\end{figure}

\section{Tasks Formulation}
\label{sec:task_formulation}

We formulate malaria diagnosis as a set of $M\leq4$ synergistic tasks, centered on Multi-Class Cell Detection. Each task is defined by its objective, label space, and a loss function.

\begin{itemize}
	\item \textbf{Task 1: Multi-Class Cell Detection ($\mathcal{T}_{\text{det}}$):} The primary task.
	\begin{itemize}
		\item \textbf{Objective:} Localize every cell of interest with a bounding box and assign it a class label.
		\item \textbf{Label Space ($\mathcal{Y}_{\text{det}}$):} $y_{\text{det}} = \{(\mathbf{b}_j, c_j)\}_{j=1}^{N_{\text{obj}}}$, where $\mathbf{b}_j \in \mathbb{R}^4$ are coordinates and $c_j \in \{1, \dots, C\}$ is the class label.
		\item \textbf{Loss Function ($\mathcal{L}_{\text{det}}$):} Aggregates four equally weighted components from Faster R-CNN \parencite{Ren2015FasterRCNN}:
		\begin{equation}
			\mathcal{L}_{\text{det}} = 
			\mathcal{L}_{\text{RPN\_cls}} + 
			\mathcal{L}_{\text{RPN\_reg}} + 
			\mathcal{L}_{\text{RoI\_cls}} + 
			\mathcal{L}_{\text{RoI\_reg}}
			\myequations{Total Multi-Class Detection Loss.}
			\label{eq:det_loss}
		\end{equation}
		The Region Proposal Network (RPN) uses Binary Cross-Entropy to distinguish objects from background. The final Region of Interest (RoI) classification employs \textbf{Focal Loss} \parencite{Lin2017Focal} ($\alpha_t=0.25, \gamma=2.0$) to handle severe class imbalance. Bounding box regression uses the \textbf{Smooth L1 Loss} for stability:
		\begin{equation}
			\text{smooth}_{L_1}(x) = \begin{cases} 0.5x^2, & |x| < 1 \\ |x| - 0.5, & \text{otherwise} \end{cases}
			\myequations{Smooth L1 Loss component.}
		\end{equation}
	\end{itemize}
	
	\item \textbf{Task 2: Cell Segmentation ($\mathcal{T}_{\text{seg}}$):} An auxiliary dense prediction task providing shape priors.
	\begin{itemize}
		\item \textbf{Objective:} Generate a binary mask for all cells.
		\item \textbf{Label Space ($\mathcal{Y}_{\text{seg}}$):} Binary mask $\mathbf{M} \in \{0, 1\}^{64 \times 64}$.
		\item \textbf{Loss Function ($\mathcal{L}_{\text{seg}}$):} An equally weighted sum ($w_{\text{bce}}=w_{\text{dice}}=0.5$) of Binary Cross-Entropy and Dice Loss:
		\begin{equation}
			\mathcal{L}_{\text{seg}} = \frac{1}{2} \left( -\frac{1}{N}\sum_{i} \big[ M_i\log\hat{M}_i + (1{-}M_i)\log(1{-}\hat{M}_i) \big] + 1 - \frac{2 \sum \hat{M}_i M_i + \epsilon}{\sum \hat{M}_i + \sum M_i + \epsilon} \right)
			\myequations{Cell Segmentation Loss.}
			\label{eq:seg_loss}
		\end{equation}
	\end{itemize}
	
	\item \textbf{Task 3: Infection Localization ($\mathcal{T}_{\text{loc}}$):} An auxiliary regression task serving as an attention prior.
	\begin{itemize}
		\item \textbf{Objective:} Produce a heatmap highlighting parasite centers.
		\item \textbf{Label Space ($\mathcal{Y}_{\text{loc}}$):} Continuous heatmap $\mathbf{H} \in [0, 1]^{64 \times 64}$.
		\item \textbf{Loss Function ($\mathcal{L}_{\text{loc}}$):} A composite loss designed for imbalance and boundary alignment inspired by \textcite{abraham2019novel}:
		\begin{equation}
			\begin{aligned}
				\mathcal{L}_{\text{loc}} &= 0.8 \cdot \left( 1 - \text{TI} \right)^{\gamma} + 0.2 \cdot \left\lVert \nabla \hat{\mathbf{H}} - \nabla \mathbf{H} \right\rVert_1 \\
				\text{where } \text{TI} &= \frac{\sum \hat{H}_i H_i + \epsilon}{\sum \hat{H}_i H_i + \alpha \sum \hat{H}_i(1-H_i) + \beta \sum (1-\hat{H}_i)H_i + \epsilon}
			\end{aligned}
			\myequations{Infection Localization Loss.}
			\label{eq:heatmap_loss}
		\end{equation}
		We use Focal Tversky Loss ($\alpha=0.7, \beta=0.3, \gamma=0.75$) to penalize false negatives, supplemented by a Boundary Loss on Sobel gradients.
	\end{itemize}
	
	\item \textbf{Task 4: RoI Classification ($\mathcal{T}_{\text{roi}}$):} Provides direct supervisory signals to the backbone.
	\begin{itemize}
		\item \textbf{Objective:} Classify pooled features from ground-truth RoIs.
		\item \textbf{Label Space ($\mathcal{Y}_{\text{roi}}$):} Class label $c \in \{1, \dots, C\}$.
		\item \textbf{Loss Function ($\mathcal{L}_{\text{roi}}$):} \textbf{Weighted Cross-Entropy Loss} to address imbalance:
		\begin{equation}
			\mathcal{L}_{\text{roi}}(\mathbf{z}, c) = - w_c \log\left(\frac{\exp(z_c)}{\sum_{j=1}^{C} \exp(z_j)}\right)
			\myequations{RoI Classification Loss.}
			\label{eq:classif_loss}
		\end{equation}
		Weights $w_c$ are set to $[23.0, 1.0, 100.0]$ for \{Infected, Healthy, WBC\} respectively.
	\end{itemize}
\end{itemize}

\noindent Figure~\ref{fig:task_data_visualization} visualizes these tasks on a sample image. Detection identifies cell types (Infected: red, Healthy: blue, WBC: lime), while Segmentation outlines boundaries. Localization highlights parasite regions, and RoI Classification distinguishes cell features. Together, these tasks mirror the step-wise reasoning of a human expert.

\begin{figure}[t]
	\centering
	\caption{Visualization of Ground Truth Data for the Multi-Task Framework.}
	\label{fig:task_data_visualization}
	
	\captionsetup{skip=2pt, font=small} 
	
	\begin{subfigure}[b]{0.19\textwidth}
		\includegraphics[width=\linewidth, keepaspectratio]{Images/Task_Visualizations/input_sample_1.jpg}
		\caption*{Input 1} 
	\end{subfigure}\hfill
	\begin{subfigure}[b]{0.19\textwidth}
		\includegraphics[width=\linewidth, keepaspectratio]{Images/Task_Visualizations/input_sample_2.jpg}
		\caption*{Input 2}
	\end{subfigure}\hfill
	\begin{subfigure}[b]{0.19\textwidth}
		\includegraphics[width=\linewidth, keepaspectratio]{Images/Task_Visualizations/input_sample_3.jpg}
		\caption*{Input 3}
	\end{subfigure}\hfill
	\begin{subfigure}[b]{0.19\textwidth}
		\includegraphics[width=\linewidth, keepaspectratio]{Images/Task_Visualizations/input_sample_4.jpg}
		\caption*{Input 4}
	\end{subfigure}
	
	\vspace{1mm}
	
	\centerline{\footnotesize\textbf{Task: Detection ($\mathcal{T}_{\text{det}}$)}}
	\vspace{1mm}
	\begin{subfigure}[b]{0.19\textwidth}
		\includegraphics[width=\linewidth, keepaspectratio]{Images/Task_Visualizations/det_sample_1.jpg}
	\end{subfigure}\hfill
	\begin{subfigure}[b]{0.19\textwidth}
		\includegraphics[width=\linewidth, keepaspectratio]{Images/Task_Visualizations/det_sample_2.jpg}
	\end{subfigure}\hfill
	\begin{subfigure}[b]{0.19\textwidth}
		\includegraphics[width=\linewidth, keepaspectratio]{Images/Task_Visualizations/det_sample_3.jpg}
	\end{subfigure}\hfill
	\begin{subfigure}[b]{0.19\textwidth}
		\includegraphics[width=\linewidth, keepaspectratio]{Images/Task_Visualizations/det_sample_4.jpg}
	\end{subfigure}
	
	\vspace{1mm} 
	
	\centerline{\footnotesize\textbf{Task: Segmentation ($\mathcal{T}_{\text{seg}}$)}}
	\vspace{1mm}
	\begin{subfigure}[b]{0.19\textwidth}
		\includegraphics[width=\linewidth, keepaspectratio]{Images/Task_Visualizations/seg_sample_1.jpg}
	\end{subfigure}\hfill
	\begin{subfigure}[b]{0.19\textwidth}
		\includegraphics[width=\linewidth, keepaspectratio]{Images/Task_Visualizations/seg_sample_2.jpg}
	\end{subfigure}\hfill
	\begin{subfigure}[b]{0.19\textwidth}
		\includegraphics[width=\linewidth, keepaspectratio]{Images/Task_Visualizations/seg_sample_3.jpg}
	\end{subfigure}\hfill
	\begin{subfigure}[b]{0.19\textwidth}
		\includegraphics[width=\linewidth, keepaspectratio]{Images/Task_Visualizations/seg_sample_4.jpg}
	\end{subfigure}
	
	\vspace{1mm} 
	
	\centerline{\footnotesize\textbf{Task: Localization ($\mathcal{T}_{\text{loc}}$)}}
	\vspace{1mm}
	\begin{subfigure}[b]{0.19\textwidth}
		\includegraphics[width=\linewidth, keepaspectratio]{Images/Task_Visualizations/heatmap_sample_1.jpg}
	\end{subfigure}\hfill
	\begin{subfigure}[b]{0.19\textwidth}
		\includegraphics[width=\linewidth, keepaspectratio]{Images/Task_Visualizations/heatmap_sample_2.jpg}
	\end{subfigure}\hfill
	\begin{subfigure}[b]{0.19\textwidth}
		\includegraphics[width=\linewidth, keepaspectratio]{Images/Task_Visualizations/heatmap_sample_3.jpg}
	\end{subfigure}\hfill
	\begin{subfigure}[b]{0.19\textwidth}
		\includegraphics[width=\linewidth, keepaspectratio]{Images/Task_Visualizations/heatmap_sample_4.jpg}
	\end{subfigure}
	
\end{figure}

\section{MTTL Model Architecture}
\label{sec:model_architecture}
The architecture of our proposed system is a modular and flexible instantiation of the MTTL framework, designed for both rigorous experimentation and high performance. It adheres to the prevalent hard parameter sharing paradigm, comprising two primary parts a single, powerful shared backbone that learns a common feature representation, and a set of specialized, task-specific decoders (heads) that interpret these features. \\

\begin{figure}[htbp]
	\centering
	\scalebox{1.0}{
		\begin{tikzpicture}[
			xshift=-1.5cm,
			node distance=1.1cm and 0.9cm,
			every node/.style={font=\sffamily\small},
			]
			
			\tikzset{
				stem_block/.style={rectangle, draw=purple!40!black, thick, fill=purple!12, rounded corners, minimum width=3.0cm, minimum height=0.85cm, align=center},
				backbone_block/.style={rectangle, draw=blue!50!black, thick, fill=blue!6, rounded corners, minimum width=3.0cm, minimum height=0.95cm, align=center},
				trainable_block/.style={rectangle, draw=purple!90!black, thick, fill=purple!20, rounded corners, minimum width=3.0cm, minimum height=0.95cm, align=center},
				lora_block/.style={rectangle, draw=orange!60!black, thick, fill=orange!22, rounded corners, minimum width=3.4cm, minimum height=1.3cm},
				fpn_block/.style={rectangle, draw=cyan!60!black, thick, fill=cyan!12, rounded corners, minimum width=3.6cm, minimum height=1.05cm, align=center},
				head_block/.style={rectangle, draw=green!50!black, thick, fill=green!12, rounded corners, minimum width=3.2cm, minimum height=0.95cm, align=center},
				head_group/.style={draw=gray!50, fill=gray!5, rounded corners, inner sep=8pt},
				small_badge/.style={draw=orange!65!black, fill=orange!10, rounded corners=2pt, font=\sffamily\scriptsize\bfseries, inner sep=2pt, text=orange!85!black},
				thin_arrow/.style={-Stealth, line width=0.9pt, draw=black!70},
				feat_arrow/.style={-Stealth, line width=0.9pt, rounded corners=3pt, draw=orange!65!black},
				fpn_arrow/.style={-Stealth, dashed, line width=0.9pt, draw=cyan!70!black},
				output_bbox/.style={rectangle, draw=black!40, thick, fill=gray!10, rounded corners, minimum width=2.8cm, minimum height=1.2cm, align=center},
				output_mask/.style={rectangle, draw=black!40, thick, fill=gray!10, minimum size=1.5cm,
					path picture={
						\foreach \i in {0,1,...,7} {
							\draw[gray!40] (path picture bounding box.north west) ++(0.1875*\i cm, 0) -- ++(0, -1.5cm);
							\draw[gray!40] (path picture bounding box.north west) ++(0, -0.1875*\i cm) -- ++(1.5cm, 0);
						}
				}},
			}
			
			\node[draw=black!70, thick, minimum size=1.8cm, label={[below, yshift=3mm, font=\tiny\ttfamily]512x512x3}] (input) {\includegraphics[width=1.6cm]{DIAGRAMS/input.png}};
			\node[stem_block, below=0.8cm of input] (stem) {Stem};
			\node[below=0mm of stem, font=\tiny\ttfamily, text=black!70] (stemshape) {256x256x64};
			\draw[thin_arrow, draw=purple!70!black] (input) -- (stem);
			
			\node[lora_block, below=1.0cm of stem] (l1) {};
			\node[backbone_block, at=(l1)] (s1) {Stage 1};
			\node[below=2mm of s1, font=\tiny\ttfamily, text=black!70] {128x128x256};
			\coordinate (badge_shift) at (-3cm,0.12cm);
			\node[small_badge, anchor=south west] (l1badge) at ($ (l1.north east) + (badge_shift) $) {LoRA};
			
			\node[lora_block, below=of l1] (l2) {};
			\node[backbone_block, at=(l2)] (s2) {Stage 2};
			\node[below=2mm of s2, font=\tiny\ttfamily, text=black!70] {64x64x512};
			\node[small_badge, anchor=south west] (l2badge) at ($ (l2.north east) + (badge_shift) $) {LoRA};
			
			\node[lora_block, below=of l2] (l3) {};
			\node[backbone_block, at=(l3)] (s3) {Stage 3};
			\node[below=2mm of s3, font=\tiny\ttfamily, text=black!70] {32x32x1024};
			\node[small_badge, anchor=south west] (l3badge) at ($ (l3.north east) + (badge_shift) $) {LoRA};
			
			\node[lora_block, below=of l3] (l4) {};
			\node[trainable_block, at=(l4)] (s4) {Stage 4};
			\node[below=3mm of s4, font=\tiny\ttfamily, text=black!70] {16x16x2048};
			\node[small_badge, anchor=south west] (l4badge) at ($ (l4.north east) + (badge_shift) $) {LoRA};
			
			\node[circle, draw=black!60, thick, fill=gray!8, below=of l4] (gap) {GAP};
			\node[rectangle, draw=orange!60!black, thick, fill=orange!18, rounded corners, minimum height=1.0cm, minimum width=0.45cm, below=0.4cm of gap] (gvec) {};
			\node[below=0.1cm of gvec, font=\tiny\ttfamily\bfseries] {2048-D};
			\draw[thin_arrow, draw=blue!70!black] (stem) -- (l1);
			\draw[thin_arrow, draw=blue!70!black] (l1) -- (l2);
			\draw[thin_arrow, draw=blue!70!black] (l2) -- (l3);
			\draw[thin_arrow, draw=blue!70!black] (l3) -- (l4);
			\draw[thin_arrow, draw=blue!70!black] (l4) -- (gap);
			\draw[thin_arrow, draw=blue!70!black] (gap) -- (gvec);
			
			\coordinate (head_start) at ($(l2.east) + (5.0, 1.5)$); 
			
			\node[fpn_block, at=(head_start)] (fpn) {Feature Pyramid Network};
			\node[head_block, below=0.5cm of fpn] (det_head) {Detection \& RoI Head};
			\node[output_bbox, right=1.2cm of det_head] (out_det) {BBoxes \& Classes};
			
			\node[head_block, below=3.0cm of det_head] (seg_head) {Segmentation Head};
			\node[output_mask, right=1.2cm of seg_head] (out_seg) {};
			\node[below=1mm of out_seg, font=\tiny\ttfamily] {64$\times$64 Mask};
			
			\node[head_block, below=2.5cm of seg_head] (heat_head) {Localization Head};
			\node[output_mask, right=1.2cm of heat_head] (out_heat) {};
			\node[below=1mm of out_heat, font=\tiny\ttfamily] {64$\times$64 Mask};
			
			\begin{scope}[on background layer]
				\node[draw=blue!50, fill=blue!5, rounded corners, inner sep=8pt, fit=(input) (stem) (l1) (l4) (gvec)] (backbone_zone) {};
				\node[head_group, fit=(fpn) (det_head) (out_det)] (det_group) {};
				\node[head_group, fit=(seg_head) (out_seg)] (seg_group) {};
				\node[head_group, fit=(heat_head) (out_heat)] (heat_group) {};
			\end{scope}
			
			\draw[fpn_arrow] (fpn.south) -- (det_head.north) node[midway, right, font=\tiny] {Pyramid Feats};
			\draw[thin_arrow] (det_head) -- (out_det);
			\draw[thin_arrow] (seg_head) -- (out_seg);
			\draw[thin_arrow] (heat_head) -- (out_heat);
			
			\coordinate (c1_fork) at ($(l1.east)+(1.8cm,0)$);
			\coordinate (c2_fork) at ($(l2.east)+(1.8cm,0)$);
			\coordinate (c3_fork) at ($(l3.east)+(1.8cm,0)$);
			\coordinate (c4_fork) at ($(l4.east)+(1.8cm,0)$);
			
			\draw[feat_arrow] (l1.east) -- (c1_fork) node[midway, above, font=\tiny] {C1};
			\draw[feat_arrow] (l2.east) -- (c2_fork) node[midway, above, font=\tiny] {C2};
			\draw[feat_arrow] (l3.east) -- (c3_fork) node[midway, above, font=\tiny] {C3};
			\draw[feat_arrow] (l4.east) -- (c4_fork) node[midway, above, font=\tiny] {C4};
			
			\draw[feat_arrow] (c2_fork) |- ($(det_group.west)+(0,0.8cm)$) node[above,pos=0.8,font=\tiny] {C2};
			\draw[feat_arrow] (c3_fork) |- ($(det_group.west)+(0,0.5cm)$) node[above,pos=0.8,font=\tiny] {C3};
			\draw[feat_arrow] (c4_fork) |- ($(det_group.west)+(0,0.2cm)$) node[above,pos=0.8,font=\tiny] {C4};
			
			\draw[feat_arrow] (c1_fork) |- ($(seg_group.west)+(0,0.6cm)$) node[above,pos=0.8,font=\tiny] {C1};
			\draw[feat_arrow] (c2_fork) |- ($(seg_group.west)+(0,0.2cm)$) node[above,pos=0.8,font=\tiny] {C2};
			\draw[feat_arrow] (c3_fork) |- ($(seg_group.west)+(0,-0.2cm)$) node[above,pos=0.8,font=\tiny] {C3};
			\draw[feat_arrow] (c4_fork) |- ($(seg_group.west)+(0,-0.6cm)$) node[above,pos=0.8,font=\tiny] {C4};
			
			\draw[feat_arrow] (c1_fork) |- ($(heat_group.west)+(0,0.2cm)$) node[above,pos=0.8,font=\tiny] {C1};
			\draw[feat_arrow] (c2_fork) |- ($(heat_group.west)+(0,-0.2cm)$) node[above,pos=0.8,font=\tiny] {C2};
			
		\end{tikzpicture}
	}
	\captionof{figure}{
		High-level diagram of our MTTL model architecture.
	}
	\label{fig:full_mttl_system}
\end{figure}

\noindent The design is shown conceptually in Figure~\ref{fig:full_mttl_system}, a 512$\times$512$\times$3 input image passes through the backbone stages (blue), with Stage 4 fine-tuned (purple) alongside \acp{LoRA} adapters (orange). Multi-scale features C1 to C4 are extracted from the backbone and passed to task-specific heads: C2 to C4 feed the \ac{FPN} for object detection, C1 to C4 are used for segmentation, and C1 to C2 support localization. This modular architecture enables efficient transfer learning through selective parameter updates while maintaining the representational capacity of the pretrained backbone. The design also facilitates exploration of different task combinations during training. Further component details are provided in the following sections.

\subsection{Shared Backbone with Integrated LoRA Adapters}
\label{ssec:shared_backbone}
We use a \textbf{ResNet-50} backbone \parencite{He2016DeepResidualLearning} pre-trained on ImageNet. We selected this architecture because it captures rich features effectively while avoiding the vanishing gradient issues found in other deep networks. It is more efficient than older models like VGG \parencite{Simonyan2014VeryDeepConvolutional} and offers a cleaner, more modular structure than Inception designs \parencite{Szegedy2016RethinkingInception}. This modularity makes it ideal for integrating adaptation layers.\\

\noindent To adapt the model, we insert \textbf{Low-Rank Adaptation (LoRA)} modules into the residual blocks. As shown in Figure~\ref{fig:lora_diagrams}, these modules add small trainable matrices $(W_A, W_B)$ alongside the frozen layers. This forms our \textbf{Adapted ResNet Backbone}. During training, the original weights $\theta_\phi^*$ remain frozen; only the LoRA parameters $\theta_\psi$ are updated. This approach preserves the pre-trained knowledge while allowing the model to learn the specific requirements of malaria diagnosis. In our \textbf{Hybrid Tuning} strategy, we also unfreeze the final residual block (\textit{layer4}). We fine-tune this layer with a very low learning rate to refine the most abstract features without disrupting the overall hierarchy.

\begin{figure}[htbp]
	\centering
	\begin{subfigure}[b]{0.9\textwidth}
		\centering
		\resizebox{\textwidth}{!}{
			
\begin{tikzpicture}[node distance=2cm and 2.2cm]
    \node (input) {Input $\mathbf{z}$};
    \node[op, right=5cm of input] (add) {$+$};
    \node[right=5cm of add] (output) {Output $\mathbf{z}'$};

    \coordinate (lora_center) at ($(input)!0.5!(add)$);
    \node[cuboid, trainable, below=2cm of lora_center, minimum height=1.8cm] (lora_a) 
        {$\mathbf{W}_A$ \\[-2pt] {\scriptsize $C \times r$}};
    \node[cuboid, trainable, right=2.2cm of lora_a, minimum height=1.2cm] (lora_b) 
        {$\mathbf{W}_B$ \\[-2pt] {\scriptsize $r \times C$}};
    \node[op, right=2.2cm of lora_b] (scale) {$\times \tfrac{\alpha}{r}$};

    \draw[arrow] (input.east) -- (add.west);
    \draw[arrow] (add.east) -- (output.west);
    \draw[loraarrow] (input.east) ++(1.0,0) |- (lora_a.north);
    \draw[loraarrow] (lora_a.east) -- (lora_b.west);
    \draw[loraarrow] (lora_b.east) -- (scale.west);
    \draw[loraarrow] (scale.north) |- (add.south);

    \node[lorabox, fit=(lora_a)(lora_b)(scale)] {};
    \node[blocktitle, above=1.5cm of lora_center] {};
\end{tikzpicture}
		}
		\caption{LoRA Module.}
		\label{fig:lora_module}
	\end{subfigure}
	
	\vspace{2cm}
	
	\begin{subfigure}[b]{0.9\textwidth}
		\centering
		\resizebox{\textwidth}{!}{
			\begin{tikzpicture}[node distance=1.5cm and 2.5cm, 
	block/.style={rectangle, draw, fill=blue!10, rounded corners, minimum height=1cm, minimum width=2.5cm, align=center},
	adapter/.style={rectangle, draw, fill=orange!10, rounded corners, minimum height=1cm, minimum width=2.5cm, align=center},
	arrow/.style={->, thick, >=stealth},
	font=\sffamily
	]
	
	% NODES
	\node (input) [block] {Input Feature Map};
	\node (resnet) [block, right=of input] {Frozen ResNet \\ Bottleneck Block};
	\node (lora) [adapter, right=of resnet] {LoRA Adapter \\ {\scriptsize (Trainable)}};
	\node (output) [block, right=of lora] {Output Feature Map};
	
	% ARROWS
	\draw[arrow] (input) -- (resnet) node[midway, above, font=\scriptsize] {$\mathbf{z} \rightarrow \mathbf{z}'$};
	\draw[arrow] (resnet) -- (lora) node[midway, above, font=\scriptsize] {$\mathbf{z}' \rightarrow \mathbf{z}''$};
	\draw[arrow] (lora) -- (output) node[midway, above, font=\scriptsize] {$\mathbf{z}'' \rightarrow \mathbf{z}_{\text{out}}$};
	
	% BACKGROUND HIGHLIGHT ZONES 
	\begin{scope}[on background layer]
		\node[draw=blue!50, fill=blue!5, rounded corners, inner sep=6pt, fit=(resnet)] {};
		\node[draw=orange!60, fill=orange!5, rounded corners, inner sep=6pt, fit=(lora)] {};
	\end{scope}
	
	% LABELS 
	\node[font=\sffamily\scriptsize, text=blue!80!black, above=0.1cm of resnet.north] {Frozen};
	\node[font=\sffamily\scriptsize, text=orange!80!black, above=0.1cm of lora.north] {Trainable};
	
\end{tikzpicture}

		}
		\caption{Adapted ResNet Block.}
		\label{fig:adapted_block}
	\end{subfigure}
	
	\vspace{2cm}
	
	\begin{subfigure}[b]{0.9\textwidth}
		\centering
		\resizebox{\textwidth}{!}{
			\begin{tikzpicture}[
	node distance=1.5cm and 0.7cm,
	every node/.style={font=\scriptsize\sffamily},
	dimlabel/.style={below, font=\tiny\ttfamily\bfseries, text=black!80, yshift=-4mm},
	backbone_block/.style={rectangle, draw=blue!50!black, thick, fill=blue!6, rounded corners, minimum width=2.4cm, minimum height=0.95cm, align=center},
	lora_block/.style={rectangle, draw=orange!60!black, thick, fill=orange!22, rounded corners, minimum width=2.8cm, minimum height=1.3cm},
	stem_block/.style={rectangle, draw=purple!40!black, thick, fill=purple!12, rounded corners, minimum width=2.4cm, minimum height=0.95cm, align=center},
	trainable_block/.style={rectangle, draw=purple!90!black, thick, fill=purple!20, rounded corners, minimum width=2.4cm, minimum height=0.95cm, align=center},
	arrow/.style={-Stealth, thick, draw=black!80}
	]
	
	% --- INPUT ---
	\node[draw, thick, minimum width=1.6cm, minimum height=1.6cm] (input) 
	{\includegraphics[width=1.5cm]{DIAGRAMS/input.png}};
	\node[dimlabel] at (input.south) {512x512x3};
	
	% --- STEM ---
	\node[stem_block, right=of input] (stem) {Stem};
	\node[dimlabel] at (stem.south) {128x128x64};
	
	% --- STAGE 1 (LoRA wrapping Stage) ---
	\node[lora_block, right=of stem] (l1) {};
	\node[backbone_block, at=(l1)] (s1) {Stage 1};
	\node[dimlabel] at (l1.south) {128x128x256};
	
	% --- STAGE 2 (LoRA wrapping Stage) ---
	\node[lora_block, right=of l1] (l2) {};
	\node[backbone_block, at=(l2)] (s2) {Stage 2};
	\node[dimlabel] at (l2.south) {64x64x512};
	
	% --- STAGE 3 (LoRA wrapping Stage) ---
	\node[lora_block, right=of l2] (l3) {};
	\node[backbone_block, at=(l3)] (s3) {Stage 3};
	\node[dimlabel] at (l3.south) {32x32x1024};
	
	% --- STAGE 4 (LoRA wrapping Stage) ---
	\node[lora_block, right=of l3] (l4) {};
	\node[trainable_block, at=(l4)] (s4) {Stage 4};
	\node[dimlabel] at (l4.south) {16x16x2048};
	
	% --- GLOBAL FEATURE VECTOR ---
	\node[circle, draw, thick, fill=gray!10, below=1.4cm of l4] (gap) {GAP};
	\node[rectangle, draw, thick, fill=green!10, rounded corners,
	minimum height=1.6cm, minimum width=0.3cm, below=0.8cm of gap] (vector) {};
	\node[below=0.2cm of vector] {\scriptsize 2048-D};
	\node[left=0.6cm of vector] (out_global) {\textbf{Global Vector}};
	
	% --- MULTI-SCALE OUTPUTS ---
	\node[above=1.2cm of l1] (out_c1) {\textbf{C1}};
	\node[above=1.2cm of l2] (out_c2) {\textbf{C2}};
	\node[above=1.2cm of l3] (out_c3) {\textbf{C3}};
	\node[above=1.2cm of l4] (out_c4) {\textbf{C4}};
	\node[above=0.25cm of out_c1] {\tiny 128x128x256};
	\node[above=0.25cm of out_c2] {\tiny 64x64x512};
	\node[above=0.25cm of out_c3] {\tiny 32x32x1024};
	\node[above=0.25cm of out_c4] {\tiny 16x16x2048};
	
	% --- ARROWS (main flow) ---
	\draw[arrow] (input) -- (stem);
	\draw[arrow] (stem) -- (l1);
	\draw[arrow] (l1) -- (l2);
	\draw[arrow] (l2) -- (l3);
	\draw[arrow] (l3) -- (l4);
	\draw[arrow] (l4) -- (gap);
	\draw[arrow] (gap) -- (vector);
	\draw[arrow] (vector.west) -- (out_global.east);
	
	% --- ARROWS (multi-scale outputs) ---
	\draw[arrow] (l1.north) -- (out_c1.south);
	\draw[arrow] (l2.north) -- (out_c2.south);
	\draw[arrow] (l3.north) -- (out_c3.south);
	\draw[arrow] (l4.north) -- (out_c4.south);
	
	% --- BACKGROUND PATH (Blue frozen backbone path) ---
	\begin{scope}[on background layer]
		\node[draw=blue!50, fill=blue!5, rounded corners, inner sep=8pt,
		fit=(stem) (l1) (l2) (l3) (l4)] {};
	\end{scope}
	
\end{tikzpicture}
		}
		\caption{Adapted ResNet50 Backbone.}
		\label{fig:adapted_backbone}
	\end{subfigure}
	
	\caption{Architectural details of the shared backbone, illustrating (a) the core LoRA module, (b) its integration into a standard ResNet block, and (c) the overall adapted backbone structure.}
	\label{fig:lora_diagrams}
\end{figure}

\subsection{Task-Specific Heads}
\label{ssec:task_specific_decoders}
The shared backbone produces multi-scale features, which are then directed into specialized decoder heads, each designed to solve a specific diagnostic task.

\begin{figure}[htbp]
	\centering
	\resizebox{0.5\textwidth}{!}{
		
    \begin{tikzpicture}[
        node distance=1.2cm and 0.8cm,
        block/.style={rectangle, draw, thick, fill=gray!20, rounded corners, minimum width=4cm, minimum height=0.9cm, align=center, font=\sffamily\small},
        trainable_block/.style={block, fill=violet!25, text=black!80},
        op/.style={rectangle, draw, thick, fill=violet!15, rounded corners, font=\sffamily\tiny, minimum height=0.55cm},
        arrow/.style={-Stealth, thick, draw=black!70, rounded corners=3pt},
        feat_map/.style={cuboid, cuboidfill, minimum width=2.5cm, minimum height=0.55cm, font=\sffamily\tiny, fill=orange!30},
        pyramid_map/.style={cuboid, cuboidfill, minimum width=2.2cm, minimum height=0.5cm, font=\sffamily\tiny, fill=cyan!20},
        out/.style={rectangle, draw, thick, fill=green!25, rounded corners, align=center, font=\sffamily\tiny, minimum width=2.8cm},
        % Custom arrows
        pyramid_arrow/.style={-Stealth, thick, draw=cyan!70, rounded corners=2pt, line width=1pt},
        proposal_arrow/.style={-Stealth, thick, draw=red!70!black, dashed, rounded corners=2pt, line width=1pt, font=\sffamily\tiny, red},
        dim_label_right/.style={midway, right, font=\tiny\ttfamily, text=black!60, xshift=2mm},
        dim_label_left/.style={midway, left, font=\tiny\ttfamily, text=black!60, xshift=-2mm}
    ]
    
    % TITLE
    \node[blocktitle] (title) {};
    
    % INPUTS
    \node[feat_map] (c3) {C3\\32×32×1024};
    \node[feat_map, left=0.15cm of c3] (c4) {C4\\16×16×2048};
    \node[feat_map, right=0.15cm of c3] (c2) {C2\\64×64×512};

    % FPN 
    \node[trainable_block, below=1.2cm of c3] (fpn) {Feature Pyramid Network (FPN)};

    % Pyramid Maps (P2–P4) 
    \node[pyramid_map] (p3) [below=1.4cm of fpn] {P3\\32×32×64};
    \node[pyramid_map, left=0.15cm of p3] (p4) {P4\\16×16×64};
    \node[pyramid_map, right=0.15cm of p3] (p2) {P2\\64×64×64};

    % Pyramid zone
    \begin{scope}[on background layer]
        \node[draw=cyan!70, fill=cyan!10, rounded corners, inner sep=4pt,
              fit=(p2) (p3) (p4),
              label={[font=\sffamily\bfseries\tiny, text=cyan!80!black]135:Feature Pyramid Maps (P2–P4)}] {};
    \end{scope}
    
    % RPN & RoIAlign
    \node[trainable_block, below=2cm of p3] (rpn) {Region Proposal Network};
    \node[block, right=1.2cm of rpn] (roi_align) {MultiScaleRoIAlign};

    % Lightweight Box Head
    \node[op, below=1.5cm of rpn, fill=violet!15] (head_conv) {Conv 3×3, BN, ReLU};
    \node[op, below=0.4cm of head_conv] (head_ds_conv) {Depthwise-Separable Conv};
    \node[op, below=0.4cm of head_ds_conv, fill=gray!20] (head_flatten) {Flatten};
    \node[op, below=0.4cm of head_flatten] (head_fc) {FC Layer};
    \node[op, below=0.4cm of head_fc] (predictor) {Predictor FC Layer};
    
    % Outputs
    \node[outnode, below left=0.7cm and 0.2cm of predictor] (class_out) {Class Scores};
    \node[outnode, below right=0.7cm and 0.2cm of predictor] (bbox_out) {BBox Offsets};

    % Connections
    % Backbone -> FPN
    \draw[arrow] (c2.south) -- (fpn.north);
    \draw[arrow] (c3.south) -- (fpn.north);
    \draw[arrow] (c4.south) -- (fpn.north);
    
    % FPN -> Pyramid Maps
    \draw[pyramid_arrow] (fpn.south) -- ++(0,-0.4) -| (p2.north);
    \draw[pyramid_arrow] (fpn.south) -- (p3.north);
    \draw[pyramid_arrow] (fpn.south) -- ++(0,-0.4) -| (p4.north);

    % Pyramid -> bus -> RPN & RoIAlign
    \coordinate (bus_top) at ($(p4.south)!0.5!(p2.south)$);
    \coordinate (bus_bottom) at ($(bus_top)+(0,-0.5cm)$);
    \draw[pyramid_arrow, line width=1.2pt] (bus_top) -- (bus_bottom);
    \draw[pyramid_arrow] (p2.south) -- ++(0,-0.15) -| (bus_top);
    \draw[pyramid_arrow] (p3.south) -- ++(0,-0.15) -| (bus_top);
    \draw[pyramid_arrow] (p4.south) -- ++(0,-0.15) -| (bus_top);
    \draw[pyramid_arrow] (bus_bottom) -- (rpn.north);
    \draw[pyramid_arrow] (bus_bottom) -- (roi_align.north);

    % RPN -> RoIAlign (Proposals)
    \draw[proposal_arrow] (rpn.east) -- node[midway, above, font=\sffamily\tiny] {Proposals} (roi_align.west);
    
    % RoIAlign -> Box Head 
    \draw[arrow] (roi_align.south) -- ++(0,-0.4) -| node[dim_label_right] {N×64×7×7} (head_conv.north);
    \draw[arrow] (head_conv) -- node[dim_label_left] {N×256×7×7} (head_ds_conv);
    \draw[arrow] (head_ds_conv) -- node[dim_label_left] {N×64×7×7} (head_flatten);
    \draw[arrow] (head_flatten) -- node[dim_label_left] {N×3136} (head_fc);
    \draw[arrow] (head_fc) -- node[dim_label_left] {N×256} (predictor);

    % Final Outputs
    \draw[arrow] (predictor.south) -- ++(0,-0.3) -| (class_out.north);
    \draw[arrow] (predictor.south) -- ++(0,-0.3) -| (bbox_out.north);

    \begin{scope}[on background layer]
        \node[draw=violet!60, fill=violet!10, rounded corners, inner sep=6pt,
              fit=(head_conv) (head_ds_conv) (head_flatten) (head_fc) (predictor),
              label={[font=\sffamily\bfseries\tiny, text=violet!80!black]135:Lightweight Box Head}] {};
    \end{scope}

\end{tikzpicture}

	}
	\caption{Custom Faster R-CNN Detection Head Architecture.}
	\label{fig:rcnn_head_diagram}
\end{figure}

\begin{figure}[htbp]
	\centering
	\resizebox{0.5\textwidth}{!}{
		\begin{tikzpicture}[
    node distance=1.2cm and 1cm,
    block/.style={rectangle, draw, thick, fill=blue!15, rounded corners, minimum width=2.4cm, minimum height=0.7cm, align=center, font=\sffamily\small},
    feat_map/.style={cuboid, cuboidfill, minimum width=2.6cm, minimum height=0.55cm, font=\sffamily\tiny, fill=orange!30},
    op/.style={circle, draw, thick, fill=cyan!20, minimum size=0.7cm, font=\sffamily\tiny},
    out_map/.style={cuboid, fill=green!25, minimum width=1.8cm, minimum height=0.45cm, font=\sffamily\tiny},
    arrow/.style={-Stealth, thick, draw=black!70},
    skip_arrow/.style={-Stealth, thick, draw=cyan!60!black, dashed},
    dim_label_above/.style={midway, above, font=\tiny\ttfamily, text=black!60, yshift=1.5mm},
    % ADJUSTMENT: Changed 'midway' to 'pos=0.8' to move labels closer to the end of the arrow
    dim_label_right/.style={pos=0.2, right, font=\tiny\ttfamily, text=black!60, xshift=2mm},
    dim_label_on_arrow/.style={pos=0.5, above, font=\tiny\ttfamily, text=black!60}
]
    
    % --- TITLE ---
    \node[blocktitle, yshift=-0.5cm] (title) {};

    % --- INPUTS (Encoder Path) ---
    \node[feat_map, below=1.2cm of title] (c4_in) {C4\\16×16×2048};
    \node[feat_map, above=of c4_in] (c3_in) {C3\\32×32×1024};
    \node[feat_map, above=of c3_in] (c2_in) {C2\\64×64×512};
    \node[feat_map, above=of c2_in] (c1_in) {C1\\128×128×256};

    % --- DECODER PATH ---
    \node[block, right=4.5cm of c4_in] (d4) {Up-Conv + Fuse Block}; 
    \node[op, right=1.5cm of c4_in] (concat4) {+};

    \node[block, above=of d4] (d3) {Up-Conv + Fuse Block};
    \node[op, right=1.5cm of c3_in] (concat3) {+};

    \node[block, above=of d3] (d2) {Up-Conv + Fuse Block};
    \node[op, right=1.5cm of c2_in] (concat2) {+};
    
    \node[block, above=of d2, fill=violet!20] (final_conv) {Final Conv 1x1};
    \node[out_map, above=of final_conv] (output) {Mask Logits};

    % --- ARROWS (curved + straight) ---
    \draw[arrow] (c4_in) -- node[dim_label_above] {16x16x2048} (concat4);
    \draw[skip_arrow] (c3_in) -- node[dim_label_above] {32x32x1024} (concat4);
    \draw[arrow, bend right=30, looseness=1.5] (concat4.south east) to node[dim_label_on_arrow] {32x32x(1024+64)} (d4.west);

    % The labels on these two arrows will now be closer to the concat nodes
    \draw[arrow] (d4) -- node[dim_label_right] {32x32x64} (concat3);
    \draw[skip_arrow] (c2_in) -- node[dim_label_above] {64x64x512} (concat3);
    \draw[arrow, bend right=30, looseness=1.5] (concat3.south east) to node[dim_label_on_arrow] {64x64x(512+32)} (d3.west);

    \draw[arrow] (d3) -- node[dim_label_right] {64x64x32} (concat2);
    \draw[skip_arrow] (c1_in) -- node[dim_label_above] {128x128x256} (concat2);
    \draw[arrow, bend right=30, looseness=1.5] (concat2.south east) to node[dim_label_on_arrow] {128x128x(256+16)} (d2.west);

    \draw[arrow] (d2) -- node[dim_label_right] {128x128x16} (final_conv);
    \draw[arrow] (final_conv) -- node[dim_label_right, pos=0.5] {64x64x1} (output); % Kept this one centered

    % --- BACKGROUND GROUPING (with zone labels) ---
    \begin{scope}[on background layer]
        \node[draw=orange!50, fill=orange!5, rounded corners, inner sep=6pt,
              fit=(c1_in) (c4_in),
              label={[font=\sffamily\bfseries\tiny, text=orange!80!black, yshift=0.2cm]90:Backbone Features}] {};
        \node[draw=blue!50, fill=blue!5, rounded corners, inner sep=6pt,
              fit=(d2) (d3) (d4) (final_conv) (output) (concat2) (concat3) (concat4),
              label={[font=\sffamily\bfseries\tiny, text=blue!80!black, yshift=0.2cm]90:Decoder Path}] {};
    \end{scope}

\end{tikzpicture}
	}
	\caption{Segmentation Head Architecture.}
	\label{fig:seg_head_diagram}
\end{figure}

\begin{figure}[htbp]
	\centering
	\resizebox{0.3\textwidth}{!}{
		\begin{tikzpicture}[
    node distance=1.3cm and 1.5cm,
    block/.style={rectangle, draw, thick, fill=violet!20, rounded corners, minimum width=3cm, minimum height=0.8cm, align=center, font=\sffamily\small},
    feat_map/.style={cuboid, cuboidfill, minimum width=2.5cm, minimum height=0.55cm, font=\sffamily\tiny, fill=orange!30},
    op/.style={circle, draw, thick, fill=gray!10, minimum size=0.8cm, font=\sffamily\small},
    out_map/.style={cuboid, fill=green!25, minimum width=1.5cm, minimum height=0.4cm, font=\sffamily\tiny},
    arrow/.style={-Stealth, thick, draw=black!70}, % Sharp lines
    dim_label_right/.style={midway, right, font=\tiny\ttfamily, text=black!60, xshift=2mm}
]
    
    % --- TITLE ---
    \node[blocktitle] (title) {};
    
    % --- INPUTS ---
    \node[feat_map, below=0.8cm of title] (c1) {C1\\128×128×256};
    \node[feat_map, right=3cm of c1] (c2) {C2\\64×64×512};

    % --- PROJECTION & FUSION PATH ---
    \node[block, below=of c1] (proj1) {Conv 1x1};
    \node[block, below=of c2] (proj2) {Conv 1x1};
    \node[block, below=of proj2] (upsample) {Upsample (Bilinear)};
    
    \node[op, shape=cylinder, shape border rotate=90, minimum height=0.9cm, fill=cyan!20, below left=0.8cm and 0.5cm of upsample] (concat) {Concat};

    % --- ATTENTION PATH ---
    \node[block, below=1.2cm of concat, fill=yellow!25] (attention) {Spatial Attention};
    \node[op, right=1.2cm of attention, fill=yellow!30] (multiply) {$\times$};

    % --- DECODER PATH (with uniform spacing) ---
    \newcommand{\decoderVspace}{0.8cm} 
    
    \node[block, below=1.2cm of multiply, fill=blue!15] (sep_conv1) {Separable Conv Block};
    \node[block, below=\decoderVspace of sep_conv1, fill=blue!15] (sep_conv2) {Separable Conv Block};
    \node[block, below=\decoderVspace of sep_conv2, fill=blue!15] (final_conv) {Conv 1x1};
    \node[block, below=\decoderVspace of final_conv, fill=blue!15] (relu) {ReLU};
    \node[block, below=\decoderVspace of relu, fill=green!20] (sigmoid) {Sigmoid};

    % --- OUTPUT ---
    \node[out_map, below=\decoderVspace of sigmoid] (output) {Heatmap};
    
    % --- ARROWS & DIMENSIONS ---
    \draw[arrow] (c1) -- node[dim_label_right] {128×128×256} (proj1);
    \draw[arrow] (c2) -- node[dim_label_right] {64×64×512} (proj2);
    
    \draw[arrow] (proj1.south) -- ++(0,-0.4) -| node[dim_label_right] {128×128×64} (concat.west);
    \draw[arrow] (proj2.south) -- node[dim_label_right] {64×64×64} (upsample);
    \draw[arrow] (upsample.south) -- ++(0,-0.4) -| node[dim_label_right] {128×128×64} (concat.east);
    
    \draw[arrow] (concat.south) -- node[dim_label_right] {128×128×128} (attention.north);
    \draw[arrow] (concat.south) -- ++(0,-0.6) -| (multiply.west);
    \draw[arrow, draw=yellow!60!black] (attention.east) -- node[midway, above, font=\sffamily\tiny] {Att Map} (multiply);
    
    \draw[arrow] (multiply.south) -- node[dim_label_right] {128×128×128} (sep_conv1.north);
    \draw[arrow] (sep_conv1.south) -- node[dim_label_right] {128×128×64} (sep_conv2.north);
    \draw[arrow] (sep_conv2.south) -- node[dim_label_right] {128×128×32} (final_conv.north);
    \draw[arrow] (final_conv.south) -- node[dim_label_right] {128×128×1} (relu.north);
    \draw[arrow] (relu.south) -- (sigmoid.north);
    \draw[arrow] (sigmoid.south) -- node[dim_label_right] {64×64×1} (output.north);
    
    % --- ZONES & GROUPING ---
    \begin{scope}[on background layer]
        % Group for Decoder Path
        \node[draw=blue!50, fill=blue!5, rounded corners, inner sep=6pt,
              fit=(sep_conv1) (sep_conv2) (final_conv) (relu),
              label={[font=\sffamily\bfseries\tiny, text=blue!80!black]135:Decoder Path}] {};

        % Group for C1 Path
        \node[draw=violet!50, fill=violet!5, rounded corners, inner sep=8pt,
              fit=(c1) (proj1),
              label={[font=\sffamily\bfseries\tiny, text=violet!80!black]135:C1 Path}] {};
        
        % Group for C2 Path
        \node[draw=violet!50, fill=violet!5, rounded corners, inner sep=8pt,
              fit=(c2) (proj2) (upsample),
              label={[font=\sffamily\bfseries\tiny, text=violet!80!black]135:C2 Path}] {};
    \end{scope}
    
\end{tikzpicture}
	}
	\caption{Infection Localization Head Architecture.}
	\label{fig:heatmap_head_diagram}
\end{figure}

\begin{figure}[htbp]
	\centering
	\resizebox{0.5\textwidth}{!}{
		\begin{minipage}{0.48\textwidth}
    \centering
    \begin{tikzpicture}[
        node distance=0.7cm,
        block/.style={rectangle, draw, thick, fill=violet!20, rounded corners, minimum width=3.5cm, minimum height=0.8cm, align=center, font=\sffamily\small},
        feat_vec/.style={rectangle, draw, thick, fill=orange!30, rounded corners, minimum height=1.5cm, minimum width=0.5cm},
        out_vec/.style={feat_vec, fill=green!25},
        arrow/.style={-Stealth, thick, draw=black!70},
        dim_label_right/.style={midway, right, font=\tiny\ttfamily\bfseries, text=black!60, xshift=2mm}
    ]

        % Title
        \node[blocktitle] (title) {};

        % Input
        \node[feat_vec, below=1cm of title] (input_vec) {};
        \node[above=0.1cm of input_vec, font=\sffamily\small] {Global Feature Vector};

        % MLP Layers
        \node[block, below=of input_vec] (fc1) {Linear 2048→256};
        \node[block, below=of fc1, fill=blue!15] (norm1) {LayerNorm};
        \node[block, below=of norm1, fill=blue!15] (relu1) {ReLU};
        \node[block, below=of relu1, fill=gray!20] (dropout1) {Dropout 0.25};
        \node[block, below=of dropout1] (fc2) {Linear 256→128};
        \node[block, below=of fc2, fill=blue!15] (norm2) {LayerNorm};
        \node[block, below=of norm2, fill=blue!15] (relu2) {ReLU};
        \node[block, below=of relu2, fill=gray!20] (dropout2) {Dropout 0.25};
        \node[block, below=of dropout2] (fc3) {Linear 128→N};

        % Output
        \node[out_vec, below=of fc3] (output_vec) {};
        \node[above=0.1cm of output_vec, font=\sffamily\small] {Logits};

        % Arrows with full dimensions
        \draw[arrow] (input_vec) -- node[dim_label_right] {B × 2048} (fc1);
        \draw[arrow] (fc1) -- node[dim_label_right] {B × 256} (norm1);
        \draw[arrow] (norm1) -- node[dim_label_right] {B × 256} (relu1);
        \draw[arrow] (relu1) -- node[dim_label_right] {B × 256} (dropout1);
        \draw[arrow] (dropout1) -- node[dim_label_right] {B × 128} (fc2);
        \draw[arrow] (fc2) -- node[dim_label_right] {B × 128} (norm2);
        \draw[arrow] (norm2) -- node[dim_label_right] {B × 128} (relu2);
        \draw[arrow] (relu2) -- node[dim_label_right] {B × 128} (dropout2);
        \draw[arrow] (dropout2) -- node[dim_label_right] {B × N} (fc3);
        \draw[arrow] (fc3) -- node[dim_label_right] {B × N} (output_vec);

    \end{tikzpicture}
\end{minipage}%
\hfill
% --- RIGHT: RoI Classification Head ---
\begin{minipage}{0.48\textwidth}
    \centering
    \begin{tikzpicture}[
        node distance=0.7cm,
        block/.style={rectangle, draw, thick, fill=violet!20, rounded corners, minimum width=3.5cm, minimum height=0.8cm, align=center, font=\sffamily\small},
        feat_vec/.style={rectangle, draw, thick, fill=orange!30, rounded corners, minimum height=1cm, minimum width=1cm},
        out_vec/.style={feat_vec, fill=green!25},
        arrow/.style={-Stealth, thick, draw=black!70},
        proposal_arrow/.style={-Stealth, thick, draw=red!70!black, dashed, font=\sffamily\tiny},
        dim_label_right/.style={midway, right, font=\tiny\ttfamily\bfseries, text=black!60, xshift=2mm}
    ]

        % Title
        \node[blocktitle] (title) {};

        % Inputs
        \node[feat_vec, below=0.8cm of title] (p2) {P2};
        \node[feat_vec, right=0.4cm of p2] (p3) {P3};
        \node[feat_vec, right=0.4cm of p3] (p4) {P4};
        \node[above=0.2cm of p3, font=\sffamily\small] {FPN Feature Maps};

        % Proposals
        \node[rectangle, draw, thick, dashed, draw=red!70!black, rounded corners, below=0.6cm of p3] (proposals) {Region Proposals};

        % RoI Pooling
        \node[block, fill=blue!15, below=0.8cm of proposals] (roi_align) {MultiScaleRoIAlign};
        \node[block, fill=gray!20, below=of roi_align] (flatten) {Flatten};

        % MLP
        \node[block, below=of flatten] (mlp1) {Linear 3136→256};
        \node[block, below=of mlp1, fill=blue!15] (norm1) {BatchNorm};
        \node[block, below=of norm1, fill=blue!15] (relu1) {ReLU};
        \node[block, below=of relu1, fill=gray!20] (dropout1) {Dropout 0.4};
        \node[block, below=of dropout1] (mlp2) {Linear 256→128};
        \node[block, below=of mlp2, fill=blue!15] (norm2) {BatchNorm};
        \node[block, below=of norm2, fill=blue!15] (relu2) {ReLU};
        \node[block, below=of relu2, fill=gray!20] (dropout2) {Dropout 0.3};
        \node[block, below=of dropout2] (mlp3) {Linear 128→N};

        % Output
        \node[out_vec, below=of mlp3] (output_vec) {};
        \node[above=0.1cm of output_vec, font=\sffamily\small] {Logits};

        % Arrows with full dimensions
        \draw[arrow] (p2) -- node[dim_label_right] {} (roi_align);
        \draw[arrow] (p3) -- node[dim_label_right] {} (roi_align);
        \draw[arrow] (p4) -- node[dim_label_right] {} (roi_align);
        \draw[proposal_arrow] (proposals) -- node[dim_label_right] {RoIs × 4 (x1,y1,x2,y2)} (roi_align);
        \draw[arrow] (roi_align) -- node[dim_label_right] {RoIs × 64 × 7 × 7} (flatten);
        \draw[arrow] (flatten) -- node[dim_label_right] {RoIs × 3136} (mlp1);
        \draw[arrow] (mlp1) -- node[dim_label_right] {RoIs × 256} (norm1);
        \draw[arrow] (norm1) -- node[dim_label_right] {RoIs × 256} (relu1);
        \draw[arrow] (relu1) -- node[dim_label_right] {RoIs × 256} (dropout1);
        \draw[arrow] (dropout1) -- node[dim_label_right] {RoIs × 128} (mlp2);
        \draw[arrow] (mlp2) -- node[dim_label_right] {RoIs × 128} (norm2);
        \draw[arrow] (norm2) -- node[dim_label_right] {RoIs × 128} (relu2);
        \draw[arrow] (relu2) -- node[dim_label_right] {RoIs × 128} (dropout2);
        \draw[arrow] (dropout2) -- node[dim_label_right] {RoIs × N} (mlp3);
        \draw[arrow] (mlp3) -- node[dim_label_right] {RoIs × N} (output_vec);

    \end{tikzpicture}
\end{minipage}

	}
	\caption{MLP Classification Head Architecture.}
	\label{fig:clf_head_diagram}
\end{figure}

\paragraph{\acf{FPN}.} To provide the detection and RoI heads with rich multi-scale information, we pass the backbone features through an \acf{FPN} \parencite{Lin2017FeaturePyramidNetworks}. This structure mixes coarse and fine features via a top-down pathway, ensuring the network operates with feature maps that are both semantically rich and spatially detailed.

\paragraph{Detection Head.}
For the primary detection task ($\mathcal{T}_{\text{det}}$), we utilize a customized Faster R-CNN head (Figure~\ref{fig:rcnn_head_diagram}) built on the FPN outputs. It comprises the \acf{RPN}, Multi-Scale RoI Align, and a box head. We replaced the standard, heavy MLP box head with a lighter design using separable convolutions. Early tests indicated the default head was prone to overfitting and computationally expensive during multi-task training. This lightweight version reduces parameters and maintain sufficient capacity to model fine cell structures.

\paragraph{RoI Classification Head.}
We initially explored two classification strategies. The first, cell-level classification, processed pre-cropped $64 \times 64$ cells. However, early experiments showed this caused conflicting gradients with the detection task. We therefore discarded it in favor of \acf{RoI} classification ($\mathcal{T}_{\text{roi}}$). This approach extracts features from ground-truth boxes on the full image using RoI Align and feeds them into the shared \texttt{clf\_head} (Figure~\ref{fig:clf_head_diagram}). This method proved synergistic, encouraging clean class separation on well-localized regions without disrupting the detection objective.

\paragraph{Segmentation Head.} 
The segmentation head ($\mathcal{T}_{\text{seg}}$) employs a lightweight decoder (Figure~\ref{fig:seg_head_diagram}) inspired by U-Net \parencite{RonnebergerFB15}. It aggregates features from multiple backbone stages and uses upsampling with skip connections to reconstruct a full-resolution mask.

\paragraph{Infection Localization Head.} 
The localization head ($\mathcal{T}_{\text{loc}}$) is an attention-based decoder that generates a heatmap of infection centers (Figure~\ref{fig:heatmap_head_diagram}). It fuses high-resolution features with a spatial attention block and uses depth-wise separable convolutions to produce the final map.

\medskip
\noindent \textit{Note: The architectural diagrams referenced in this section were created specifically to illustrate the exact implementation used in this work.}

\section{Experimental Protocol}
\label{sec:experimental_protocol}
To rigorously test whether MTTL yields a more robust diagnostic system than STL, we designed a two-phase experimental protocol. This structure isolates key variables to ensure our insights on tuning, data handling, and task interactions are reproducible. All models were trained using the AdamW optimizer \parencite{loshchilov2019decoupled} with a cosine annealing scheduler and linear warm-up.

\subsection{Phase 1: Establishing Optimal STL Baselines}
\label{ssec:exp_phase1}

The goal of this first phase is to identify the strongest possible \acf{STL} baseline models. We investigate two main design choices:

\begin{itemize}
	\item \textbf{Tuning Strategy:} We compare \textbf{LoRA-Only} against \textbf{Hybrid Tuning} to determine the best method for knowledge transfer.
	\item \textbf{Data Strategy:} We assess how best to handle class imbalance by contrasting a \textbf{Class-Balanced Sampler} with a standard sampler.
\end{itemize}

The best performing configuration from these tests is then fixed as the foundation for all subsequent experiments.

\subsection{Phase 2: Evaluating Diagnostic Completeness and MTTL Synergy}
\label{ssec:exp_phase2}

The second phase addresses our research question that \textit{Does an \ac{MTTL} approach provide superior efficacy and robustness compared to \ac{STL} ?} We break this down into two analyses.

\paragraph{(1) Task Complexity.}  
We progressively increase the number of detection classes to stress-test the STL paradigm. We start with \textbf{1-Class:} to detect only \emph{Infected} cells, then \textbf{2-Class:} to detect both \emph{Infected} and \emph{Healthy} cells, and finally \textbf{3-Class:} to detect \emph{Infected}, \emph{Healthy}, and \emph{WBC} cells.

\paragraph{(2) MTTL Synergy.}
We then test whether auxiliary tasks can overcome STL limitations by providing additional inductive signals. We evaluate various combinations:

\begin{itemize}
	\item \textbf{Pairwise:} Adding Segmentation (shape prior), Localization (attention prior), or RoI Classification (feature refinement) to the detector.
	\item \textbf{Triple:} Combining two auxiliary tasks with detection (e.g., Segmentation and Localization) to merge their benefits.
	\item \textbf{Full Combination:} Integrating all four tasks (Detection, Segmentation, Localization, RoI Classification) to create the most complete MTTL setup.
\end{itemize}

\section{Evaluation Protocol and Metrics}
\label{sec:evaluation_protocol}

Our evaluation protocol is designed to ensure fair comparison, clinical interpretability, and deep diagnostic analysis beyond simple aggregate scores.

\subsection{Diagnostic Test Datasets}
\label{ssec:diagnostic_datasets}
We evaluate each model on three distinct, patient-disjoint test sets to profile its capabilities:

\begin{enumerate}
	\item \textbf{The Full Test Set:} Reflects the natural distribution of the original data and serves as the primary benchmark for real-world performance.
	\item \textbf{The Infected-Only Test Set:} Contains only images from infected patients. This target-rich environment rigorously evaluates \textbf{sensitivity and recall}.
	\item \textbf{The Healthy-Only Test Set:} Contains only images from healthy patients. This pure negative dataset evaluates \textbf{specificity and precision}, measuring the ability to avoid false positives.
	\item \textbf{Parasitemia Quantification:} We also assess the ability to quantify disease severity on the \textbf{Full Test Set}. For each slide, parasitemia ($\mathcal{P}$) is estimated as:
	\begin{equation}
		\mathcal{P} = \frac{N_{\text{infected}}}{N_{\text{infected}} + N_{\text{healthy}}} \times 100\%
		\label{eq:parsitemia_formula}
	\end{equation}
	where $N$ represents the predicted cell counts. Accuracy is measured against ground truth using \textbf{Mean Absolute Error (MAE)} and the \textbf{Pearson Correlation Coefficient (r)}.
\end{enumerate}

\subsection{Task-Specific Performance Metrics}
\label{ssec:performance_metrics}

\paragraph{Detection.}
The primary metric is \textbf{\ac{mAP}} at an \ac{IoU} threshold of 0.5 (mAP@50), following PASCAL VOC standards \parencite{Everingham2010PascalVOC}. We also report class-specific \textbf{Precision}, \textbf{Recall}, and \textbf{F1-Score} at an optimal confidence threshold derived from the Precision-Recall curve \parencite{Davis2006PrecisionRecall}. The F1-Score for the \textit{Infected class} is a key performance indicator.

\paragraph{RoI Classification.}
We use the \textbf{F1-Macro} score, which is robust to class imbalance, along with overall \textbf{Accuracy}.

\paragraph{Segmentation.}
We report the \textbf{Dice Similarity Coefficient} \parencite{Dice1945Measures} and the \textbf{Intersection over Union (IoU)} on binarized output masks.

\paragraph{Infection Localization.}
This heatmap task is evaluated using the \textbf{Dice Score}, \textbf{Pearson Correlation}, and \textbf{\ac{MAE}}. On pure-negative datasets where Dice is undefined, we report the \textbf{Average Prediction Value} to measure false positive activation.

\begin{table}[htbp]
	\centering
	\caption{Summary of Evaluation Metrics for Each Task.}
	\label{tab:evaluation_metrics}
	\renewcommand{\arraystretch}{1.5}
	\begin{tabular}{p{3cm} p{10cm}}
		\toprule
		\textbf{Task} & \textbf{Primary Evaluation Metrics} \\
		\midrule
		$\mathcal{T}_{\text{det}}$ & mAP@50, Precision, Recall, F1-Score (per-class) \\
		$\mathcal{T}_{\text{loc}}$ & Dice Score, Pearson Correlation, \acs{MAE}, \acs{MSE} \\
		$\mathcal{T}_{\text{seg}}$ & Dice Score, Intersection over Union (IoU) \\
		$\mathcal{T}_{\text{roi}}$ & Accuracy, F1-Macro, F1-Score (per-class) \\
		\bottomrule
	\end{tabular}
\end{table}

\section{Implementation Details and Experimental Setup}
\label{sec:implementation_details}

All experiments were conducted on the Kaggle platform (\url{https://www.kaggle.com}), utilizing its cloud-based computational resources. This section provides a detailed account of the hardware, software, and training configurations employed, ensuring that our experiments are reproducible and systematically controlled.

\subsection{Hardware and Software Environment}
\label{ssec:hardware_software}

Training and evaluation of all models were performed on a Kaggle Notebook instance with the following specifications:

\begin{itemize}
	\item \textbf{Hardware:}
	\begin{itemize}
		\item \textbf{GPU:} Dual NVIDIA Tesla T4 GPUs, each with 16 GB of VRAM.
		\item \textbf{CPU:} Intel Xeon processors, with 4–8 cores.
		\item \textbf{RAM:} 30 GB of system memory.
	\end{itemize}
	\item \textbf{Software Environment:}
	\begin{itemize}
		\item \textbf{Operating System:} Debian-based Linux distribution provided by Kaggle.
		\item \textbf{Programming Language:} Python (Version 3.11.13).
		\item \textbf{Deep Learning Framework:} PyTorch (Version 2.6.0+cu124), with CUDA support on 2 NVIDIA Tesla T4 GPUs
		\item \textbf{Key Scientific Libraries:} NumPy for numerical operations, Pandas for data handling, Scikit-learn for additional metric calculations, and Matplotlib, Seaborn, and Plotly for visualization. A complete list of package versions is available in the project’s dependency manifest.
	\end{itemize}
\end{itemize}

\subsection{Training Configuration}
\label{ssec:training_configuration}

To ensure fair and controlled comparisons, all experiments followed a consistent training protocol, unless specific variables were intentionally altered.

\begin{itemize}
	\item \textbf{Random Seed and Reproducibility:} A fixed seed of 12 was used for all experiments to ensure reproducibility. This seed was applied to Python's \texttt{random} module, NumPy, and PyTorch (including CUDA operations and worker initialization). CUDNN deterministic settings were enforced and benchmark mode disabled to guarantee consistent results across runs.
	
	\item \textbf{Optimizer:} All experiments used the \textbf{AdamW} optimizer \parencite{loshchilov2019decoupled}, preferred over standard Adam for its improved weight decay, which often leads to better generalization.
	\item \textbf{Learning Rate Schedule:} A \textbf{linear warm-up followed by cosine annealing} schedule was applied. The learning rate gradually increased from a low starting value to its target over the warm-up epochs, then decayed following a cosine curve. The primary learning rate for heads and adapters was set to $1 \times 10^{-3}$.
	
	\item \textbf{Discriminative Fine-Tuning:} For Hybrid Tuning experiments, backbone layers were unfrozen with a significantly lower learning rate to avoid catastrophic forgetting. Preliminary experiments indicated an optimal backbone learning rate of $\mathbf{1 \times 10^{-6}}$, maintaining a 1000:1 ratio with the head learning rate.
	
	\item \textbf{Batch Size and Gradient Accumulation:} Due to GPU memory limitations, MTTL models used a base batch size of 8. To simulate an effective batch size of 16 (comparable to STL experiments), \textbf{gradient accumulation} over 2 steps was applied.
	
	\item \textbf{\acf{AMP}:} PyTorch’s \textbf{AMP} was employed to accelerate training and reduce memory usage. AMP leverages 16-bit floating-point precision for selected operations while preserving 32-bit precision where necessary, ensuring numerical stability with faster computation.
	
	\item \textbf{Early Stopping:} To prevent overfitting and reduce computation, training was halted when the primary validation metric (e.g., mAP@50 for detection) showed no improvement for a predefined number of epochs.
\end{itemize}

\noindent With these settings, we ensure that our training process is consistent, reliable, and directly supports the experimental results that follow.
\cleardoublepage

%RESULTS
\chapter{Results and Discussion}
\label{chap:results}

This chapter presents the empirical results of our Multi-Task Transfer Learning (MTTL) study for malaria detection. We first establish strong Single-Task Learning (STL) baselines and track how their performance changes as the diagnostic setup becomes more complex. We then report the results of our MTTL models, analyze the role of each auxiliary task, and evaluate the best configurations through quantitative metrics and visual examples. Extended results and full training details are provided in the Appendices.

\section{Single-Task Learning Detection Baselines}
\label{sec:optimal_stl_recipe}

To build reliable STL baselines, we evaluated detection in 1, 2, and 3 classes using four fine-tuning setups: LoRA-Only and Hybrid Tuning, each tested with and without a sampler. This provides a solid reference point for comparing STL performance with our MTTL approach.

\begin{table}[htbp]
	\centering
	\caption{1-Class Detection Baselines (Ranked by F1(Inf)) }
	\label{tab:stl_1class}
	\begin{tabular}{lcccccc}
		\toprule
		\textbf{Run} & \textbf{mAP@50} & \textbf{mAP@75} & \textbf{F1(Inf)} & \textbf{R(Inf)} & \textbf{P(Inf)} \\
		\midrule
		STL\_Det\_1cls\_Hybrid\_NoSampler & \textbf{0.8260} & 0.5020 & \textbf{0.8182} & \textbf{0.8146} & \textbf{0.8218} \\
		STL\_Det\_1cls\_LoRA\_WithSampler & 0.8012 & 0.4930 & 0.8007 & 0.7991 & 0.8023 \\
		STL\_Det\_1cls\_Hybrid\_WithSampler & 0.8201 & 0.4865 & 0.7994 & 0.8011 & 0.7978 \\
		\bottomrule
	\end{tabular}
\end{table}

\begin{table}[htbp]
	\centering
	\caption{2-Class Detection Baselines (Ranked by F1(Inf)) }
	\label{tab:stl_2class}
	\begin{tabular}{lcccccc}
		\toprule
		\textbf{Run} & \textbf{mAP@50} & \textbf{mAP@75} & \textbf{F1(Inf)} & \textbf{R(Inf)} & \textbf{P(Inf)} \\
		\midrule
		STL\_Det\_2cls\_Hybrid\_WithSampler & \textbf{0.5834} & \textbf{0.4144} & \textbf{0.5990} & \textbf{0.4631} & \textbf{0.8478} \\
		STL\_Det\_2cls\_LoRA\_WithSampler & 0.4442 & 0.2628 & 0.3043 & 0.1847 & 0.8642 \\
		STL\_Det\_2cls\_Hybrid\_NoSampler & 0.3457 & 0.1965 & 0.0000 & 0.0000 & 0.0000 \\
		\bottomrule
	\end{tabular}
\end{table}

\begin{table}[htbp]
	\centering
	\caption{3-Class Detection Baselines (Ranked by F1(Inf)) }
	\label{tab:stl_3class}
	\begin{tabular}{lcccccc}
		\toprule
		\textbf{Run} & \textbf{mAP@50} & \textbf{mAP@75} & \textbf{F1(Inf)} & \textbf{R(Inf)} & \textbf{P(Inf)} \\
		\midrule
		STL\_Det\_3cls\_LoRA\_WithSampler & \textbf{0.3341} & \textbf{0.2304} & \textbf{0.3791} & \textbf{0.2533} & \textbf{0.7529} \\
		STL\_Det\_3cls\_Hybrid\_WithSampler & 0.3191 & 0.1802 & 0.0401 & 0.0205 & 0.7273 \\
		STL\_Det\_3cls\_Hybrid\_NoSampler & 0.2145 & 0.1262 & 0.0000 & 0.0000 & 0.0000 \\
		\bottomrule
	\end{tabular}
\end{table}

\noindent  
\textbf{Observations:} Ranking by the infected-cell F1 score shows that the best STL setup changes with task complexity. Hybrid without a sampler works best for 1-class detection, Hybrid with a sampler for 2 classes, and LoRA-Only with a sampler for 3 classes, where it offers the most stable training. STL delivers strong specialists, but performance drops quickly as more classes are introduced. The best 3-class configuration, LoRA-Only with a sampler, serves as the main reference for MTTL comparisons.

\section{Evaluating STL Baselines and Their Limitations}
\label{sec:stl_baselines_and_limits}

\subsection{STL Baseline Performance Across All Diagnostic Tasks}
\label{ssec:stl_all_tasks}
We first trained specialist models for each task using the best STL settings. The results in Tables~\ref{tab:stl_baselines_full_det}--\ref{tab:stl_baselines_rest} give the reference performance for single-task approaches on this dataset. Figures \ref{fig:stl_dynamics_row1} through \ref{fig:stl_dynamics_row4} illustrates the stable training dynamics of these models.

\begin{table}[htbp]
	\centering
	\caption{STL baseline performance for Detection.}
	\label{tab:stl_baselines_full_det}
	\renewcommand{\arraystretch}{1.25}
	\begin{tabular}{lcccccc}
		\toprule
		\textbf{Task} & \textbf{mAP@50} & \textbf{mAP@75} & \textbf{F1(Inf)} & \textbf{F1-Macro} & \textbf{P-Macro} & \textbf{R-Macro} \\
		\midrule
		1-Class & \textbf{0.8260} & \textbf{0.502} & \textbf{0.8182} & 0.8182 & 0.8218 & 0.8146 \\
		2-Class & 0.5834 & 0.4144 & 0.5990 & \textbf{0.9274} & \textbf{0.9742} & \textbf{0.8848} \\
		3-Class & 0.3341 & 0.2304 & 0.3791 & 0.8977 & 0.9517 & 0.8496 \\
		\bottomrule
	\end{tabular}
\end{table}

\begin{table}[htbp]
	\centering
	\caption{STL baseline performance across tasks.}
	\label{tab:stl_baselines_rest}
	\begin{tabular}{lcccccc}
		\toprule
		\textbf{Task} & \textbf{Dice} & \textbf{IoU} & \textbf{Pearson} & \textbf{MAE} & \textbf{F1-Macro} & \textbf{Accuracy} \\
		\midrule
		Segmentation & 0.9656 & 0.9468 & -- & --  & --   & --    \\
		Heatmap Localization & 0.5422 & --  & 0.7866 & 0.9719 & -- & --     \\
		ROI Classification (2-Class) & -- & -- & -- & -- & 0.9029 & 0.9859 \\
		ROI Classification (3-Class) & -- & -- & --  & --   & 0.8047 & 0.9373 \\
		\bottomrule
	\end{tabular}
\end{table}

\noindent  
\textbf{Observations:} Segmentation, Localization Heatmap and RoI Classification STL models perform very well when each task is learned on its own. Detection on the other hand reaches strong infected-cell scores in the one-class case, but performance decreases as the number of classes grows. The high F1-Macro values in multi-class detection reflect the prevalence of the healthy class. This suggests that STL works for focused tasks but becomes less reliable as diagnostic complexity increases.

\subsection{Training Dynamics of STL Models}
Figures \ref{fig:stl_dynamics_row1} to \ref{fig:stl_dynamics_row4} show the training and validation curves for the best STL models in each task. The losses and main validation metrics evolve smoothly, indicating stable training and no signs of overfitting. This supports the reliability of the baseline results reported above.

\begin{figure}[htbp]
	\centering
	\begin{subfigure}[b]{0.49\textwidth}
		\centering
		\includegraphics[width=\textwidth]{figures/STL_detection_1cls_curves.png}
		\caption{Detection (1-Class)}
		\label{fig:dyn_det1}
	\end{subfigure}
	\hfill
	\begin{subfigure}[b]{0.49\textwidth}
		\centering
		\includegraphics[width=\textwidth]{figures/STL_detection_2cls_curves.png}
		\caption{Detection (2-Class)}
		\label{fig:dyn_det2}
	\end{subfigure}
	\caption{Training dynamics for 1-Class and 2-Class optimal STL detection models.}
	\label{fig:stl_dynamics_row1}
\end{figure}

\begin{figure}[htbp]
	\centering
	\begin{subfigure}[b]{0.49\textwidth}
		\centering
		\includegraphics[width=\textwidth]{figures/STL_detection_3cls_curves.png}
		\caption{Detection (3-Class)}
		\label{fig:dyn_det3}
	\end{subfigure}
	\hfill
	\begin{subfigure}[b]{0.49\textwidth}
		\centering
		\includegraphics[width=\textwidth]{figures/STL_roi_classif_2cls_curves.png}
		\caption{RoI Classification (2-Class)}
		\label{fig:dyn_roi2}
	\end{subfigure}
	\caption{Training dynamics for 3-Class detection and 2-Class RoI classification models.}
	\label{fig:stl_dynamics_row2}
\end{figure}

\begin{figure}[htbp]
	\centering
	\begin{subfigure}[b]{0.49\textwidth}
		\centering
		\includegraphics[width=\textwidth]{figures/STL_roi_classif_3cls_curves.png}
		\caption{RoI Classification (3-Class)}
		\label{fig:dyn_roi3}
	\end{subfigure}
	\hfill
	\begin{subfigure}[b]{0.49\textwidth}
		\centering
		\includegraphics[width=\textwidth]{figures/STL_segmentation_3cls_curves.png}
		\caption{Segmentation}
		\label{fig:dyn_seg}
	\end{subfigure}
	\caption{Training dynamics for 3-Class RoI classification and Segmentation models.}
	\label{fig:stl_dynamics_row3}
\end{figure}

\begin{figure}[htbp]
	\centering
	\begin{subfigure}[b]{0.49\textwidth}
		\centering
		\includegraphics[width=\textwidth]{figures/STL_heatmap_3cls_curves.png}
		\caption{Heatmap Localization}
		\label{fig:dyn_heat}
	\end{subfigure}
	\caption{Training dynamics for the Heatmap Localization model.}
	\label{fig:stl_dynamics_row4}
\end{figure}

\subsection{STL Performance Under Increasing Complexity}
\label{ssec:stl_degradation}

A key finding from our baseline analysis is the STL detector's inability to scale with increasing diagnostic complexity. As shown in Figure~\ref{fig:stl_collapse_viz}, F1-Score (Infected class), mAP@50 and mAP@75 show substantial performance drops as the model is required to distinguish between more cell classes.

\begin{figure}[htbp] 
	\centering 
	\begin{tikzpicture} 
		\begin{axis}[ 
			name=mainplot,
			width=0.75\textwidth, 
			height=8.5cm, 
			ymin=0, ymax=1.0, 
			ylabel={Performance Score}, 
			ylabel style={font=\bfseries\Large}, 
			xlabel={Task Complexity}, 
			xlabel style={font=\bfseries\Large}, 
			symbolic x coords={1-Class, 2-Class, 3-Class}, 
			xtick=data, 
			x tick label style={font=\normalsize\bfseries}, 
			ymajorgrids=true, 
			grid style={dotted, gray!20}, 
			enlarge x limits=0.3, 
			bar width=12pt,
			yticklabel style={/pgf/number format/fixed, /pgf/number format/precision=2}
			] 
			
			\fill[green!6, opacity=0.25] 
			(axis cs:1-Class,0.7) rectangle (axis cs:3-Class,1.0);
			\node[font=\scriptsize, text=green!60!black, anchor=south east] 
			at (axis cs:3-Class,0.98) {High Performance};
			
			\fill[yellow!8, opacity=0.25] 
			(axis cs:1-Class,0.4) rectangle (axis cs:3-Class,0.7);
			\node[font=\scriptsize, text=orange!65!black, anchor=east] 
			at (axis cs:3-Class,0.55) {Moderate};
			
			\fill[red!6, opacity=0.25] 
			(axis cs:1-Class,0.0) rectangle (axis cs:3-Class,0.4);
			\node[font=\scriptsize, text=red!65!black, anchor=north east] 
			at (axis cs:3-Class,0.02) {Critical Zone};
			
			\addplot[ybar, bar shift=-14pt, fill=blue!65, draw=blue!80!black, line width=0.7pt] 
			coordinates {(1-Class,0.8260) (2-Class,0.5834) (3-Class,0.3341)}; 
			
			\addplot[ybar, bar shift=0pt, fill=teal!55, draw=teal!75!black, line width=0.7pt] 
			coordinates {(1-Class,0.5020) (2-Class,0.4144) (3-Class,0.2304)}; 
			
			\addplot[ybar, bar shift=14pt, fill=orange!70, draw=orange!85!black, line width=0.7pt] 
			coordinates {(1-Class,0.8182) (2-Class,0.5990) (3-Class,0.3791)}; 
			
			\addplot[red!75!black, ultra thick, mark=square*, mark size=4pt, 
			mark options={fill=red!75!black}] 
			coordinates {(1-Class,0.8182) (2-Class,0.5990) (3-Class,0.3791)}; 
			
			\addplot[blue!70, thick, dashed, mark=*, mark size=3pt, 
			mark options={fill=blue!70}] 
			coordinates {(1-Class,0.8260) (2-Class,0.5834) (3-Class,0.3341)}; 
			
			\node[font=\tiny\bfseries, above=2pt, text=blue!90!black] 
			at (axis cs:1-Class,0.8260) [xshift=-14pt] {0.83};
			\node[font=\tiny\bfseries, above=2pt, text=teal!90!black] 
			at (axis cs:1-Class,0.5020) {0.50};
			\node[font=\tiny\bfseries, above=2pt, text=orange!90!black] 
			at (axis cs:1-Class,0.8182) [xshift=14pt] {0.82};
			
			\node[font=\tiny\bfseries, above=2pt, text=blue!90!black] 
			at (axis cs:2-Class,0.5834) [xshift=-14pt] {0.58};
			\node[font=\tiny\bfseries, above=2pt, text=teal!90!black] 
			at (axis cs:2-Class,0.4144) {0.41};
			\node[font=\tiny\bfseries, above=2pt, text=orange!90!black] 
			at (axis cs:2-Class,0.5990) [xshift=14pt] {0.60};
			
			\node[font=\tiny\bfseries, above=2pt, text=blue!90!black] 
			at (axis cs:3-Class,0.3341) [xshift=-14pt] {0.33};
			\node[font=\tiny\bfseries, above=2pt, text=teal!90!black] 
			at (axis cs:3-Class,0.2304) {0.23};
			\node[font=\tiny\bfseries, above=2pt, text=orange!90!black] 
			at (axis cs:3-Class,0.3791) [xshift=14pt] {0.38};
			 
			\draw[red!75!black, very thick, -stealth] 
			(axis cs:1-Class,0.92) -- (axis cs:3-Class,0.44) 
			node[midway, above=1pt, sloped, font=\footnotesize\bfseries, 
			text=red!75!black, fill=white, inner sep=2pt, opacity=0.95, rounded corners=1pt] 
			{$-54\%$ Degradation}; 
			
		\end{axis} 
		
		\node[right=0.5cm of mainplot.east, anchor=west, align=left, font=\small] (legend) {
			\textbf{Metrics:}\\[4pt]
			\tikz\draw[fill=blue!65, draw=blue!80!black, line width=0.7pt] (0,0) rectangle (0.3,0.15); \, mAP@50\\[2pt]
			\tikz\draw[fill=teal!55, draw=teal!75!black, line width=0.7pt] (0,0) rectangle (0.3,0.15); \, mAP@75\\[2pt]
			\tikz\draw[fill=orange!70, draw=orange!85!black, line width=0.7pt] (0,0) rectangle (0.3,0.15); \, F1-Score\\[6pt]
			\textbf{Trends:}\\[4pt]
			\tikz\draw[red!75!black, ultra thick, mark=square*] (0,0.075) -- (0.3,0.075); \, F1 Trend\\[2pt]
			\tikz\draw[blue!70, thick, dashed, mark=*] (0,0.075) -- (0.3,0.075); \, mAP@50
		};
		
	\end{tikzpicture} 
	
	\caption{STL detector performance degrades with complexity.} 
	\label{fig:stl_collapse_viz} 
\end{figure}

\noindent
\textbf{Observations:} As detection moves from 1 to 3 classes, the F1 score on the infected class drops noticeably, showing how STL struggles to keep focus on the rare but clinically important class. The high F1-Macro values in the multi-class setup mostly reflect the model’s confidence on the dominant uninfected class, which remains easy to learn. STL therefore performs well in simple, focused settings but becomes less reliable once the diagnostic problem gains complexity. This motivates the move toward a multi-task transfer learning approach that can handle richer and more realistic conditions.

\section{MTTL Performance and Synergy Analysis}
\label{sec:mttl_performance}
We now turn to our central hypothesis that a multi-task model, guided by auxiliary supervision, should counter the degradation observed in STL as detection becomes more complex. All experiments in this section use Hybrid Tuning without a sampler, since class imbalance is mild in the 1-Class setting and sampling does not bring meaningful benefits. We then assess the effect of adding segmentation (Seg), ROI classification (RoI) and heatmap localization (Loc) as auxiliary tasks. To measure how much each task contributes, we report both the raw F1-score on the infected class and the relative gain over the STL baseline for the same class configuration. For a metric \(X\), this gain is defined as:

\begin{equation}
	\label{eq:mttl_delta_gain}
	\Delta X(\%) = 
	\frac{X_{\text{MTTL}} - X_{\text{STL}}}{X_{\text{STL}}} \times 100\\
	\myequations{Performance gain formula.}
\end{equation}

\noindent A positive \(\Delta\) indicates improvement, whereas a negative \(\Delta\) reflects a decline. Tables~\ref{tab:mttl_1_class}–\ref{tab:mttl_3_class} present these values for the top STL baselines alongside the MTTL models for each modality.

\begin{table}[htbp]
	\centering
	\caption{MTTL experiments results for \textbf{1-Class Detection}.}
	\label{tab:mttl_1_class}
	\renewcommand{\arraystretch}{1.2}
	\begin{tabularx}{\textwidth}{lYYYYYY}
		\toprule
		\textbf{Task(s)} & \textbf{mAP@50} & \textbf{mAP@75} & \textbf{F1(Inf)} & \textbf{R(Inf)} & \textbf{P(Inf)} & \textbf{$\Delta$F1(Inf)} \\
		\midrule
			\textit{STL 1-Class Baseline } & \textbf{0.8260} & \textbf{0.5020} & \textbf{0.8182} & \textbf{0.8218} & \textbf{0.8146} & -- \\
		\addlinespace
		\addlinespace
		+Seg & 0.7885 & 0.6067 & 0.7654 & 0.6953 & 0.8514 & -6.45\% \\
		+Seg+Loc & 0.7709 & 0.5805 & 0.7497 & 0.7388 & 0.7609 & -8.38\% \\
		+Loc & 0.7367 & 0.5363 & 0.7285 & 0.7098 & 0.7483 & -10.96\% \\
		\bottomrule
	\end{tabularx}
\end{table}

\begin{table}[htbp]
	\centering
	\caption{MTTL experiments results for \textbf{2-Class Detection}.}
	\label{tab:mttl_2_class}
	\renewcommand{\arraystretch}{1.2}
	\begin{tabularx}{\textwidth}{lYYYYYYY}
		\toprule
		\textbf{Task(s)} & \textbf{mAP@50} & \textbf{mAP@75} & \textbf{F1(Inf)} & \textbf{F1-M} & \textbf{Prec-M} & \textbf{Rec-M} & \textbf{$\Delta$F1(Inf)} \\
		\midrule
		\textit{STL 2-Class Baseline} & 0.5834 & 0.4144 & 0.5990 & 0.9274 & 0.9742 & 0.8848 & -- \\
		\addlinespace
		\addlinespace
		+RoI & \textbf{0.6452} & \textbf{0.4588} & \textbf{0.7810} & \textbf{0.9411} & 0.9761 & \textbf{0.9085} & \textbf{+30.38\%} \\
		+Seg & 0.6281 & 0.4462 & 0.7464 & 0.9330 & 0.9734 & 0.8959 & +24.60\% \\
		+Loc & 0.6398 & 0.4562 & 0.7415 & 0.9282 & 0.9737 & 0.8867 & +23.80\% \\
		\addlinespace
		\addlinespace
		+Seg+RoI & 0.6113 & 0.3851 & 0.6976 & 0.9256 & \textbf{0.9785} & 0.8780 & +16.46\% \\
		+Seg+Loc & 0.6025 & 0.4203 & 0.6120 & 0.9025 & 0.9550 & 0.8555 & +2.17\% \\
		+RoI+Loc & 0.5958 & 0.4180 & 0.6065 & 0.8958 & 0.9492 & 0.8481 & +1.25\% \\
		\addlinespace
		\addlinespace
		+Seg+RoI+Loc & 0.5872 & 0.4091 & 0.6030 & 0.8897 & 0.9453 & 0.8402 & +0.67\% \\
		\bottomrule
	\end{tabularx}
\end{table}

\begin{table}[htbp]
	\centering
	\caption{MTTL experiments results for \textbf{3-Class Detection}.}
	\label{tab:mttl_3_class}
	\renewcommand{\arraystretch}{1.2}
	\begin{tabularx}{\textwidth}{lYYYYYYY}
		\toprule
		\textbf{Task(s)} & \textbf{mAP@50} & \textbf{mAP@75} & \textbf{F1(Inf)} & \textbf{F1-M} & \textbf{Prec-M} & \textbf{Rec-M} & \textbf{$\Delta$F1(Inf)} \\
		\midrule
		\textit{STL 3-Class Baseline} & 0.3341 & 0.2304 & 0.3791 & 0.8977 & 0.9517 & 0.8496 & -- \\
		\addlinespace
		\addlinespace
		+Seg & 0.7013 & 0.5543 & 0.7230 & 0.9351 & 0.9780 & 0.8958 & +90.74\% \\
		+Loc & 0.6994 & 0.5217 & 0.7034 & 0.9195 & 0.9595 & 0.8827 & +85.55\% \\
		+RoI & \textbf{0.7402} & 0.5475 & \textbf{0.7710} & \textbf{0.9409} & 0.9739 & \textbf{0.9101} & \textbf{+103.39\%} \\
		\addlinespace
		\addlinespace
		+Seg+RoI & 0.7089 & 0.4375 & 0.6662 & 0.9289 & 0.9729 & 0.8888 & +75.73\% \\
		+Loc+RoI & 0.7289 & 0.5128 & 0.6722 & 0.9375 & \textbf{0.9803} & 0.8983 & +77.32\% \\
		+Seg+Loc & 0.6913 & 0.4853 & 0.5468 & 0.9362 & 0.9739 & 0.9013 & +44.24\% \\
		\addlinespace
		\addlinespace
		+Seg+RoI+Loc & 0.7324 & \textbf{0.5538} & 0.7360 & 0.9364 & 0.9761 & 0.8997 & +94.15\% \\
		\bottomrule
	\end{tabularx}
\end{table} 

\subsection{Analysis of MTTL Synergy}
\label{ssec:mttl_synergy_analysis}

Our experiments results provide evidence for our central hypothesis that auxiliary tasks, while sometimes detrimental to highly specialized single-task models, become increasingly beneficial as the complexity of the primary task grows.The benefits provided by auxiliary tasks depends on the complexity of the primary detection problem. As the task becomes more nuanced, auxiliary supervision can provide meaningful guidance, though the choice and combination of tasks.\\

\noindent For \textbf{1-Class Detection}, adding auxiliary tasks consistently hurts performance. The STL baseline dominates, and all MTTL variants show relative F1(Inf) drops from 6\% to 11\%. The task is simple enough that extra objectives introduce interference rather than useful signal. In this case, dedicating the model solely to detecting infected cells is most effective.\\

\noindent For \textbf{2-Class Detection}, auxiliary tasks begin to provide clear gains. RoI Classification is the most synergistic, giving +30.38\% in F1(Inf), likely due to its region level feature refinement aiding distinction between infected and healthy cells. Segmentation and Localization also help, with +24.60\% and +23.80\% gains, respectively, showing moderate complementary effects. When multiple tasks are combined, the benefits diminish, +Seg+RoI drops to +16.46\%, and combinations like +Seg+Loc or +RoI+Loc give only +2.17\% and +1.25\%, while the full three-task setup barely improves (+0.67\%). This suggests that a few well chosen auxiliary tasks are synergistic, adding too many can create conflicts, possibly due to competing gradients or overlapping supervision signals.\\

\noindent For \textbf{3-Class Detection}, auxiliary supervision provides the largest advantage. All MTTL variants outperform the weak STL baseline, with relative F1(Inf) improvements from +44\% to +103\%. RoI Classification leads (+103.39\%), followed by Segmentation (+90.74\%) and Localization (+85.55\%), each contributing complementary strengths: RoI refines local discriminative features, Segmentation provides structural guidance, and Localization helps with spatial attention. Paired combinations like Seg+RoI (+75.73\%) and Loc+RoI (+77.32\%) remain strong, whereas Seg+Loc (+44.24\%) shows weaker synergy, suggesting overlap in the signals they provide. The full three-task configuration recovers to +94.15\%, indicating that RoI stabilizes the learning when combined with multiple auxiliaries. Overall, Segmentation and RoI are the most consistently synergistic, while Localization is helpful but can conflict when stacked with multiple tasks. 

\subsection{Training Dynamics for Top MTTL Models}
We illustrate the training dynamics of the best-performing MTTL models for each detection category \autoref{fig:mttl_dynamics_1cls} to \autoref{fig:mttl_dynamics_3cls}. Plots correspond to loss curves, detection metrics, and auxiliary metrics.

\begin{figure*}[htbp]
	\centering
	\begin{subfigure}{0.32\textwidth}
		\includegraphics[width=\linewidth]{figures/MTTL_Det_1cls__Seg_Sampler_LossCurves.png}
		\caption{Loss}
	\end{subfigure}
	\begin{subfigure}{0.32\textwidth}
		\includegraphics[width=\linewidth]{figures/MTTL_Det_1cls__Seg_Sampler_Metrics_Detection.png}
		\caption{Detection}
	\end{subfigure}	
	\begin{subfigure}{0.32\textwidth}
		\includegraphics[width=\linewidth]{figures/MTTL_Det_1cls__Seg_Sampler_Metrics_Segmentation.png}
		\caption{Heatmap}
	\end{subfigure}
	
	\caption{Best 1-Class Detection MTTL model (+Seg, F1=0.7654).}
	\label{fig:mttl_dynamics_1cls}
\end{figure*}

\begin{figure*}[htbp]
	\centering
	\begin{subfigure}{0.32\textwidth}
		\includegraphics[width=\linewidth]{figures/MTTL_Det_2cls__ROI_NoSampler_LossCurves.png}
		\caption{Loss}
	\end{subfigure}
	\begin{subfigure}{0.32\textwidth}
		\includegraphics[width=\linewidth]{figures/MTTL_Det_2cls__ROI_NoSampler_Metrics_Detection.png}
		\caption{Detection}
	\end{subfigure}
	\begin{subfigure}{0.32\textwidth}
		\includegraphics[width=\linewidth]{figures/MTTL_Det_2cls__ROI_NoSampler_Metrics_Roi_classif.png}
		\caption{ROI Classification}
	\end{subfigure}
	
	\caption{Best 2-Class Detection MTTL model (+RoI, F1=0.7810).}
	\label{fig:mttl_dynamics_2cls}
\end{figure*}

\begin{figure*}[htbp]
	\centering
	\begin{subfigure}{0.32\textwidth}
		\includegraphics[width=\linewidth]{figures/MTTL_Det_3cls__ROI_NoSampler_LossCurves.png}
		\caption{Loss}
	\end{subfigure}
	\begin{subfigure}{0.32\textwidth}
		\includegraphics[width=\linewidth]{figures/MTTL_Det_3cls__ROI_NoSampler_Metrics_Detection.png}
		\caption{Detection}
	\end{subfigure}
	\begin{subfigure}{0.32\textwidth}
		\includegraphics[width=\linewidth]{figures/MTTL_Det_3cls__ROI_NoSampler_Metrics_Roi_classif.png}
		\caption{Segmentation}
	\end{subfigure}
	
	\caption{Best 3-Class Detection MTTL model (+RoI, F1=0.7710).}
	\label{fig:mttl_dynamics_3cls}
\end{figure*}


\section{Final Model Evaluation and Robustness Analysis}
\label{sec:final_evaluation}
Following the baseline evaluation, we now focus on clinical robustness using the specialized subsets defined in \autoref{ssec:diagnostic_datasets}. We assess \textbf{sensitivity} on infected-only slides and \textbf{specificity} on healthy-only slides. Furthermore, we extend the analysis to \textbf{parasitemia quantification}, estimating infection load via the count-based method in \autoref{eq:parsitemia_formula}. These tests provide a complete view of reliability and clinical utility.

\subsection{Detection Performance Leaderboard} 
\label{ssec:sota_leaderboard}
Table~\ref{tab:final_leaderboard} presents a leaderboard of the top performing models. Champions are selected based on the highest F1 score for the infected class, enabling a direct comparison across 1, 2, and 3-class tasks. The results illustrate a clear trend: while STL dominates the simplest setting, MTTL becomes increasingly superior as diagnostic complexity grows.

\begin{table}[htbp]
	\centering
	\caption{Best performers by detection task complexity.}
	\label{tab:final_leaderboard}
	\renewcommand{\arraystretch}{1.2}
	\begin{tabular}{p{6.5cm} l c c c c}
		\toprule
		\textbf{Model Configuration} & \textbf{Paradigm} & \textbf{Classes} & \textbf{mAP@50} & \textbf{F1-Macro} & \textbf{F1 (Inf.)} \\
		\midrule
		Det(1) & STL  & 1 & \textbf{0.8260} & \textbf{0.8182} & \textbf{0.8182} \\
		Det(1)+Seg & MTTL & 1 & 0.7885 & 0.7654 & 0.7654 \\
		\midrule
		Det(2) & STL  & 2 & 0.5834 & 0.9274 & 0.5990 \\
		Det(2)+RoI & MTTL & 2 & \textbf{0.6452} & \textbf{0.9411} & \textbf{0.7810} \\
		\midrule
		Det(3) & STL  & 3 & 0.3341 & 0.8977 & 0.3791 \\
		Det(3)+RoI & MTTL & 3 & \textbf{0.7402} & \textbf{0.9409} & \textbf{0.7710} \\
		\bottomrule
	\end{tabular}
\end{table}

\subsection{Diagnostic Robustness on Specialized Subsets}
\label{ssec:robustness_analysis}

To test hypothesis H3 (Robustness), we evaluated the champion models on clinically relevant subsets. We measured F1, Recall, and Precision on infected-only slides, and counted total false positives on healthy-only slides (Table \ref{tab:spec_sets}).

\begin{table}[htbp]
	\centering
	\caption{Performance of best detection models on specialized test subsets.}
	\label{tab:spec_sets}
	\renewcommand{\arraystretch}{1.2}
	\begin{tabularx}{\textwidth}{lYYYYYYY}
		\toprule
		\textbf{Model} & \textbf{Paradigm} & \textbf{Classes} & \textbf{F1(Inf)} & \textbf{R(Inf)} & \textbf{P(Inf)} & \textbf{$\sum$\acp{FP}(H)} \\
		\midrule
		Det(1) & STL & 1 & 0.7725 & \textbf{0.8491} & 0.7086 & 12 \\
		Det(2) & STL & 2 & 0.6021 & 0.4631 & 0.8603 & 8 \\
		Det(3) & STL & 3 & 0.2376 & 0.1385 & 0.8333 & 6 \\
		Det(1)+Seg & MTTL & 1 & 0.7250 & 0.8193 & 0.6503 & 27 \\
		Det(2)+RoI & MTTL & 2 & \textbf{0.7907} & 0.7401 & 0.8487 & 10 \\
		Det(3)+RoI & MTTL & 3 & 0.7359 & 0.6214 & \textbf{0.9023} & \textbf{3} \\
		\bottomrule
	\end{tabularx}
\end{table}

\noindent Consistent with previous findings, STL excels in the 1-class setting with high sensitivity ($F1=0.7725$, Recall=0.8491). However, performance degrades sharply with complexity; 3-class STL drops to an F1 of $0.2376$ and Recall of $0.1385$. In contrast, MTTL models sustain performance. The 3-class MTTL model achieves an F1 of $0.7359$ and Recall of $0.6214$, while generating only 3 false positives on healthy slides. This confirms that MTTL offers a superior balance of sensitivity and specificity for complex diagnostics.

\subsection{Parasitemia Quantification on Multi-Class Detection Models}
\label{ssec:parasitemia_quantification}

A key measure of diagnostic utility is accurate slide-level parasitemia estimation on the \textbf{full test set}. We therefore evaluated the champion 2-Class and 3-Class models from \autoref{ssec:sota_leaderboard}, deriving parasitemia using the count-based formulation in \autoref{eq:parsitemia_formula}. The results are summarized below.

\begin{figure}[htbp]
	\centering
	\begin{subfigure}[b]{0.23\textwidth}
		\includegraphics[width=\textwidth,keepaspectratio]{figures/STL_Det_2cls_Hybrid_WithSampler_correlation.png}
		\caption{STL(2) Corr.}
	\end{subfigure}
	\hfill
	\begin{subfigure}[b]{0.23\textwidth}
		\includegraphics[width=\textwidth,keepaspectratio]{figures/MTTL_Det_2cls__ROI_NoSampler_correlation.png}
		\caption{MTTL(2) Corr.}
	\end{subfigure}
	\hfill
	\begin{subfigure}[b]{0.23\textwidth}
		\includegraphics[width=\textwidth,keepaspectratio]{figures/STL_Det_2cls_Hybrid_WithSampler_residuals.png}
		\caption{STL(2) Resid.}
	\end{subfigure}
	\hfill
	\begin{subfigure}[b]{0.23\textwidth}
		\includegraphics[width=\textwidth,keepaspectratio]{figures/MTTL_Det_2cls__ROI_NoSampler_residuals.png}
		\caption{MTTL(2) Resid.}
	\end{subfigure}
	
	\vspace{2mm}
	\begin{subfigure}[b]{0.23\textwidth}
		\includegraphics[width=\textwidth,keepaspectratio]{figures/STL_Det_3cls_Hybrid_WithSampler_correlation.png}
		\caption{STL(3) Corr.}
	\end{subfigure}
	\hfill
	\begin{subfigure}[b]{0.23\textwidth}
		\includegraphics[width=\textwidth,keepaspectratio]{figures/MTTL_Det_3cls__Seg_NoSampler_correlation.png}
		\caption{MTTL(3) Corr.}
	\end{subfigure}
	\hfill
	\begin{subfigure}[b]{0.23\textwidth}
		\includegraphics[width=\textwidth,keepaspectratio]{figures/STL_Det_3cls_Hybrid_WithSampler_residuals.png}
		\caption{STL(3) Resid.}
	\end{subfigure}
	\hfill
	\begin{subfigure}[b]{0.23\textwidth}
		\includegraphics[width=\textwidth,keepaspectratio]{figures/MTTL_Det_3cls__Seg_NoSampler_residuals.png}
		\caption{MTTL(3) Resid.}
	\end{subfigure}
	
	\caption{Parasitemia estimation for all best multi-class detection models.}
	\label{fig:parasitemia_plots_all}
\end{figure}

\begin{table}[htbp]
	\centering
	\caption{Parasitemia estimation performance of the champion models on the full test set.}
	\label{tab:parasitemia_results}
	\renewcommand{\arraystretch}{1.2}
	\begin{tabular}{p{6.5cm} l c c}
		\toprule
		\textbf{Model Configuration} & \textbf{Paradigm} & \textbf{MAE (\%)} & \textbf{Pearson Correlation (r)} \\
		\midrule
		Det(1) & STL  & -- & -- \\
		Det(1)+Seg & MTTL & -- & -- \\
		\midrule
		Det(2) & STL  & 2.2628 & 0.9048 \\
		Det(2)+RoI & MTTL & \textbf{0.9211} & \textbf{0.9824} \\
		\midrule
		Det(3) & STL  & 3.2963 & 0.8145 \\
		Det(3)+RoI & MTTL & \textbf{1.0805} & \textbf{0.9739} \\
		\bottomrule
	\end{tabular}
\end{table}

\noindent The quantitative results in \autoref{tab:parasitemia_results} show clear clinical benefits for MTTL. In the 2-class setting, MTTL reduces MAE to $0.92\%$ from STL's $2.26\%$ (a $59\%$ reduction). In the 3-class setting, it lowers MAE to $1.08\%$ from $3.30\%$ (a $67\%$ reduction) and improves correlation to $0.97$. These figures confirm that multi-task learning significantly improves both detection robustness and the precision of infection load quantification.\\

\noindent Visual analysis (\autoref{fig:parasitemia_plots_all}) supports this, with MTTL predictions aligning closely to ground-truth values. Residual plots show smaller, less biased errors compared to STL. This combination of accuracy and stability points to the framework's reliability for clinical use.

\section{Benchmarking Against State-Of-The-Art}
\label{sec:benchmarking_sota}

To the best of our knowledge at the moment this work is written, direct comparison with prior malaria detection studies is challenging due to differences in staining protocols, parasite species, and diagnostic objectives. We therefore selected YOLOv8 \parencite{jocher2023yolo}, a widely adopted single-stage detector, as our baseline. YOLO variants have been successfully applied to malaria detection, for example \textcite{Abdurahman2021ModifiedYOLO} used a modified YOLOv4 on thick-smear images, while \textcite{sukumarran_automated_2023} extended it to thin smears with multiple \textit{Plasmodium} species.

\subsection{Experimental Setup}
\label{subsec:sota_setup}

We trained YOLOv8-Nano and YOLOv8-Small to represent distinct computational constraints. To ensure a fair comparison, we strictly maintained the experimental conditions used for the MTTL framework:
\begin{itemize}
	\item \textbf{Data Consistency:} The pre-processed patient-stratified train, validation, and test splits used for MTTL were converted to YOLO format, guaranteeing evaluation on identical unseen data.
	\item \textbf{Training Protocol:} Models were trained for 100 epochs with early stopping at 15 epochs, using the SGD optimizer and mosaic augmentation at $512 \times 512$ resolution.
	\item \textbf{Evaluation Logic:} Raw predictions were processed through our evaluation pipeline \autoref{sec:experimental_protocol}, ensuring mAP and F1-scores are calculated identically across all models.
\end{itemize}

\subsection{Architecture and Training Dynamics}
\label{ssec:architecture_training}

Table~\ref{tab:arch_comparison} contrasts the architectures. Our model utilizes a ResNet-50 backbone with LoRA adapters, whereas YOLOv8 employs CSPDarknet \parencite{wang2020cspnet}, optimized for single-stage detection. Despite a higher total parameter count, our approach selectively fine-tunes only $18.12\text{M}$ parameters via LoRA.

\begin{table}[htbp]
	\centering
	\caption{Architecture comparison: YOLOv8 baselines vs. MTTL detector.}
	\label{tab:arch_comparison}
	\begin{tabular}{l c c c}
		\toprule
		\textbf{Model} & \textbf{Backbone} & \textbf{Total Params} & \textbf{Trainable Params} \\
		\midrule
		YOLOv8-n & CSPDarknet & ~$3.0\text{M}$ & ~$3.0\text{M}$ \\
		YOLOv8-s & CSPDarknet & ~$12.7\text{M}$ & ~$12.7\text{M}$ \\
		\textbf{MTTL (Ours)} & \textbf{ResNet-50 + LoRA} & \textbf{26.66M} & \textbf{18.12M} \\
		\bottomrule
	\end{tabular}
\end{table}

\noindent Figure~\ref{fig:yolo_convergence} illustrates training convergence. Both YOLOv8 variants converge smoothly, with the Small model achieving lower validation loss. However, these aggregate curves mask the model's struggle with the minority infected class, which only becomes apparent in the detailed per-class metrics discussed below.

\begin{figure}[htbp]
	\centering
	\begin{subfigure}[b]{1\textwidth} 
		\centering
		\includegraphics[width=\linewidth, keepaspectratio]{figures/yolov8n_results.png}
		\caption{YOLOv8-Nano}
		\label{fig:yolov8n_results}
	\end{subfigure}
	
	\vspace{0.5cm} 

	\begin{subfigure}[b]{1\textwidth}
		\centering
		\includegraphics[width=\linewidth, keepaspectratio]{figures/yolov8s_results.png}
		\caption{YOLOv8-Small}
		\label{fig:yolov8s_results}
	\end{subfigure}
	
	\caption{Training convergence for YOLOv8 variants.}
	\label{fig:yolo_convergence}
\end{figure}

\subsection{Detection Performance}
\label{ssec:detection_performance}

3-Class Detection performance on the full test set (\autoref{ssec:diagnostic_datasets}) is presented in Table~\ref{tab:sota_comparison}. While YOLOv8-Small reaches a solid mAP@50 of $0.674$, it underperforms on the minority infected class ($\text{F1} = 0.572$), likely prioritizing the abundant healthy cells to minimize global loss. Our MTTL detector significantly outperforms it, achieving an infected class F1-score of $0.771$, a $34.8\%$ improvement. The macro-averaged F1-score also shows a substantial $62.0\%$ gain, reflecting balanced performance across all three classes rather than dominance by the majority class.

\begin{table}[htbp]
	\centering
	\caption{Detection performance on the 3-class test set. Best scores in bold.}
	\label{tab:sota_comparison}
		\begin{tabular}{l c c c c}
			\toprule
			\textbf{Metric} & \textbf{YOLOv8-n} & \textbf{YOLOv8-s} & \textbf{MTTL (Ours)} & \textbf{$\Delta$ vs v8s} \\
			\midrule
			mAP@50 & 0.587 & 0.674 & \textbf{0.740} & +9.8\% \\
			mAP@75 & 0.400 & 0.462 & \textbf{0.548} & +18.6\% \\
			\midrule
			F1-Macro & 0.570 & 0.581 & \textbf{0.941} & +62.0\% \\
			Recall-Macro & 0.543 & 0.784 & \textbf{0.910} & +16.1\% \\
			Prec.-Macro & 0.652 & 0.563 & \textbf{0.974} & +73.0\% \\
			\midrule
			F1 (Inf) & 0.583 & 0.572 & \textbf{0.771} & +34.8\% \\
			Recall (Inf) & 0.573 & 0.575 & \textbf{0.715} & +24.3\% \\
			Precision (Inf) & 0.592 & 0.569 & \textbf{0.836} & +46.8\% \\
			\bottomrule
		\end{tabular}
\end{table}

\subsection{Parasitemia Estimation}
\label{ssec:clinical_impact}

Superior detection translates directly to parasitemia estimation accuracy (Table~\ref{tab:parasitemia_sota}). MTTL achieves an MAE of just $1.08\%$, compared to $3.78\%$ for YOLOv8-Small, a $71.4\%$ error reduction. The Pearson correlation with ground truth jumps from $0.548$ to $0.953$, indicating near-linear agreement between predicted and actual values.

\begin{table}[htbp]
	\centering
	\caption{Parasitemia estimation accuracy across models.}
	\label{tab:parasitemia_sota}
	\begin{tabular}{l c c c}
		\toprule
		\textbf{Metric} & \textbf{YOLOv8-n} & \textbf{YOLOv8-s} & \textbf{MTTL (Ours)} \\
		\midrule	
		MAE (\%) & 2.48 & 3.78 & \textbf{1.08} \\
		Pearson $r$ & 0.883 & 0.548 & \textbf{0.953} \\
		\bottomrule
	\end{tabular}
\end{table}

\noindent Notably, the smaller YOLOv8-Nano yielded a lower estimation error ($2.48\%$) than the larger YOLOv8-Small ($3.78\%$), despite having a lower mAP. This anomaly occurs because parasitemia is a ratio, YOLOv8-Small exhibited lower precision on the abundant healthy cells, skewing the total cell count denominator and amplifying the ratio error. MTTL avoids this trade-off by maintaining balanced performance across classes.\\

\noindent Clinically, this accuracy is vital as WHO treatment protocols rely on specific severity thresholds, a $3.78\%$ error risks misclassification, whereas $1.08\%$ remains within safe bounds. This reliability comes from the multi-task architecture, where auxiliary segmentation and RoI classification regularize the feature extractor to capture structural details that single-task detectors miss. We conclude that for imbalanced medical datasets, the diagnostic reliability gained from MTTL justifies the computational cost over standard object detectors.
\cleardoublepage

%DISCUSSION
\chapter{Discussion}
\label{chap:discussion}

This chapter interprets our results in relation to the initial hypotheses. We discuss the trade-off between specialized and generalized models, the answer to our central research question, the implications for medical AI, and the limitations of the study.

\section{The Specialist vs. The Generalist}
\label{sec:specialist_vs_generalist}

A key finding from our experiments is the clear trade-off between specialization and generalization. The 1-Class STL detector, trained solely to find infected cells, acted as a formidable specialist. On the full test set, it achieved the highest F1-Score for the infected class ($0.8182$) and an outstanding Recall of $0.8491$ on the challenging infected-only subset. However, this aggressive sensitivity reduced specificity, resulting in 12 false positives on healthy-only slides. More critically, the specialist model was brittle. When we expanded the task to the realistic 3-class setting, the STL framework collapsed, with the infected-class F1-Score dropping to a clinically inadequate $0.3791$.\\

\noindent Conversely, the 3-Class MTTL model augmented with RoI Classification proved to be a superior generalist. While it did not quite match the specialist's peak score on the simple task, it dramatically outperformed the STL baselines in the complex setting. It maintained a strong infected-class F1-Score of $0.7710$ in the 3-class scenario, a relative improvement of over \textbf{103\%} compared to the STL equivalent. Importantly, it achieved this while improving specificity, committing only 3 false positive errors on the healthy-only set. This ability to trade a small amount of narrow-task performance for reliable, broad, and safe diagnostic coverage aligns better with clinical requirements.

\section{Comparing MTTL to State-of-the-Art Detection}
\label{sec:discussion_mttl_vs_sota}

The comparison against YOLOv8 (Section \ref{sec:benchmarking_sota}) reveals important details. YOLOv8 is a highly optimized, single-stage architecture designed to maximize mean Average Precision (mAP) across all classes. While it achieved a decent overall score, it struggled with the minority \textit{Infected} class (F1 0.57) compared to our model (F1 0.77).\\

\noindent This difference shows that pure data-driven learning has limits in medical domains. YOLO learns \textit{where} objects are based on bounding boxes. However, it does not necessarily learn the fine morphological details needed to distinguish a trophozoite from an artifact when data is scarce.\\

\noindent Our MTTL framework, even with an older ResNet-50 backbone, outperformed the modern YOLO architecture because of the structure provided by the auxiliary tasks:
\begin{itemize}
	\item The \textbf{Segmentation} task forces the model to understand the exact pixel boundaries of the parasite, preventing it from confusing overlapping cells.
	\item The \textbf{RoI Classification} task forces a second feature extraction step on specific cell regions, refining the decision boundary between healthy cells, infected cells and WBCs.
\end{itemize}
\noindent This suggests that for specialized medical tasks with limited data, explicitly modeling the problem structure works better than simply using a faster, newer architecture.

\section{Answering the Central Research Question}
\label{sec:answering_research_question}

We sought to determine \textit{whether a multi-task transfer learning framework provides greater efficacy and robustness than single-task learning for automated malaria diagnosis}. Our findings support the conclusion: \textbf{Yes, for diagnostically realistic settings, multi-task transfer learning is the superior paradigm.} This is supported by four pillars of evidence:
\begin{itemize}
	\item \textbf{Performance Scaling:} MTTL prevented the severe performance degradation that STL suffered when scaling from 1-class to complex 3-class detection (Hypothesis H1).
	\item \textbf{Auxiliary Task Synergy:} The performance gains from specific auxiliary tasks confirm that adding relevant inductive biases benefits this domain (Hypothesis H2).
	\item \textbf{Robustness:} The 3-Class MTTL model achieved a far better balance of sensitivity (Recall $0.6214$) and specificity (low false positives) compared to the STL models, which struggled with complexity (Hypothesis H3).
	\item \textbf{Clinical Utility:} The improved representations translated to better downstream utility. The 3-Class MTTL model reduced the Parasitemia Mean Absolute Error from $3.30\%$ (STL) to $1.08\%$, a 67\% reduction in estimation error.
\end{itemize}

\section{Implications of Key Findings}
\label{sec:implications}

These results have practical implications for the design of diagnostic systems in computational pathology. First, benchmarking on simple, binary tasks is risky. A model that appears state-of-the-art on a narrow benchmark may fail in a real-world workflow that requires multiple, interdependent outputs. The collapse of our STL models in the multi-class setting demonstrates this danger.\\

\noindent Second, specific auxiliary tasks act as powerful regularizers. RoI Classification was the most effective, delivering performance gains of over 103\% for 3-class detection. This suggests that forcing the model to refine features on localized cell instances is highly beneficial. Segmentation also proved synergistic, providing shape priors that improved performance by roughly 90\%. However, we observed negative transfer in the simple 1-class setting, showing that synergy is not automatic, it requires matching auxiliary objectives to the complexity of the primary task.\\

\noindent Finally, the success of our Hybrid Tuning strategy offers a blueprint for future work. It demonstrates that combining the efficiency of PEFT methods like LoRA with targeted fine-tuning of deep backbone layers is a computationally feasible way to adapt large, pre-trained models to complex medical domains.

\section{Limitations of the Study}
\label{sec:limitations}

While this research provides strong evidence for the benefits of MTTL, we must acknowledge specific limitations that define the scope of our conclusions.

\begin{itemize}
	\item \textbf{Dataset Scope:} We used only the NLM dataset, which contains Giemsa-stained thin blood smears of \textit{P. falciparum}. We have not validated these models on other species (e.g., \textit{P. vivax}) or thick blood smears. True clinical robustness requires evaluation on multi-center datasets that account for variations in slide preparation and imaging hardware.
	
	\item \textbf{Annotation Noise:} The dataset contains inherent noise, such as ambiguous cell boundaries and potential inconsistencies in labeling early-stage trophozoites. This noise creates a performance ceiling for supervised training and may explain some discrepancies between segmentation and detection performance.
	
	\item \textbf{Task Interference:} Multi-task synergy is not guaranteed. The heatmap localization task consistently interfered with other auxiliary tasks in higher-order combinations. Future work should explore architectures like cross-stitch networks or gradient surgery to mitigate this negative transfer.
	
	\item \textbf{Optimization Constraints:} Practical limits constrained our hyperparameter tuning. While we applied a consistent protocol, each configuration would likely benefit from an exhaustive search for optimal learning rates and loss weights. Additionally, Automatic Mixed Precision required careful implementation to prevent numerical instability.
	
	\item \textbf{Simplified Parasitemia Estimation:} We estimated parasitemia based on cell counts from single image fields. Clinical practice typically requires counting parasites over many fields to ensure statistical validity. A fully validated clinical tool would need to aggregate counts across multiple fields.
\end{itemize}
\cleardoublepage

%CONCLUSION
\chapter{Conclusion}
%\addcontentsline{toc}{chapter}{Conclusion}
\label{chap:conclusion}

This work investigated Multi-Task Transfer Learning (MTTL) as a solution for automated malaria diagnosis. We compared this approach against Single-Task Learning (STL) to determine if learning related tasks simultaneously could improve diagnostic reliability. Our findings provide clear evidence that MTTL, when combined with parameter-efficient fine-tuning, offers a robust and scalable framework for medical image analysis. This chapter summarizes our contributions, the principal findings, and the implications for future research.

\section*{Summary of Contributions}

The primary contribution of this work is the formulation and validation of a multi-task transfer learning framework specifically designed for malaria detection. The key contributions are:

\begin{itemize}
	\item \textbf{A Mathematical Framework for MTTL:} We proposed a formal mathematical definition for MTTL that explicitly handles task interactions, loss aggregation, and regularization. This provides a reproducible basis for future research and ensures that performance gains are analytically interpretable.
	\item \textbf{System Design and Hybrid Tuning:} We designed a modular architecture that integrates detection, segmentation, and region-of-interest classification. By combining Low-Rank Adaptation (LoRA) with selective fine-tuning of deep layers, we introduced a computationally efficient strategy to adapt large pre-trained models without the cost of full retraining.
	\item \textbf{Empirical Validation of Task Synergy:} Through rigorous experimentation, we identified which auxiliary tasks provide beneficial inductive biases. We demonstrated that RoI Classification and Segmentation significantly enhance feature learning, while also highlighting cases where negative transfer can occur, such as with heatmap localization in certain configurations.
\end{itemize}

\section*{Principal Findings}

Our experiments yielded several critical insights into the behavior of deep learning models in diagnostic settings:

\begin{itemize}
	\item \textbf{MTTL Outperforms STL in Realistic Scenarios:} STL models proved to be excellent specialists but poor generalists. While the 1-Class STL detector achieved a high F1-score of $0.8182$ on simple tasks, it collapsed when faced with the realistic complexity of multi-class detection. In contrast, MTTL models maintained robust performance, achieving a superior balance of sensitivity and specificity even as the diagnostic difficulty increased.
	\item \textbf{Performance against Industry Standards:} Our framework showed higher diagnostic reliability than the YOLOv8 model. While YOLOv8 provided strong general detection, MTTL improved sensitivity for infected cells by 34.7\% and reduced the parasitemia estimation error by over 70\%. This highlights the value of task-specific regularization.
	\item \textbf{Auxiliary Tasks act as Regularizers:} The addition of specific auxiliary tasks improved generalization. RoI Classification emerged as the most effective, delivering a relative gain of over 103\% in the 3-class F1-score. Segmentation also provided strong structural cues, yielding a 90\% improvement. However, these benefits are not automatic; in simple 1-class settings, auxiliary tasks sometimes reduced performance, indicating that task selection must match the problem complexity.
	\item \textbf{Clinical Relevance:} The improved stability of the MTTL models translated directly into better clinical metrics. For parasitemia quantification, the 3-Class MTTL model reduced the mean absolute error from $3.30\%$ (STL) to $1.08\%$. This 67\% reduction in estimation error demonstrates that the theoretical advantages of our framework lead to tangible improvements in diagnostic utility.
\end{itemize}

\section*{Future Directions}

This work establishes a foundation for several avenues of future research:

\begin{itemize}
	\item \textbf{Expanded Benchmarking:} While we outperformed YOLOv8, future studies should test against a wider range of architectures, such as Detection Transformers (DETR) or EfficientDet, to verify if these structural advantages hold across different deep learning paradigms.
	\item \textbf{Offline-First Tool Development:} Internet connectivity is limited in many malaria-endemic regions. A critical next step is optimizing the model for edge devices to create a standalone, offline mobile application that delivers real-time diagnostics without reliance on cloud infrastructure.
	\item \textbf{Domain Expansion:} The MTTL framework is not limited to malaria. With careful task selection, it could be adapted to other pathologies, medical imaging domains and even other domains.
	\item \textbf{Data Diversity and Generalization:} To confirm true clinical robustness, future work must validate these models on multi-center datasets. This includes testing on different malaria species, such as \textit{P. vivax}, varied staining protocols, and thick blood smears.
	\item \textbf{Advanced Multi-Task Architectures:} While our shared backbone approach proved effective, exploring advanced architectures like cross-stitch networks or gradient surgery could help mitigate the negative transfer observed with conflicting tasks.
	\item \textbf{Clinical Deployment:} Beyond algorithmic improvements, the next logical step is to evaluate this system in a real-world workflow. Assessing its impact in low-resource settings, where expert microscopists are scarce, would be the ultimate test of its value.
\end{itemize}

\noindent In conclusion, this study demonstrates that Multi-Task Transfer Learning is a powerful paradigm for automated malaria diagnosis. By learning to perform complementary tasks simultaneously, the model develops richer representations that are more accurate, robust, and clinically useful than those learned in isolation. This work offers a clear path toward scalable, reliable AI systems capable of supporting global health initiatives.
\cleardoublepage

%REFERENCES
\cleardoublepage
\chapter*{References}                    
\addcontentsline{toc}{chapter}{References} 
\printbibliography[heading=none]         

% ===================================================================
% APPENDIX 
% ===================================================================

\appendix

\chapter{Experimental Configurations}
\label{app:configs}

This appendix details the configurations for our Single-Task Learning (STL) and Multi-Task Transfer Learning (MTTL) models.

\section{Training and Optimization Strategy}

We applied a \textbf{Hybrid tuning strategy} across all models. The backbone remained frozen except for the final residual block (\texttt{layer4}), which was fine-tuned at a low learning rate ($1 \times 10^{-6}$) to refine high-level features. Task-specific heads and LoRA adapters were trained at a higher rate ($1 \times 10^{-3}$).\\

\noindent Optimization utilized \textbf{AdamW} with a weight decay of $1 \times 10^{-4}$ and \textbf{Automatic Mixed Precision (AMP)}, following a linear warmup and cosine annealing schedule. For model selection, STL training stopped when the primary validation metric plateaued. For MTTL, we used a \textbf{composite score} of weighted task metrics to guide early stopping. We implemented a dual checkpointing system: the backbone was saved based on the best composite score, while individual heads were saved whenever they achieved their specific peak performance.

\section{Model Configurations}

Table~\ref{tab:stl_configs_final} lists the best STL configurations. Table~\ref{tab:mttl_configs_final} outlines the consistent setup used for all MTTL experiments, while Table~\ref{tab:mttl_detailed} provides the specific schedules and weighting schemes for each combination.

\begin{table}[htbp]
	\centering
	\caption{Single-Task Learning (STL) Champion Model Configurations.}
	\label{tab:stl_configs_final}
	\begin{tabular}{l c c c c c c}
		\toprule
		\textbf{Task} & \textbf{Classes} & \textbf{Batch Size} & \textbf{Epochs} & \textbf{LR Head} & \textbf{LR Backbone} & \textbf{Warmup} \\
		\midrule
		Det           & 1 & 16 & 150 & $1 \times 10^{-3}$ & $1 \times 10^{-6}$ & 15 epochs \\
		Det           & 2 & 16 & 150 & $1 \times 10^{-3}$ & $1 \times 10^{-6}$ & 15 epochs \\
		Det           & 3 & 16 & 200 & $1 \times 10^{-3}$ & $1 \times 10^{-6}$ & 20 epochs \\
		\addlinespace
		RoI Classif  & 2 & 16 & 150 & $1 \times 10^{-3}$ & $1 \times 10^{-6}$ & 15 epochs \\
		RoI Classif  & 3 & 16 & 200 & $1 \times 10^{-3}$ & $1 \times 10^{-6}$ & 20 epochs \\
		\addlinespace
		Seg        & -- & 16 & 200 & $1 \times 10^{-3}$ & $1 \times 10^{-6}$ & 20 epochs \\
		Loc& -- & 16 & 200 & $1 \times 10^{-3}$ & $1 \times 10^{-6}$ & 15 epochs \\
		\bottomrule
	\end{tabular}
\end{table}

\begin{table}[htbp]
	\centering
	\caption{Multi-Task Transfer Learning (MTTL) Configurations.}
	\label{tab:mttl_configs_final}
	\begin{tabular}{l l}
		\toprule
		\textbf{Parameter} & \textbf{Value} \\
		\midrule
		Batch Size & 8 (with 2 gradient accumulation steps) \\
		Optimizer & AdamW \\
		Learning Rate (Heads \& Adapters) & $1 \times 10^{-3}$ \\
		Learning Rate (Backbone) & $1 \times 10^{-6}$ \\
		Scheduler & Cosine \\
		\midrule
		\multicolumn{2}{l}{\textit{Task Weighting Schemes (Primary Task Weights)}} \\
		Det + Seg & Det: 0.7, Seg: 0.3 \\
		Det + Loc & Det: 0.7, Loc: 0.3 \\
		Det + RoI (2- or 3-Class) & Det: 0.7, RoI: 0.3 \\
		Det + Seg + Loc & Det: 0.6, Seg: 0.2, Loc: 0.2 \\
		Det + Seg + RoI & Det: 0.6, Seg: 0.2, RoI: 0.2 \\
		Det + Loc + RoI & Det: 0.6, Loc: 0.2, RoI: 0.2 \\
		Det + Seg + Loc + RoI & Det: 0.5, Seg: 0.2, Loc: 0.1, RoI: 0.2 \\
		\bottomrule
	\end{tabular}
\end{table}

\begin{table}[htbp]
	\centering
	\caption{Detailed training configurations for MTTL task combinations.}
	\label{tab:mttl_detailed}
	\begin{tabular}{l l l c l}
		\toprule
		\textbf{Tasks} & \textbf{Classes} & \textbf{Epc,W} & \textbf{Patience} & \textbf{Primary Task Weights} \\
		\midrule
		\multicolumn{5}{l}{\textit{\textbf{Combinations with 1-Class Detection}}} \\
		Det + Seg                 & Det: 1          & 250, 25    & 30 & Det: 0.7, Seg: 0.3 \\
		Det + Loc             & Det: 1          & 250, 25   & 30 & Det: 0.7, Loc: 0.3 \\
		Det + Seg + Loc       & Det: 1          & 300, 30   & 30 & Det: 0.6, Seg: 0.2, Loc: 0.2 \\
		\addlinespace
		\multicolumn{5}{l}{\textit{\textbf{Combinations with 2-Class Detection / RoI}}} \\
		Det + Seg                 & Det: 2          & 325, 30   & 30 & Det: 0.7, Seg: 0.3 \\
		Det + Loc             & Det: 2          & 325, 30  & 30 & Det: 0.7, Loc: 0.3 \\
		Det + RoI                 & Det: 2, RoI: 2  & 325, 30  & 30 & Det: 0.7, RoI: 0.3 \\
		Det + Seg + Loc       & Det: 2          & 375, 30   & 35 & Det: 0.6, Seg: 0.2, Loc: 0.2 \\
		Det + Seg + RoI           & Det: 2, RoI: 2  & 375, 30   & 35 & Det: 0.6, Seg: 0.2, RoI: 0.2 \\
		Det + Loc + RoI       & Det: 2, RoI: 2  & 375, 30  & 35 & Det: 0.6, Loc: 0.2, RoI: 0.2 \\
		Det + Seg + Loc + RoI & Det: 2, RoI: 2  & 400, 40  & 40 & Det: 0.5, Seg: 0.2, Loc: 0.1, RoI: 0.2 \\
		\addlinespace
		\multicolumn{5}{l}{\textit{\textbf{Combinations with 3-Class Detection / RoI}}} \\
		Det + Seg                 & Det: 3          & 375, 30   & 40 & Det: 0.7, Seg: 0.3 \\
		Det + Loc             & Det: 3          & 375, 30   & 40 & Det: 0.7, Loc: 0.3 \\
		Det + RoI                 & Det: 3, RoI: 3  & 375, 30   & 40 & Det: 0.7, RoI: 0.3 \\
		Det + Seg + Loc       & Det: 3          & 400, 40   & 40 & Det: 0.6, Seg: 0.2, Loc: 0.2 \\
		Det + Seg + RoI           & Det: 3, RoI: 3  & 400, 40   & 40 & Det: 0.6, Seg: 0.2, RoI: 0.2 \\
		Det + Loc + RoI       & Det: 3, RoI: 3  & 400, 30   & 40 & Det: 0.6, Loc: 0.2, RoI: 0.2 \\
		Det + Seg + Loc + RoI & Det: 3, RoI: 3  & 400, 40   & 40 & Det: 0.5, Seg: 0.2, Loc: 0.1, RoI: 0.2 \\
		\bottomrule
	\end{tabular}
\end{table}

\chapter{Performance Leaderboards for All Tasks}
\label{app:leaderboard}

This appendix summarizes the final performance of all STL and MTTL models. For each primary task, we compare the top STL baseline with the relevant MTTL configurations, for a direct assessment of performance and task synergy. Models are ranked by the most relevant metric F1(Inf) for detection, macro F1 for RoI classification, and Dice for segmentation and heatmap localization.

\section{Detection Performance and Synergy}
Detection was evaluated in 1-, 2-, and 3-class settings. The tables report MTTL performance relative to the strongest STL baseline, with $\Delta$ F1(Inf) highlighting the impact of multi-task learning.

\begin{table}[H]
	\centering
	\caption{MTTL results for \textbf{1-Class Detection}.}
	\label{tab:mttl_1_class_synergy}
	\begin{tabular}{l l c c c c}
		\toprule
		\textbf{Task(s)} & \textbf{Paradigm} & \textbf{mAP@50} & \textbf{F1(Inf)} & \textbf{R(Inf)} & \textbf{P(Inf)} \\
		\midrule
		\textit{Det(1) Baseline} & \textit{STL} & \textbf{0.8260} & \textbf{0.8182} & \textbf{0.8218} & \textbf{0.8146} \\
		\addlinespace
		Det(1)+Seg & MTTL & 0.7885 & 0.7654 (-6.5\%) & 0.6953 & 0.8514 \\
		Det(1)+Loc & MTTL & 0.7367 & 0.7285 (-11.0\%) & 0.7098 & 0.7483 \\
		Det(1)+Seg+Loc & MTTL & 0.7709 & 0.7497 (-8.4\%) & 0.7388 & 0.7609 \\
		\bottomrule
	\end{tabular}
\end{table}

\begin{table}[H]
	\centering
	\caption{MTTL results for \textbf{2-Class Detection}.}
	\label{tab:mttl_2_class_synergy}
	\begin{tabular}{l l c c c}
		\toprule
		\textbf{Task(s)} & \textbf{Paradigm} & \textbf{mAP@50} & \textbf{F1(Inf)} & \textbf{F1-Macro} \\
		\midrule
		\textit{Det(2) Baseline} & \textit{STL} & 0.5834 & 0.5990 & 0.9274 \\
		\addlinespace
		Det(2)+RoI(2) & MTTL & \textbf{0.6452} & \textbf{0.7810} (+30.4\%) & \textbf{0.9411} \\
		Det(2)+Seg & MTTL & 0.6281 & 0.7464 (+24.6\%) & 0.9330 \\
		Det(2)+Loc & MTTL & 0.6398 & 0.7415 (+23.8\%) & 0.9282 \\
		\addlinespace
		Det(2)+Seg+RoI(2) & MTTL & 0.6113 & 0.6976 (+16.5\%) & 0.9256 \\
		Det(2)+Seg+Loc & MTTL & 0.6025 & 0.6120 (+2.2\%) & 0.9025 \\
		Det(2)+RoI(2)+Loc & MTTL & 0.5958 & 0.6065 (+1.3\%) & 0.8958 \\
		\addlinespace
		Det(2)+Seg+RoI(2)+Loc & MTTL & 0.5872 & 0.6030 (+0.7\%) & 0.8897 \\
		\bottomrule
	\end{tabular}
\end{table}

\begin{table}[H]
	\centering
	\caption{MTTL results for \textbf{3-Class Detection}.}
	\label{tab:mttl_3_class_synergy}
	\begin{tabular}{l l c c c}
		\toprule
		\textbf{Task(s)} & \textbf{Paradigm} & \textbf{mAP@50} & \textbf{F1(Inf)} & \textbf{F1-Macro} \\
		\midrule
		\textit{Det(3) Baseline} & \textit{STL} & 0.3341 & 0.3791 & 0.8977 \\
		\addlinespace
		Det(3)+RoI(3) & MTTL & \textbf{0.7402} & \textbf{0.7710} (+103.4\%) & \textbf{0.9409} \\
		Det(3)+Seg & MTTL & 0.7013 & 0.7230 (+90.7\%) & 0.9351 \\
		Det(3)+Loc & MTTL & 0.6994 & 0.7034 (+85.6\%) & 0.9195 \\
		\addlinespace
		Det(3)+Seg+RoI(3) & MTTL & 0.7089 & 0.6662 (+75.7\%) & 0.9289 \\
		Det(3)+Loc+RoI(3) & MTTL & 0.7289 & 0.6722 (+77.3\%) & 0.9375 \\
		Det(3)+Seg+Loc & MTTL & 0.6913 & 0.5468 (+44.2\%) & 0.9362 \\
		\addlinespace
		Det(3)+Seg+RoI(3)+Loc & MTTL & 0.7324 & 0.7360 (+94.2\%) & 0.9364 \\
		\bottomrule
	\end{tabular}
\end{table}

\section{Auxiliary Task Leaderboards}
We now present the leaderboards for the auxiliary tasks to evaluate how they perform as standalone specialists (STL) versus when they are part of a larger multi-task system (MTTL).

\begin{table}[H]
	\centering
	\caption{Segmentation Performance Leaderboard (Ranked by Dice).}
	\label{tab:leaderboard_seg}
	\begin{tabular}{l l c c}
		\toprule
		\textbf{Task(s)} & \textbf{Paradigm} & \textbf{Dice} & \textbf{IoU} \\
		\midrule
		Seg & STL & \textbf{0.9656} & \textbf{0.9468} \\
		\addlinespace
		Det(2)+Seg & MTTL & 0.9575 & 0.9332 \\
		Det(3)+Seg & MTTL & 0.9564 & 0.9327 \\
		Det(3)+Seg+RoI(3)+Loc & MTTL & 0.9518 & 0.9293 \\
		Det(2)+Seg+RoI(2) & MTTL & 0.9504 & 0.9264 \\
		Det(3)+Seg+Loc & MTTL & 0.9502 & 0.9276 \\
		\bottomrule
	\end{tabular}
\end{table}

\begin{table}[H]
	\centering
	\caption{Heatmap Localization Performance Leaderboard (Ranked by Dice).}
	\label{tab:leaderboard_loc}
	\begin{tabular}{l l c c c}
		\toprule
		\textbf{Task(s)} & \textbf{Paradigm} & \textbf{Dice} & \textbf{Pearson} & \textbf{MAE} \\
		\midrule
		Loc & STL & \textbf{0.5422} & \textbf{0.7866} & \textbf{0.9719} \\
		\addlinespace
		Det(3)+Loc & MTTL & 0.5108 & 0.7680 & 0.9900 \\
		Det(1)+Seg+Loc & MTTL & 0.5081 & 0.7636 & 0.9893 \\
		Det(3)+Loc+RoI(3) & MTTL & 0.5007 & 0.7571 & 0.9891 \\
		Det(1)+Loc & MTTL & 0.4975 & 0.7541 & 0.9898 \\
		Det(2)+Loc & MTTL & 0.0557 & 0.1698 & 0.5930 \\
		Det(3)+Seg+Loc+RoI(3) & MTTL & 0.0358 & 0.1565 & 0.3221 \\
		\bottomrule
	\end{tabular}
\end{table}

\begin{table}[H]
	\centering
	\caption{RoI Classification Performance Leaderboard (Ranked by F1-Macro).}
	\label{tab:leaderboard_roi}
	\begin{tabular}{l l c c c}
		\toprule
		\textbf{Task(s)} & \textbf{Paradigm} & \textbf{Classes} & \textbf{Accuracy} & \textbf{F1-Macro} \\
		\midrule
		\multicolumn{5}{l}{\textit{\textbf{2-Class Classification}}} \\
		\addlinespace
		RoI(2) & STL & 2 & \textbf{0.9859} & \textbf{0.9029} \\
		Det(2)+RoI(2) & MTTL & 2 & 0.9832 & 0.9024 \\
		Det(2)+Seg+RoI(2) & MTTL & 2 & 0.9626 & 0.8278 \\
		\addlinespace
		\midrule
		\addlinespace
		\multicolumn{5}{l}{\textit{\textbf{3-Class Classification}}} \\
		\addlinespace
		Det(3)+RoI(3) & MTTL & 3 & \textbf{0.9802} & \textbf{0.9103} \\
		Det(3)+Seg+RoI(3)+Loc & MTTL & 3 & 0.9709 & 0.8732 \\
		Det(3)+Loc+RoI(3) & MTTL & 3 & 0.9470 & 0.8326 \\
		RoI(3) & STL & 3 & 0.9373 & 0.8047 \\
		Det(3)+Seg+RoI(3) & MTTL & 3 & 0.9296 & 0.7618 \\
		\bottomrule
	\end{tabular}
\end{table}

\noindent
\textbf{Observations:} The synergy of MTTL is highly dependent on the complexity of the primary task, in our case object detection. For simple 1-class detection, all auxiliary tasks introduce a performance penalty (ranging from $-6.5\%$ to $-11.0\%$). However, as the task shifts to the more diagnostically realistic 2-class and 3-class problems, MTTL demonstrates its power. For 2-class detection, RoI Classification provides the most significant boost ($+30.4\%$), followed by Segmentation ($+24.6\%$) and Localization ($+23.8\%$). For 3-class detection, RoI Classification becomes the most effective auxiliary task, yielding a remarkable $+103.4\%$ improvement in $F1(\text{Inf})$ over the weak STL baseline, with Segmentation also providing substantial gains ($+90.7\%$) and Localization contributing $+85.6\%$. This highlights that as the primary task becomes more challenging, the regularization and shared representations learned from auxiliary tasks become increasingly beneficial. \\

\noindent
For the auxiliary tasks themselves (Segmentation, Heatmap Localization, and RoI Classification), the specialist STL models consistently outperform their MTTL counterparts on their respective primary metrics. Segmentation achieves $\text{Dice} = 0.9656$ (STL) versus $0.9575$ (best MTTL), Heatmap Localization reaches $\text{Dice} = 0.5422$ (STL) versus $0.5108$ (best MTTL), while RoI Classification shows a notable exception: for 3-class problems, MTTL ($F1 = 0.9103$) surpasses STL ($F1 = 0.8047$) by $+13.1\%$, demonstrating that the shared representations in multi-task learning can benefit auxiliary tasks when the primary task provides complementary supervision. For 2-class RoI Classification, STL and MTTL perform nearly identically ($0.9029$ versus $0.9024$). This pattern reveals a key trade-off: while auxiliary tasks can significantly boost primary task performance (especially in complex scenarios like 3-class detection), they typically do so at the cost of their own specialist performance, as the shared backbone must find a compromise representation serving all tasks simultaneously.

\chapter{Qualitative Examples}
\label{app:qualitative}

This appendix provides qualitative examples to visually complement the quantitative leaderboards in Appendix \ref{app:leaderboard}. We compare the outputs of the best Single-Task (STL) and Multi-Task (MTTL) models against the ground truth for each core task. These visualizations offer tangible insights into the practical strengths and weaknesses of each learning paradigm.

\section{Auxiliary Segmentation and Localization}

We begin by visualizing performance on the pixel- and region-level auxiliary tasks, as their ability to provide spatial priors is critical to the success of MTTL.\\

\noindent For segmentation, Figure \ref{fig:qual_seg} compares the predicted cell masks from the best STL and MTTL models against the ground truth. While both models produce high-quality outputs, the specialist STL model's mask is visibly cleaner and aligns more precisely with the ground truth boundaries. This visual evidence supports the quantitative findings that specialist models achieve peak performance on their designated task, as the shared backbone in MTTL must find a representation that is a compromise across all tasks.

\begin{figure}[H]
	\centering
	\includegraphics[width=\textwidth]{figures/Qualitative_Seg.png}
	\caption{Qualitative comparison for Cell Segmentation.}
	\label{fig:qual_seg}
\end{figure}

\noindent Both paradigms locate infections (Figures \ref{fig:qual_heatmap1}, \ref{fig:qual_heatmap2}), but the STL maps are sharper. This reflects the trade-off where auxiliary tasks sacrifice their own precision to regularize the primary detector.

\begin{figure}[H]
	\centering
	\includegraphics[width=0.85\textwidth]{figures/Qualitative_Heatmap_Sample_nlm_0481.png}
	\caption{Qualitative comparison for Heatmap Localization.}
	\label{fig:qual_heatmap1}
\end{figure}

\begin{figure}[H]
	\centering
	\includegraphics[width=0.85\textwidth]{figures/Qualitative_Heatmap_Sample_nlm_0753.png}
	\caption{Qualitative comparison for Heatmap Localization.}
	\label{fig:qual_heatmap2}
\end{figure}

\clearpage
\section{Auxiliary RoI Classification}

To analyze the classification capabilities at a granular level, we perform inference on a curated set of ground truth cell patches. Figure \ref{fig:qual_roi3} shows the cropped cell image alongside the predicted probability distributions from the STL and MTTL models for a 3-class task. Both models demonstrate high accuracy. 

\begin{figure}[H]
	\centering
	\includegraphics[width=0.95\textwidth]{figures/Qualitative_RoI_Grid_cls3.png}
	\caption{Qualitative comparison for 3-Class RoI Classification.}
	\label{fig:qual_roi3}
\end{figure}

\clearpage
\section{Detection Performance}

Finally, we visualize the performance on the primary detection task.

\begin{figure}[H]
	\centering
	\includegraphics[width=\textwidth]{figures/Qualitative_Det_1cls.png}
	\caption{Qualitative comparison for 1-Class Detection.}
	\label{fig:qual_det1}
	\vspace{0.5em}
	\centering
	\begin{tabular}{lll}
		\tikz{\draw[lime, very thick, dashed] (0,0) -- (0.6,0);} Infected (GT) & 
		\tikz{\draw[cyan, very thick, dashed] (0,0) -- (0.6,0);} Healthy (GT) & 
		\tikz{\draw[magenta, very thick, dashed] (0,0) -- (0.6,0);} WBC (GT) \\
		\tikz{\draw[red, very thick] (0,0) -- (0.6,0);} Infected (Pred) & 
		\tikz{\draw[blue, very thick] (0,0) -- (0.6,0);} Healthy (Pred) & 
		\tikz{\draw[violet, very thick] (0,0) -- (0.6,0);} WBC (Pred) \\
	\end{tabular}
\end{figure}

\noindent In the simple 1-class setting (Figure \ref{fig:qual_det1}), the specialist STL model demonstrates excellent precision and recall. The MTTL model, however, produces several false negatives, visually confirming the negative synergy observed in the quantitative results when auxiliary tasks are applied to a simple primary objective.

\noindent As task complexity increases to the diagnostically realistic 3-class problem (Figure \ref{fig:qual_det3}), the limitations of the STL approach become stark. The specialist model struggles to generalize, missing numerous infected cells. The MTTL model, in contrast, leverages the regularizing effects of its auxiliary tasks to maintain robust performance, accurately identifying most cells from all three classes. This figure provides the core visual evidence for our work MTTL is a superior paradigm for developing comprehensive and effective diagnostic models in complex settings.

\begin{figure}[H]
	\centering
	\includegraphics[width=\textwidth]{figures/Qualitative_Det_2cls.png}
	\caption{Qualitative comparison for 2-Class Detection.}
	\label{fig:qual_det2}
	\vspace{0.5em}
	\centering
	\begin{tabular}{lll}
		\tikz{\draw[lime, very thick, dashed] (0,0) -- (0.6,0);} Infected (GT) & 
		\tikz{\draw[cyan, very thick, dashed] (0,0) -- (0.6,0);} Healthy (GT) & 
		\tikz{\draw[magenta, very thick, dashed] (0,0) -- (0.6,0);} WBC (GT) \\
		\tikz{\draw[red, very thick] (0,0) -- (0.6,0);} Infected (Pred) & 
		\tikz{\draw[blue, very thick] (0,0) -- (0.6,0);} Healthy (Pred) & 
		\tikz{\draw[violet, very thick] (0,0) -- (0.6,0);} WBC (Pred) \\
	\end{tabular}
\end{figure}

\noindent In the most challenging 3-class problem (Figure \ref{fig:qual_det3}), the limitations of the STL approach become stark. The specialist model struggles to generalize, missing numerous infected cells. The MTTL model, in contrast, leverages the regularizing effects of its auxiliary tasks to maintain robust performance, accurately identifying most cells from all three classes. This figure provides the core visual evidence for our work: MTTL is a superior paradigm for developing comprehensive and effective diagnostic models in complex settings.

\begin{figure}[H]
	\centering
	\includegraphics[width=\textwidth]{figures/Qualitative_Det_3cls.png}
	\caption{Qualitative comparison for 3-Class Detection.}
	\label{fig:qual_det3}
	\vspace{0.5em}
	\centering
	\begin{tabular}{lll}
		\tikz{\draw[lime, very thick, dashed] (0,0) -- (0.6,0);} Infected (GT) & 
		\tikz{\draw[cyan, very thick, dashed] (0,0) -- (0.6,0);} Healthy (GT) & 
		\tikz{\draw[magenta, very thick, dashed] (0,0) -- (0.6,0);} WBC (GT) \\
		\tikz{\draw[red, very thick] (0,0) -- (0.6,0);} Infected (Pred) & 
		\tikz{\draw[blue, very thick] (0,0) -- (0.6,0);} Healthy (Pred) & 
		\tikz{\draw[violet, very thick] (0,0) -- (0.6,0);} WBC (Pred) \\
	\end{tabular}
\end{figure}

\end{document}