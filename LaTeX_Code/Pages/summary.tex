\begin{otherlanguage}{french}	
\noindent \textbf{Contexte et Motivation.}  
Le paludisme demeure l'une des maladies infectieuses les plus persistantes au monde et continue d'affecter des millions de personnes chaque année, particulièrement dans les régions tropicales comme l'Afrique subsaharienne. L'examen microscopique des frottis sanguins est toujours considéré comme la méthode de référence pour le diagnostic, car il permet la visualisation directe des parasites. Cependant, ce processus est lent, nécessite des techniciens hautement qualifiés et peut produire des résultats incohérents dans les environnements à faibles ressources. Ces dernières années, les méthodes d'apprentissage automatique ont été explorées pour soutenir et automatiser le diagnostic du paludisme. Malgré des progrès prometteurs, la plupart des modèles ont été développés pour des objectifs très spécifiques tels que la détection binaire de cellules infectées ou la classification des parasites, sans considérer le processus diagnostique dans son ensemble.\\

\noindent \textbf{Problématique et Objectifs.}  
Cette observation révèle une lacune importante dans la recherche, l'absence de modèles computationnels unifiés capables de gérer toute la complexité du diagnostic du paludisme au sein d'un même cadre. En pratique, le diagnostic implique plusieurs étapes interdépendantes telles que l'identification des cellules infectées, la segmentation de leurs contours, la localisation du parasite et l'estimation du niveau d'infection. Les approches existantes traitent généralement ces tâches séparément, ce qui limite leur robustesse lorsqu'elles sont appliquées à de nouvelles données ou à des conditions cliniques variées. L'objectif de ce travail est donc de concevoir, implémenter et évaluer un cadre diagnostique complet basé sur l'\textbf{Apprentissage par Transfert Multi-Tâche (MTTL)}. Le but est de développer un système qui apprend plusieurs tâches connexes simultanément et produit des prédictions diagnostiques fiables et complètes à partir d'une seule image de frottis sanguin.\\

\noindent \textbf{Méthodologie.}  
Nous commençons par définir un cadre mathématique combinant trois concepts existants : l'\textbf{Apprentissage par Transfert (TL)}, l'\textbf{Apprentissage Multi-Tâche (MTL)} et le \textbf{Réglage Fin Efficace en Paramètres (PEFT)}. Cette base théorique fournit une méthode structurée pour adapter des modèles pré-entraînés à de multiples tâches interdépendantes. Sur la base de cette formulation, nous avons implémenté une approche hybride utilisant l'Adaptation de Rang Inférieur (LoRA) conjointement avec un réglage fin partiel du réseau backbone. Le système proposé apprend quatre tâches connexes à la fois : la détection de cellules, la segmentation binaire, la localisation des parasites et la classification des cellules au niveau de l'instance. Les expériences ont été menées en utilisant le jeu de données public NLM de frottis sanguins minces à \textit{P. falciparum} dans des configurations mono-tâche et multi-tâche, permettant de comparer directement les performances de l'\textbf{Apprentissage Mono-Tâche (STL)} et de l'approche \textbf{MTTL} proposée.\\

\noindent \textbf{Résultats et Analyse.}  
Les résultats montrent clairement l'avantage de la stratégie multi-tâche. Les modèles mono-tâches sont performants pour la détection simple mais perdent en stabilité lorsque le problème se complexifie. Par exemple, le score F1 pour la détection des cellules infectées chute de $0.82$ dans le cas mono-classe à $0.38$ lorsque trois classes sont considérées. Le modèle multi-tâche, régularisé par les tâches auxiliaires, reste beaucoup plus stable. De plus, nous avons comparé notre système aux détecteurs standards de l'industrie, \textbf{YOLOv8-Nano et Small}. Notre cadre a obtenu de meilleures métriques cliniques, surpassant les deux variantes YOLO avec un score F1 supérieur de plus de \textbf{30\%} sur la classe infectée. Il a également réduit l'erreur moyenne absolue (MAE) d'estimation de la parasitémie à \textbf{1.08\%}, une amélioration significative par rapport à YOLOv8-Nano ($2.48\%$) et YOLOv8-Small ($3.78\%$). L'inspection visuelle confirme également que le MTTL produit des cartes de détection plus propres et une localisation plus précise.\\

\noindent \textbf{Conclusion et Perspectives.}  
Ces résultats confirment que la combinaison de l'Apprentissage par Transfert, de l'Apprentissage Multi-Tâche et du PEFT constitue une base solide pour l'analyse complexe d'images médicales. Le modèle MTTL proposé améliore à la fois la précision diagnostique et l'interprétabilité en liant les indices visuels connexes à travers les tâches. Bien que cette étude se concentre sur la détection du paludisme, le même cadre pourrait être étendu à d'autres domaines de l'imagerie biomédicale nécessitant l'apprentissage conjoint d'objectifs liés. Les travaux futurs pourraient inclure la validation sur des ensembles de données supplémentaires, l'évaluation dans des conditions de changement de domaine et une étude théorique plus détaillée de la transférabilité des tâches.\\

\noindent
\textbf{Mots-clés :} Diagnostic du Paludisme, Apprentissage Multi-Tâche, Apprentissage par Transfert, Apprentissage Profond, PEFT, LoRA, Pathologie Computationnelle\\

\noindent
\textbf{Dépôt de Code :} \href{https://github.com/GwadeSteve/MTTL-Research-Malaria}{https://github.com/GwadeSteve/MTTL-Research-Malaria}
\end{otherlanguage}