\thispagestyle{plain}

\noindent \textbf{Context and Motivation.}  
Malaria remains one of the most persistent infectious diseases in the world and continues to affect millions of people each year, particularly in tropical regions such as Sub-Saharan Africa. Microscopic examination of thin blood smears is still considered the reference method for diagnosis because it allows direct visualization of parasites. However, this process is slow, requires highly skilled technicians, and can produce inconsistent results in low-resource environments. In recent years, machine learning methods have been explored to support and automate malaria diagnosis. Despite promising progress, most models have been developed for very specific objectives such as binary detection of infected cells or parasite classification, without considering the complete diagnostic process.\\

\noindent \textbf{Problem Statement and Objectives.}  
This observation reveals an important research gap, the lack of unified computational models able to handle the full complexity of malaria diagnosis within a single framework. In practice, diagnosis involves several interrelated steps such as identifying infected cells, segmenting their contours, locating the parasite, and estimating the level of infection. Existing approaches usually address these tasks separately, which limits their robustness when applied to new data or varied clinical conditions. The objective of this work is therefore to design, implement, and evaluate a comprehensive diagnostic framework based on \textbf{Multi-Task Transfer Learning (MTTL)}. The goal is to develop a system that learns several related tasks at the same time and produces reliable and complete diagnostic predictions from a single blood smear image.\\

\noindent \textbf{Methodology.}  
We begin by defining a mathematical framework that combines three existing concepts: \textbf{Transfer Learning (TL)}, \textbf{Multi-Task Learning (MTL)}, and \textbf{Parameter-Efficient Fine-Tuning (PEFT)}. This theoretical basis provides a structured way to adapt pre-trained models to multiple interdependent tasks. Based on this formulation, we implemented a hybrid approach using Low-Rank Adaptation (LoRA) together with partial fine-tuning of the backbone network. The proposed system learns four related tasks at once: cell detection, binary cell segmentation, parasite localization, and instance-level cell classification. Experiments were conducted using the public NLM thin falciparum blood smear dataset in both single-task and multi-task configurations, making it possible to directly compare the performance of \textbf{Single-Task Learning (STL)} and the proposed \textbf{MTTL} approach.\\

\noindent \textbf{Results and Analysis.}  
The results clearly show the advantage of the multi-task strategy. Single-task models perform well on simple detection but lose stability when the problem becomes more complex. For example, the F1 score for detecting infected cells decreases from $0.82$ in the single-class case to $0.38$ when three classes are considered. The multi-task model, regularized by auxiliary tasks, remains much more stable. We also benchmarked our system against \textbf{YOLOv8-Small}, a standard model for real-time detection. Our framework achieved better clinical metrics, improving the infected-class F1-score by \textbf{34.7\%} and reducing the parasitemia estimation error by \textbf{71.4\%}. Visual inspection further confirms that MTTL produces cleaner detection maps, fewer false positives on healthy slides, and accurate localization.\\

\noindent \textbf{Conclusion and Perspectives.}  
These findings confirm that combining Transfer Learning, Multi-Task Learning, and Parameter-Efficient Fine-Tuning forms a strong foundation for complex medical image analysis. The proposed MTTL model improves both diagnostic accuracy and interpretability by linking related visual cues across tasks. Although this study focuses on malaria detection, the same framework could be extended to other biomedical imaging domains that require the joint learning of related objectives. Future work may include validation on additional datasets, evaluation under domain shift conditions, and a more detailed theoretical study of task transferability and generalization.\\

\noindent
\textbf{Keywords:} Malaria Diagnosis, Multi-Task Learning, Transfer Learning, Deep Learning, Parameter-Efficient Fine-Tuning, LoRA, Computational Pathology\\

\noindent
\textbf{Code Repository:} \href{https://github.com/GwadeSteve/MTTL-Research-Malaria}{https://github.com/GwadeSteve/MTTL-Research-Malaria}