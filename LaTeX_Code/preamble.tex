% -------------------------------
% PAGE GEOMETRY
% -------------------------------
\usepackage[a4paper,tmargin=2.0cm,bmargin=2.0cm,rmargin=2cm,lmargin=2cm]{geometry}

\usepackage{float}
\usepackage{subcaption}

% --- STANDARD PACKAGES ---
\usepackage{amssymb, amsfonts}
\usepackage{graphicx}
\usepackage{caption}
\usepackage{subcaption}

\usepackage{algpseudocode}
\usepackage{csquotes}
\usepackage{xcolor}
\definecolor{thesisblue}{RGB}{20, 68, 106}
\definecolor{thesisgray}{RGB}{80, 80, 80}
\usepackage{lmodern}
\usepackage{array}
\usepackage{booktabs}  % For professional-looking tables
\usepackage{tabularx}

\usepackage{enumitem}  % For customized lists (e.g., itemize, enumerate)
\usepackage{colortbl}
\usepackage{lscape}    % Optional: For landscape tables if the table is extremely wide

\usepackage{pgfplots}  % For plotting functions
\usepgfplotslibrary{groupplots}
\pgfplotsset{compat=1.18} 
\usepackage{float}

\usepackage{etoolbox}
\usepackage{tabularray}

% --- TIKZ PACKAGES & LIBRARIES ---
\usepackage{tikz}
\usetikzlibrary{
	positioning,
	fit,
	arrows.meta,
	shapes.geometric,
	calc,
	matrix,
	patterns,
	shadows,
	backgrounds,
	decorations.pathreplacing,
	shapes.misc
}

% --- CUSTOM TIKZ STYLES (plotneuralnet inspired) ---
\tikzset{
	% For 3D-like convolutional layers
	cuboid/.style={
		shape=trapezium,
		trapezium left angle=70,
		trapezium right angle=-70,
		shape border rotate=90,
		minimum width=2.2cm,
		minimum height=1cm,
		draw, thick,
		align=center,
		font=\small
	},
	% Default cuboid color (Frozen)
	cuboidfill/.style={
		fill=blue!20
	},
	% Style for trainable components (sets cuboid color to orange)
	trainable/.style={
		fill=orange!30
	},
	% Style for generic blocks
	block/.style={
		rectangle, draw, thick,
		minimum width=3cm, minimum height=1.5cm, align=center,
		rounded corners=3pt
	},
	% Standard block for operations
	op/.style={
		circle, draw, thick, fill=gray!10, minimum size=0.8cm
	},
	% Main arrows for data flow
	arrow/.style={
		-Stealth, thick, draw=black!80
	},
	% Orange LoRA arrows
	loraarrow/.style={
		-Stealth, thick, draw=orange!80!black
	},
	% Dashed highlight around LoRA path
	lorabox/.style={
		draw=orange!70!black, dashed, thick,
		inner sep=6pt, rounded corners
	},
	% Title for a diagram section
	blocktitle/.style={
		font=\sffamily\bfseries\small, text=black!90
	}, 
	outnode/.style={
		rectangle, draw, fill=blue!10, rounded corners, minimum height=1cm
	}
}

% -------------------------------
% FONTS & TYPOGRAPHY
% -------------------------------
\renewcommand{\familydefault}{\rmdefault} 
\usepackage{mathpazo}                   
\usepackage{microtype}          
\usepackage{setspace}               
\setstretch{1.2}                          

% -------------------------------
% CHAPTER & SECTION STYLING
% -------------------------------
\usepackage[Lenny]{fncychap}
\usepackage{titlesec}
\titleformat*{\section}{\Large\bfseries\sffamily}       
\titleformat*{\subsection}{\large\bfseries\sffamily}    
\titleformat*{\subsubsection}{\large\bfseries\sffamily} 
\titleformat{\chapter}[display]{\sf\bfseries\LARGE}{\vspace{-10ex}
	\filleft\MakeUppercase{\chaptertitlename}~
	\Huge\thechapter}{4ex}{\titlerule\vspace{2ex}\filright}[\vspace{2ex}\titlerule]

% -------------------------------
% HEADERS & FOOTERS
% -------------------------------
\usepackage{fancyhdr}
\pagestyle{fancy}
\fancyfoot[LE,RO]{\tiny Master of Science in Data Science and Artificial Intelligence}
\rhead{}
\lfoot{\footnotesize\tiny\textbf{Gwade Steve}}
\rfoot{\footnotesize\tiny\textbf{MTTL and Application to Malaria Detection}}

% -------------------------------
% INPUT ENCODING & LANGUAGE
% -------------------------------
\usepackage[utf8]{inputenc}
\usepackage[english]{babel}
\usepackage{verbatim}
\usepackage{url}
\usepackage{lipsum}
\usepackage{csquotes}

% -------------------------------
% MATH & SYMBOLS
% -------------------------------
\usepackage{calc,amsfonts,amssymb,amsthm,latexsym}
\usepackage{tocloft}
\usepackage[centertags]{amsmath}

% -------------------------------
% LIST OF EQUATIONS
% -------------------------------
\newcommand{\listequationsname}{List of Equations}
\newlistof{myequations}{equ}{\listequationsname}

% Command to add an equation caption to the list
\newcommand{\myequations}[1]{%
	\addcontentsline{equ}{myequations}{\protect\numberline{\theequation}#1}}


% -------------------------------
% GRAPHICS & OTHER PACKAGES
% -------------------------------
\usepackage{fancybox,wrapfig,exscale,colortbl,tcolorbox,frcursive,empheq,multicol,varwidth,lipsum}
\usepackage{pgf,tikz}
\usetikzlibrary{arrows,calc}
\usepackage{eso-pic,bclogo}
\usepackage[all]{xy}

% -------------------------------
% HYPERLINKS
% -------------------------------
\usepackage{hyperref}
\hypersetup{
	colorlinks=true,
	linkcolor=black,
	citecolor=teal,
	urlcolor=magenta,
	filecolor=orange,
	pdfauthor={Gwade Steve},
	pdftitle={Multi-Task Transfer Learning – Malaria Detection}
}

% -------------------------------
% BIBLIOGRAPHY / REFERENCES
% -------------------------------
\usepackage[backend=biber, style=numeric, sorting=none]{biblatex} 
\addbibresource{references.bib}

% -------------------------------
% ABBREVIATIONS & ALGORITHMS
% -------------------------------
\usepackage{acro}
\usepackage{longtable} 
\usepackage{multicol}

% Algorithms
\usepackage[ruled,vlined]{algorithm2e}
\SetKwInput{KwInput}{Input}
\SetKwInput{KwOutput}{Output}

% All abbreviations
% ACRONYMS LIST
\DeclareAcronym{AI}{short=AI,long=Artificial Intelligence}
\DeclareAcronym{AMP}{short=AMP,long=Automatic Mixed Precision}
\DeclareAcronym{ANN}{short=ANN,long=Artificial Neural Network}
\DeclareAcronym{AUC}{short=AUC,long=Area Under the Curve}
\DeclareAcronym{AUROC}{short=AUROC,long=Area Under the Receiver Operating Characteristic Curve}
\DeclareAcronym{API}{short=API,long=Application Programming Interface}
\DeclareAcronym{BCE}{short=BCE,long=Binary Cross-Entropy}
\DeclareAcronym{BiLSTM}{short=BiLSTM,long=Bidirectional Long Short-Term Memory}
\DeclareAcronym{BERT}{short=BERT,long=Bidirectional Encoder Representations from Transformers}
\DeclareAcronym{BN}{short=BN,long=Batch Normalization}
\DeclareAcronym{CNN}{short=CNN,long=Convolutional Neural Network}
\DeclareAcronym{CV}{short=CV,long=Computer Vision}
\DeclareAcronym{CT}{short=CT,long=Computed Tomography}
\DeclareAcronym{DA}{short=DA,long=Domain Adaptation}
\DeclareAcronym{DBN}{short=DBN,long=Deep Belief Network}
\DeclareAcronym{DNN}{short=DNN,long=Deep Neural Network}
\DeclareAcronym{DL}{short=DL,long=Deep Learning}
\DeclareAcronym{DR}{short=DR,long=Diabetic Retinopathy}
\DeclareAcronym{ECG}{short=ECG,long=Electrocardiogram}
\DeclareAcronym{EEG}{short=EEG,long=Electroencephalogram}
\DeclareAcronym{ELMo}{short=ELMo,long=Embeddings from Language Models}
\DeclareAcronym{FC}{short=FC, long=Fully Connected}
\DeclareAcronym{FP}{short=FP,long=False Positive}
\DeclareAcronym{FN}{short=FN,long=False Negative}
\DeclareAcronym{F1}{short=F1,long=F1-Score}
\DeclareAcronym{FPN}{short=FPN,long=Feature Pyramid Network}
\DeclareAcronym{FPR}{short=FPR,long=False Positive Rate}
\DeclareAcronym{GAN}{short=GAN,long=Generative Adversarial Network}
\DeclareAcronym{GRU}{short=GRU,long=Gated Recurrent Unit}
\DeclareAcronym{GPU}{short=GPU,long=Graphics Processing Unit}
\DeclareAcronym{IoU}{short=IoU,long=Intersection over Union}
\DeclareAcronym{KNN}{short=KNN,long=k-Nearest Neighbors}
\DeclareAcronym{LBP}{short=LBP,long=Local Binary Pattern}
\DeclareAcronym{LDA}{short=LDA,long=Linear Discriminant Analysis}
\DeclareAcronym{LIME}{short=LIME,long=Local Interpretable Model-agnostic Explanations}
\DeclareAcronym{LSTM}{short=LSTM,long=Long Short-Term Memory}
\DeclareAcronym{LoRA}{short=LoRA,long=Low-Rank Adaptation}
\DeclareAcronym{MAE}{short=MAE,long=Mean Absolute Error}
\DeclareAcronym{mAP}{short=mAP,long=mean Average Precision}
\DeclareAcronym{MSE}{short=MSE,long=Mean Squared Error}
\DeclareAcronym{ML}{short=ML,long=Machine Learning}
\DeclareAcronym{MTL}{short=MTL,long=Multi-Task Learning}
\DeclareAcronym{MTTL}{short=MTTL,long=Multi-Task Transfer Learning}
\DeclareAcronym{MRI}{short=MRI,long=Magnetic Resonance Imaging}
\DeclareAcronym{NLP}{short=NLP,long=Natural Language Processing}
\DeclareAcronym{NLM}{short=NLM,long=National Library of Medecine}
\DeclareAcronym{NIH}{short=NIH,long=National Institutes of Health}
\DeclareAcronym{PCA}{short=PCA,long=Principal Component Analysis}
\DeclareAcronym{PDF}{short=PDF,long=Probability Density Function}
\DeclareAcronym{PEFT}{short=PEFT,long=Parameter-Efficient Fine-Tuning}
\DeclareAcronym{PSNR}{short=PSNR,long=Peak Signal-to-Noise Ratio}
\DeclareAcronym{RBC}{short=RBC,long=Red Blood Cell}
\DeclareAcronym{RDT}{short=RDT,long=Rapid Diagnostic Test}
\DeclareAcronym{ReLU}{short=ReLU,long=Rectified Linear Unit}
\DeclareAcronym{ResNet}{short=ResNet,long=Residual Network}
\DeclareAcronym{RoI}{short=RoI,long=Region of Interest}
\DeclareAcronym{RPN}{short=RPN,long=Region Proposal Network}
\DeclareAcronym{ROC}{short=ROC,long=Receiver Operating Characteristic}
\DeclareAcronym{SGD}{short=SGD,long=Stochastic Gradient Descent}
\DeclareAcronym{STL}{short=STL,long=Single-Task Learning}
\DeclareAcronym{SVM}{short=SVM,long=Support Vector Machine}
\DeclareAcronym{TL}{short=TL,long=Transfer Learning}
\DeclareAcronym{TP}{short=TP,long=True Positive}
\DeclareAcronym{TN}{short=TN,long=True Negative}
\DeclareAcronym{TPR}{short=TPR,long=True Positive Rate}
\DeclareAcronym{TPU}{short=TPU,long=Tensor Processing Unit}
\DeclareAcronym{VGG}{short=VGG,long=Visual Geometry Group Network}
\DeclareAcronym{ViT}{short=ViT,long=Vision Transformer}
\DeclareAcronym{WBC}{short=WBC,long=White Blood Cell}
\DeclareAcronym{WHO}{short=WHO,long=World Health Organization}
\DeclareAcronym{XAI}{short=XAI,long=Explainable Artificial Intelligence}
\DeclareAcronym{YOLO}{short=YOLO,long=You Only Look Once}
\DeclareAcronym{DenseNet}{short=DenseNet,long=Densely Connected Convolutional Network}
\DeclareAcronym{Incep}{short=Incep,long=Inception Network}
\DeclareAcronym{UNet}{short=UNet,long=U-Net Convolutional Network}
\DeclareAcronym{Optuna}{short=Optuna,long=Hyperparameter Optimization Framework}
\DeclareAcronym{FasterRCNN}{short=FasterRCNN,long=Faster Region-Based Convolutional Neural Network}
\DeclareAcronym{MMDetection}{short=MMDetection,long=OpenMMLab Detection Toolbox}
\DeclareAcronym{COCO}{short=COCO,long=Common Objects in Context Dataset}
\DeclareAcronym{ImageNet}{short=ImageNet,long=ImageNet Large Scale Visual Recognition Dataset}
\DeclareAcronym{GANomaly}{short=GANomaly,long=Generative Adversarial Anomaly Detection}
\DeclareAcronym{R-CNN}{short=R-CNN,long=Region-Based Convolutional Neural Network}
\DeclareAcronym{SSD}{short=SSD,long=Single Shot Detector}
\DeclareAcronym{ViViT}{short=ViViT,long=Video Vision Transformer}
\DeclareAcronym{MixUp}{short=MixUp,long=Data Augmentation Technique}
\DeclareAcronym{CutMix}{short=CutMix,long=Data Augmentation Technique}
\DeclareAcronym{SWIN}{short=SWIN,long=Shifted Window Transformer}
\DeclareAcronym{SAM}{short=SAM,long=Segment Anything Model}
\DeclareAcronym{CLIP}{short=CLIP,long=Contrastive Language-Image Pretraining}
\DeclareAcronym{MAE-MLM}{short=MAE-MLM,long=Masked Autoencoder with Masked Language Modeling}
\DeclareAcronym{SwinUNet}{short=SwinUNet,long=Swin-UNet Network for Segmentation}
\DeclareAcronym{GradCAM}{short=GradCAM,long=Gradient-weighted Class Activation Mapping}
\DeclareAcronym{Dice}{short=Dice,long=Dice Similarity Coefficient}
\DeclareAcronym{IoBB}{short=IoBB,long=Intersection over Bounding Box}
\DeclareAcronym{Ablation}{short=Ablation,long=Ablation Study}
\DeclareAcronym{mDice}{short=mDice,long=Mean Dice Score}
\DeclareAcronym{FocalLoss}{short=FocalLoss,long=Focal Loss Function}
\DeclareAcronym{DiceLoss}{short=DiceLoss,long=Dice Loss Function}
\DeclareAcronym{AdamW}{short=AdamW,long=Adam Weight Decay Optimizer}
\DeclareAcronym{SGDR}{short=SGDR,long=Stochastic Gradient Descent with Restarts}
\DeclareAcronym{TTA}{short=TTA,long=Test Time Augmentation}
\DeclareAcronym{EMA}{short=EMA,long=Exponential Moving Average}
\DeclareAcronym{LR}{short=LR,long=Learning Rate}
\DeclareAcronym{CSAM}{short=CSAM,long=Channel Spatial Attention Module}
\DeclareAcronym{PSP}{short=PSP,long=Pyramid Scene Parsing Network}
\DeclareAcronym{DeepLabV3}{short=DeepLabV3,long=DeepLabV3 Semantic Segmentation Network}
\DeclareAcronym{MS-SSIM}{short=MS-SSIM,long=Multi-Scale Structural Similarity Index}
\DeclareAcronym{HPA}{short=HPA,long=Human Protein Atlas}
\DeclareAcronym{TCGA}{short=TCGA,long=The Cancer Genome Atlas}
\DeclareAcronym{FISH}{short=FISH,long=Fluorescence In Situ Hybridization}
\DeclareAcronym{PBS}{short=PBS,long=Phosphate Buffered Saline}
\DeclareAcronym{HE}{short=HE,long=Hematoxylin and Eosin Staining}
\DeclareAcronym{IFA}{short=IFA,long=Immunofluorescence Assay}
\DeclareAcronym{qPCR}{short=qPCR,long=Quantitative Polymerase Chain Reaction}
\DeclareAcronym{NGS}{short=NGS,long=Next-Generation Sequencing}
\DeclareAcronym{CIFAR}{short=CIFAR,long=CIFAR Image Classification Dataset}
\DeclareAcronym{MNIST}{short=MNIST,long=Modified National Institute of Standards and Technology Dataset}
\DeclareAcronym{BloodSmear}{short=BloodSmear,long=Peripheral Blood Smear}
\DeclareAcronym{PBSM}{short=PBSM,long=Peripheral Blood Smear Microscopy}
\DeclareAcronym{ParasiteDensity}{short=PD,long=Parasite Density Measurement}
\DeclareAcronym{LifeCycleStage}{short=LCS,long=Parasite Life-Cycle Stage}
\DeclareAcronym{DataAug}{short=DataAug,long=Data Augmentation}
\DeclareAcronym{RandAug}{short=RandAug,long=Random Augmentation}
\DeclareAcronym{SegmMask}{short=SegmMask,long=Segmentation Mask}
\DeclareAcronym{TrainSet}{short=TrainSet,long=Training Dataset}
\DeclareAcronym{ValSet}{short=ValSet,long=Validation Dataset}
\DeclareAcronym{TestSet}{short=TestSet,long=Testing Dataset}
\DeclareAcronym{MLOps}{short=MLOps,long=Machine Learning Operations}
\DeclareAcronym{Docker}{short=Docker,long=Docker Container Platform}
\DeclareAcronym{K8s}{short=K8s,long=Kubernetes}
\DeclareAcronym{CI}{short=CI,long=Continuous Integration}
\DeclareAcronym{CD}{short=CD,long=Continuous Deployment}
\DeclareAcronym{MLFlow}{short=MLFlow,long=Machine Learning Lifecycle Management Platform}
\DeclareAcronym{OpticalFlow}{short=OF,long=Optical Flow}
\DeclareAcronym{SIFT}{short=SIFT,long=Scale-Invariant Feature Transform}
\DeclareAcronym{SURF}{short=SURF,long=Speeded Up Robust Features}
\DeclareAcronym{HOG}{short=HOG,long=Histogram of Oriented Gradients}
\DeclareAcronym{ORB}{short=ORB,long=Oriented FAST and Rotated BRIEF}
\DeclareAcronym{YOLOv5}{short=YOLOv5,long=You Only Look Once Version 5}
\DeclareAcronym{YOLOv8}{short=YOLOv8,long=You Only Look Once Version 8}
\DeclareAcronym{FPNLite}{short=FPNLite,long=Lightweight Feature Pyramid Network}
\DeclareAcronym{CIF}{short=CIF,long=Canadian Institute for Forestry} % Example of extra
\DeclareAcronym{TPM}{short=TPM,long=Transcripts Per Million}
\DeclareAcronym{GNN}{short=GNN,long=Graph Neural Network}
\DeclareAcronym{MPNN}{short=MPNN,long=Message Passing Neural Network}
\DeclareAcronym{EdgeAI}{short=EdgeAI,long=Edge Artificial Intelligence}
\DeclareAcronym{TinyML}{short=TinyML,long=Tiny Machine Learning}
\DeclareAcronym{FPGA}{short=FPGA,long=Field-Programmable Gate Array}
\DeclareAcronym{ASIC}{short=ASIC,long=Application-Specific Integrated Circuit}
\DeclareAcronym{RL}{short=RL,long=Reinforcement Learning}
\DeclareAcronym{PPO}{short=PPO,long=Proximal Policy Optimization}
\DeclareAcronym{SAC}{short=SAC,long=Soft Actor-Critic}
\DeclareAcronym{DQN}{short=DQN,long=Deep Q-Network}
\DeclareAcronym{HER}{short=HER,long=Hindsight Experience Replay}
\DeclareAcronym{MCTS}{short=MCTS,long=Monte Carlo Tree Search}
\DeclareAcronym{OCT}{short=OCT,long=Optical Coherence Tomography}
\DeclareAcronym{fMRI}{short=fMRI,long=functional Magnetic Resonance Imaging}
\DeclareAcronym{DTI}{short=DTI,long=Diffusion Tensor Imaging}
\DeclareAcronym{EEGNet}{short=EEGNet,long=Deep Learning Network for EEG Signals}
\DeclareAcronym{EMG}{short=EMG,long=Electromyography}
\DeclareAcronym{PET}{short=PET,long=Positron Emission Tomography}
\DeclareAcronym{SPECT}{short=SPECT,long=Single Photon Emission Computed Tomography}
\DeclareAcronym{CAD}{short=CAD,long=Computer-Aided Diagnosis}
\DeclareAcronym{RIS}{short=RIS,long=Radiology Information System}
\DeclareAcronym{PACS}{short=PACS,long=Picture Archiving and Communication System}
\DeclareAcronym{DICOM}{short=DICOM,long=Digital Imaging and Communications in Medicine}
\DeclareAcronym{HL7}{short=HL7,long=Health Level 7 Standard}
\DeclareAcronym{FHIR}{short=FHIR,long=Fast Healthcare Interoperability Resources}
\DeclareAcronym{LIFEx}{short=LIFEx,long=Radiomics Feature Extraction Software}
\DeclareAcronym{PyRadiomics}{short=PyRad,long=Python Radiomics Library}
\DeclareAcronym{VAE}{short=VAE,long=Variational Autoencoder}
\DeclareAcronym{CVAE}{short=CVAE,long=Conditional Variational Autoencoder}
\DeclareAcronym{cGAN}{short=cGAN,long=Conditional Generative Adversarial Network}
\DeclareAcronym{InfoGAN}{short=InfoGAN,long=Information Maximizing GAN}
\DeclareAcronym{SRGAN}{short=SRGAN,long=Super-Resolution GAN}
\DeclareAcronym{SR}{short=SR,long=Super-Resolution}
\DeclareAcronym{ESRGAN}{short=ESRGAN,long=Enhanced SRGAN}
\DeclareAcronym{MUNIT}{short=MUNIT,long=Multimodal Unsupervised Image-to-Image Translation}
\DeclareAcronym{CycleGAN}{short=CycleGAN,long=Cycle-Consistent GAN}
\DeclareAcronym{Pix2Pix}{short=Pix2Pix,long=Paired Image-to-Image Translation GAN}
\DeclareAcronym{UDA}{short=UDA,long=Unsupervised Domain Adaptation}
\DeclareAcronym{FDA}{short=FDA,long=Fourier Domain Adaptation}
\DeclareAcronym{WD}{short=WD,long=Weight Decay}
\DeclareAcronym{SWAG}{short=SWAG,long=Stochastic Weight Averaging Gaussian}
\DeclareAcronym{LaplaceApprox}{short=LapApprox,long=Laplace Approximation}
\DeclareAcronym{MC-Dropout}{short=MCD,long=Monte Carlo Dropout}
\DeclareAcronym{KD}{short=KD,long=Knowledge Distillation}
\DeclareAcronym{SOTA}{short=SOTA,long=State-of-the-Art}
\DeclareAcronym{PoissonLoss}{short=PoissonLoss,long=Poisson Negative Log-Likelihood Loss}
\DeclareAcronym{Bbox}{short=Bbox,long=Bounding Box}
\DeclareAcronym{Heatmap}{short=Heatmap,long=Localization Heatmap}
\DeclareAcronym{WSI}{short=WSI,long=Whole Slide Imaging}
\DeclareAcronym{CLAHE}{short=CLAHE,long=Contrast Limited Adaptive Histogram Equalization}


\acsetup{
	make-links = true,
	first-style = long-short,
	list/name = {List of Acronyms},
	list/heading = none,
	list/sort = true,
	list/template = description,
	pages/display = first
}

% -------------------------------
% THEOREMS
% -------------------------------
\newtheorem{exo}{{Exercice}}[chapter]
\newtheorem{remark}{{Remarque}}[chapter]
\newtheorem{definition}{Définition}[chapter]
\newtheorem{example}{{Exemple}}[chapter]